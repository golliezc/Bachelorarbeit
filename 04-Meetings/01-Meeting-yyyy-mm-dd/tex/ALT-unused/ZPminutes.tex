% ********
% Preamble
% ********

% Documentclass using KOMA-script
% ===============================

\documentclass[
  paper=a4,                         % Paper format
  fontsize=11pt,                    % Fontsize
  DIV=12,                           % Divided page horizontally and vertically
  BCOR=10mm,                        % Binding correction
  twoside=true,                     % Double or one-sided typesetting
  parskip=half,                     % Parskip
  headings=small,                   % small font size for headings
]{scrreprt}                         % KOMA - class

% Usepackages
% ===========

\usepackage[utf8]{inputenc}         % Input caracters
\usepackage[T1]{fontenc}            % Font encodings
\usepackage[english,ngerman]{babel} % Language
\usepackage{scrlayer-scrpage}               % Headers and footers
\pagestyle{scrheadings}             % Headers and footers
\usepackage{graphicx}               % Enhanced support for graphics
\usepackage{xcolor}                 % Color extensions
\usepackage{enumitem}               % Layout of itemize, enumerate, ...
\usepackage{multicol}               % Multiple columns
\usepackage{subfig}                 % Figures broken into subfigures
\usepackage{rotating}               % Rotation tools
\usepackage{longtable}              % Long tables
\usepackage{tabularx}               % Adjustable Tabulars
\usepackage{booktabs}               % Tabular rules
\usepackage{float}                  % Float environment
\usepackage{amsmath}                % AMS mathematical facilities
\usepackage{amssymb}                % AMS symbol fonts
\usepackage{makeidx}                % Index
\usepackage[intoc]{nomencl}         % Nomenclature
\usepackage{acronym}                % Acronyms
\usepackage{tikz}                   % TikZ
\usepackage{pgfplots}               % PGF plots
\usepackage{units}                  % Typeset units
%\usepackage{cite}                   % Citations
%\usepackage{natbib}                 % Citation Style
%\renewcommand{\bibsection}{}        % No auto headings
\usepackage{blindtext}              % Some random text
\usepackage{pgfgantt}               % Gantt Chart


\usepackage{minutes}
\minutesstyle{
    header   = {list}, %or {table},
    vote     = {table}, %or {list},
    contents = {true}, %or {false}
}


% Header
% ======

\automark[section]{chapter}

\usepackage{etoolbox}
\makeatletter
\patchcmd{\chapter}{\if@openright\cleardoublepage\else\clearpage\fi}{}{}{}
\makeatother

% Table rule
% ----------

% \renewcommand{\arraystretch}{2}
% \setlength{\arrayrulewidth}{1pt}
% \lightrulewidth=1pt
% \heavyrulewidth=.5pt



% Colors
% ======

\definecolor{ibiBlue}{RGB}{0,106,175}


% Hyphenations
% =============

\hyphenation{ex-am-ple hy-phen-ate}


% MISC
% ====
\raggedbottom
\makeindex




%=============================================================================

% *************
% Main Document
% *************

% \minutesstyle{
% header = {table}
% }

% Document
% ========


\begin{document}
\selectlanguage{ngerman}

\vspace*{-2cm}
  \begin{figure}[htbp]
    \begin{minipage}{0.4\textwidth}
      \begin{tikzpicture}[y=.80pt, x=.80pt,yscale=-1, inner sep=0pt, outer sep=0pt]
        \begin{scope}[cm={{1.25,0.0,0.0,-1.25,(0.0,56.687)}}]% g3012
          \begin{scope}[shift={(46.2141,45.0657)}]% g3014
            \path[fill=black,nonzero rule] (0.0000,0.0000) -- (-1.4770,-6.1240) --
            (-5.4400,-6.1240) -- (-4.1060,0.0000) -- (-41.7030,0.0000) --
            (-45.4820,-16.9610) -- (-30.4040,-16.9610) -- (-29.4190,-12.5520) --
            (-38.2440,-12.5520) -- (-37.7560,-10.4240) -- (-29.2500,-10.4240) --
            (-28.3620,-6.4280) -- (-36.7960,-6.4280) -- (-36.3390,-4.3970) --
            (-22.6580,-4.3970) -- (-25.4520,-16.9610) -- (-19.2100,-16.9640) --
            (-16.4110,-4.3970) -- (-11.1610,-4.4010) -- (-13.9640,-16.9570) --
            (-7.8940,-16.9610) -- (-6.3200,-10.4450) -- (-2.4650,-10.4450) --
            (-3.8280,-16.9610) -- (2.0630,-16.9640) -- (5.8420,0.0000) -- (0.0000,0.0000)
            -- cycle;
          \end{scope}
          \node (ger) at (0,16) [anchor=west,font=\scriptsize] {\sffamily {\textbf{Eidgenössische Technische Hochschule Zürich}}};
          \node (eng) at (0,4.5) [anchor=west,font=\scriptsize] {\sffamily{\textbf{Swiss Federal Institute of Technology Zurich}}};
        \end{scope}
      \end{tikzpicture}
    \end{minipage}\hfill
    \begin{minipage}{0.4\textwidth}
      \hfill
      \begin{tikzpicture}[y=0.80pt, x=0.8pt,yscale=-1, inner sep=0pt, outer sep=0pt]
        \begin{scope}[yshift=-2.5mm]
          \node (a1) at (0,0) [anchor=west,font=\scriptsize] {\sffamily {\textbf{Institut für Bau- und}}};
          \node (a2) at (0,12) [anchor=west,font=\scriptsize] {\sffamily {\textbf{Infrastrukturmanagement}}};
          \node (a3) at (0,24) [anchor=west,font=\scriptsize] {\sffamily {{Professur für}}};
          \node (a4) at (0,36) [anchor=west,font=\scriptsize] {\sffamily {{Infrastrukturmanagement}}};
          \node (a5) at (0,48) [anchor=west,font=\scriptsize] {\sffamily {{Prof. Dr. Bryan T. Adey}}};
        \end{scope}
        \begin{scope}[xshift=4cm,yscale=3,xscale=3]
          \foreach \x in {0,5,10}
          \foreach \y in {0,5,10}
          \fill[ibiBlue] (\x,\y) rectangle ++ (4,4);
          \fill[white] (-0.1,-0.1) rectangle ++ (4.5,4.5);
          \begin{scope}[rotate=-20]
            \fill[ibiBlue] (0,0) rectangle ++ (4,-4);
          \end{scope}
        \end{scope}
      \end{tikzpicture}
    \end{minipage}
  \end{figure}
\vspace{-7mm}
\hrulefill\\
\nopagebreak

\begin{Minutes}{Meeting Minutes Zwischenpräsentation}
%%\subtitle{}
\moderation{Prof. Dr. Bryan T. Adey}
\minutetaker{Cyrano Golliez}
\participant{Prof. Dr. Bryan T. Adey,
			 Dr. Claudio Martani,
			 Cyrano Golliez}
%\missing[with excuse]{-}
%\missingExcused{-}
%\missingNoExcuse{-}
\guest{-}
\minutesdate{8. April 2020}
\starttime{10:00}
\endtime{10:30}
\location{Zürich}
%%\cc{}
\maketitle
%\nopagebreak

% ------------- 1 Punkt

\topic{Problemlösungszyklus}
Konkretere Ziele formulieren
\begin{description}
\item Was möchte ich erreichen
\end{description}  
\subtopic{Zielfunktion}
Verknüpfung der allg. Situation mit der von mir definierten Zielfunktion
\begin{description}
\item Wahl der Zeilfunktion darlegen \\
 Mit Argumenten bekräftigen
\item Definierte Zielfunktion begründen \\
Die Zielfunktion sollte so aufgebaut sein, dass sie representativ für die Interessen aller beteiligten Parteien ist. 
\end{description}

% ------------- 2 Punkt

\topic{Step back}  
\subtopic{Qualitative Argumente}
\begin{description}
\item Verknüpfen mit den Leitzielen von Uster \\
\\ Objektive Entscheidungsfindung, so dass der Stadtrat überzeugt werden kann.
\item Was wurde im STEK angepasst? \\
\\ Was hat Einfluss auf die Menge an Veloverkehr $\Rightarrow$ Was ist geplant von der Stadt.
\end{description}

% ------------- 3 Punkt

\topic{In grösseren Kontext stellen}   
\subtopic{Alles Berücksichtigen}
Text for the topic.


\topic{Dates}
\schedule{2000/12/24}{Christmas eve}
\schedule{2000/12/24}[20:00]{distribution of presents}
\schedule*{2000/12/25}{Christmas day (without entry in calendar)}

\topic{Tasks}
\task{Who}{Action}
\task*{Somebody has to do it}
\task*[today]{Somebody do it}
\task[done]{responsible}[yesterday]{Somebody did it}


\addtopic{Additional Topic}
\subtopic{subtopic}

\topic{Attachments}
\attachment{Attachment with two pages}{2}

\topic{Decisions and votes}
\subtopic{Opinions and Argumentations}
\opinion{Herder}{Different Opinion}

A discussion to the theme:
\begin{Opinions}
\item[Goethe] One opinion
\item[Schiller] Another opinion
\end{Opinions}

Arguments can be discussed with pro and contra
\begin{Argumentation}
\pro reason for it
\Pro important reason for it
\contra reason against it
\Contra important reason against it
\item a comment
\result result
\end{Argumentation}

\subtopic{A single vote}
\vote{Short voting}{1}{2}{3}

And a single vote with decision:\par
\vote{Short voting}{1}{2}{3}[Decision]

\subtopic{A couple of votes}
\begin{Vote}
\vote{Vote one}{1}{2}{3}
\vote{Vote two}{1}{2}{3}[decision]
\vote{Vote three}{1}{2}{3}
\end{Vote}

\subtopic{Entscheidungen}
\decisiontheme{Theme}{Theme for a decision}
\decision{Theme}{Decision}
\decision*{Decision without theme}[Long text for the decision]

\topic{Gantt Chart}

\begin{figure}[h]
\begin{center}

\begin{ganttchart}[y unit title=0.4cm,
y unit chart=0.5cm,
vgrid,hgrid, 
title label anchor/.style={below=-1.6ex},
title left shift=.05,
title right shift=-.05,
title height=1,
bar/.style={fill=darkgray!70,draw},
%incomplete/.style={fill=white},
progress label text={},
bar height=0.7,
group right shift=0,
group top shift=.6,
group height=.3,
%group peaks={}{}{.2}
]{1}{20}

%labels
\gantttitle{FS2020}{20} \\
\gantttitle{Januar}{4} 
\gantttitle{Februar}{4} 
\gantttitle{März}{4} 
\gantttitle{April}{4} 
\gantttitle{Mai}{4} \\
%tasks
\ganttbar{Task 1}{1}{2} \\
\ganttbar{Task 2}{3}{8} \\
\ganttbar{Task 3}{9}{10} \\
\ganttbar[progress=80]{Task 4}{11}{13} \\
\ganttbar[progress=33]{task 5}{13}{15} \\
\ganttbar[progress=0]{task 6}{16}{19} \\
\ganttbar[progress=66]{task 7}{9}{20} \\

%relations 
\ganttlink{elem0}{elem1} 
%\ganttlink{elem0}{elem7}
\ganttlink{elem1}{elem2}
\ganttlink{elem2}{elem3} 
\ganttlink{elem3}{elem4} 
\ganttlink{elem4}{elem5} 
\ganttlink{elem5}{elem6}  
%\ganttlink{elem5}{elem7} 

\end{ganttchart}
\end{center}
\caption{Gantt Chart}
\end{figure}


\end{Minutes}

\appendix
\selectlanguage{ngerman}
\chapter{Appendix}
\section{List of decisions}\listofdecisions
\section{List of open tasks}\listoftasks
\section{List of attachments}\listofattachments

%\usepackage[authordate,backend=biber]{biblatex-chicago}  % Citation und Style
%\addbibresource{./literature.bib}
%\setlength\bibitemsep{1.5\itemsep}

\end{document}

%=============================================================================
%%% Local Variables:
%%% mode: latex
%%% TeX-master: t
%%% End: