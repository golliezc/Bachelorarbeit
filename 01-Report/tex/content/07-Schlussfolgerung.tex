%=============================================================================
% Thesis Template in LaTex
%
% File:  07-Schlussfolgerung.tex -- Schlussfolgerung und Ausblick
% Author(s): Cyrano Golliez <golliezc@student.ethz.ch>
%
% Creation:  27 Jan 2014
% Time-stamp: <Tue 2013-08-13 20:14 juergen>
%
% Copyright (c) 2014 Infrastructure Management Group (IMG)
%               http://ibi.ethz.ch
%
% More information on LaTeX: http://www.latex-project.org/
%=============================================================================

\chapter{Schlussfolgerung und Ausblick}
\label{chap:Schlussfolgerung}



Eine Aufteilung der Kosten nach den verschiedenen Interessensverbänden wäre im Rahmen der Diskussion sehr aufschlussreich gewesen. Da die die Situation und die Möglichkeiten die mir für die Modellierung zur Verfügung standen zur Folge haben, dass die Reisezeitkosten den gesamten Risikovergleich der Varianten bestimmten, habe ich im Rahmen der Diskussion der Ergebnisse auf diese Unterteilung verzichtet. Es wäre jedoch sehr interessant, im Rahmen der weiteren Prüfung der Varianten, eine vertieftere Beurteilung der Nutzer- und Besitzerkosten durchzuführen.

Der Vergleich der Reisezeitkosten ist das ausschlaggebende Argument der Variante 2. In der Diskussion wurde die Mehrkosten

Mehrkosten Reisezeit -> V1 vs V2 übersteigen höher bau und wartungskosten umeinvielfaaches.,.. somit wäre nicht nur nachhaltig und ökologisch sondern auch ökonisch die beste option.!! 

%Hier wird die Schlussfolgerung aus der Diskussion der Resultate gezogen. 

%Somit kann die geplante Veloroute eine Reduktion der Belastung der Öffentliche Hand, in Form der Reduktion der Schadstoffemission, erreichen. 	
%Die Schwierigkeit diese Kosten zu modelieren liegt in der nicht direkt messbaren Beziehung zwischen Schadstoffbelastung und daraus resultierenden Kosten. Jedoch ist zu vermerken das die Emissionen direkt von der Anzahl an motorisierten Fahrzeuge abhängt die auf der Infrastruktur verkehren.

%Das Ziel dieser Infrastruktur ist es grössere Distanzen in kurzer Zeit und mit hoher Geschwindigkeit unmotorisiert zurück legen zu könne. Dies setzt eine kreuzungsfreie Ausführung vorraus was mit baulichen Massnahmen oder Vortrittsberechtigungen erreicht werden kann.

ausblick: mögliche erweiterung der Veloinfrastruktur bis zum Nüslikreisel oder bis zum Spital... Die Infrastrukturintervention wäre ein Ausbau der Nord-Süd Achse entlang der Pfäffikerstrasse, Brunnen- und Bahnhofstrasse zu einer Veloschnellstrasse. Das Vorbild dieser Infrastruktur ist die C99, Kopenhagen-Albertslund eine sogenannten \textit{Super-Radschnellroute} auch bekannt als \textit{Cykelsuperstier}. Diese Art von Veloinfrastruktur hat sich bereits in Dänemark bewehrt um das Pendeln leichter und sicherer zu machen. 

%In Anbetracht der enormen Kapazitätssteigerung wäre ein Ausbau der Parkplätze auf der Sportanlage Buchenholz zu prüfen. Dies wäre im Rahmen eines Programmes gedacht das sich \textit{Ride and Bike} nennen könnte, welches zum Ziel hat die Kombination von Autobahnanschluss und Parkiermöglichkeit mit einer bequemen, sicheren und schnellen Veloverbindung ins Stadtzentrum zu fördern.

Schlusspunkt Schlussfolgerung -> EINFACHER E-Baum der Entscheidungsmöglichkeiten die Uster mit dieser Vorstudie nun hat...

% ===========================================================================
% EOF
%

%%% Local Variables:
%%% mode: latex
%%% TeX-master: "../main"
%%% End:
