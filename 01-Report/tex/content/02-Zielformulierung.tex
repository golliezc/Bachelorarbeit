%=============================================================================
% Thesis Template in LaTex
%
% File:  02-Zielformulierung.tex -- Basic Theory
% Author(s): Cyrano Golliez <golliezc@student.ethz.ch>>
%
% Creation:  27 Jan 2014
% Time-stamp: <Tue 2013-08-13 20:14 juergen>
%
% Copyright (c) 2014 Infrastructure Management Group (IMG)
%               http://ibi.ethz.ch
%
% More information on LaTeX: http://www.latex-project.org/
%=============================================================================

\chapter{Zielformulierung}
\label{chap:Ziel}

Die Leitziele \guillemotleft Uster macht Mobil \guillemotright und << Velostadt Uster>> die sich Uster im Rahmen der Stadtentwicklung gesetzt hat sind klare Signale, dass der Ausbau der Veloinfrastruktur für die Bevölkerung Usters ein zentrales Anliegen ist. Die Förderung des Langsamverkehrs und die Reduktion des MIV Anteils am Modal Split hat in Anbetracht der Zentrumsentwicklung höchste Priorität. Die Aufwertung des Zentrums in Anbetracht der gesetzeten Ziele aus Uster ein urbanes Regionalzentrum zu mache, wird zu einer gesteigerten Nachfrage auf der Quelle-Ziel-Beziehung zwischen Zentrum und dem umgebenden Stadteilen führen.

Das Ziel der Infrastrukturintervention sollte demnach sein der gesteigerten Nachfrage nach Mobilität Rechnung zu tragen. Die Förderung des Langsamverkehrs soll dazu führen den Modal Split der Verkehrsbeziehungen im Raum Uster zu beeinflussen. So soll im Rahmen der Mobilitätsstrategie 2035 der Anteil an Veloverkehr um 3\% steigen\footnote{\cite{STEK Kap.7}}



% ===========================================================================
% EOF
%

%%% Local Variables:
%%% mode: latex
%%% TeX-master: "../main"
%%% End:
