%=============================================================================
% Thesis Template in LaTex
%
% File:  2-Theory.tex -- Basic Theory
% Author(s): Jürgen Hackl <hackl@ibi.baug.ethz.ch>
%            Clemens Kielhauser <kielhauser@ibi.baug.ethz.ch>
%
% Creation:  27 Jan 2014
% Time-stamp: <Tue 2013-08-13 20:14 juergen>
%
% Copyright (c) 2014 Infrastructure Management Group (IMG)
%               http://ibi.ethz.ch
%
% More information on LaTeX: http://www.latex-project.org/
%=============================================================================

\chapter{Szenarien}
\label{chap:Szenarien}

Die Szenarien dienen dazu die unsicherheiten der Kostenstrukturen zu modellieren. Der durchschnittliche tägliche MIV $DTV_{MIV}$ das durchschnittliche tägliche  Aufkommen an Fahrradfahrern $DTV_{Velo}$ sind abhängig von der Bevölkerungsmenge Usters. Das Bevölkerungswachstum in Uster wird gemäss STEK in drei verschiedenen Szenarien prognostiziert:

\begin{description}
\item[Starkes Wachstum] \hfill \\
Die Bevölkerung wächst jährlich um ... Einwohner
\item[Mittleres Wachstum] \hfill \\
Die Bevölkerung wächst jählich um ... Einwohner. \\
Dieses Szenario hat mit ... die höchste Eintrettenswarscheinlicheit.
\item[Geringes Wachstum] \hfill \\
Bevölkerung wächst jährlich um ... Einwohner
\end{description}



% ===========================================================================
% EOF
%

%%% Local Variables:
%%% mode: latex
%%% TeX-master: "../main"
%%% End:
