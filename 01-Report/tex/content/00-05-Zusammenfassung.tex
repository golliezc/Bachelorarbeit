%=============================================================================
% Thesis Template in LaTex
%
% File:  00-05-Zusammenfassung.tex -- Zusammenfassung der Thesis
% Author(s): Juergen Hackl <hackl@ibi.baug.ethz.ch>
%            Clemens Kielhauser <kielhauser@ibi.baug.ethz.ch>
%
% Creation:  27 Jan 2014
% Time-stamp: <Tue 2013-08-13 20:14 juergen>
%
% Copyright (c) 2014 Infrastructure Management Group (IMG)
%               http://www.ibi.ethz.ch
%
% More information on LaTeX: http://www.latex-project.org/
%=============================================================================

\chapter*{Zusammenfassung}
\label{chap:Zusammen}

Das Ziel dieser Arbeit war die Optimierung des Verkehrssystems von Uster durch Verbesserung eines Teilstücks der Veloinfrastruktur. Dazu wurde die Infrastruktur des Bahnübergangs
Brunnenstrasse in Abhängigkeit von unsicheren zukünftigen Nachfragebeziehung untersucht und verschiedene Vorschläge zur Verbesserung der Situation erarbeitet. Daraus wurde eine optimale Variante abgeleitet, welche die zukünftigen Bedürfnisse der Bevölkerung von Uster nach Mobilität am besten befriedigen kann.

Die Optimierung bestehender Infrastruktursystem stellt eine der grossen Herausforderungen des Infrastruktur Management dar. So ist in den meisten Fällen einerseits nicht eindeutig zu bestimmen, was genau die Probleme sind und durch was sie verursacht werden. Andererseits ist nicht immer vollständig festgelegt, was mit einer Veränderung erreicht werden soll. Hinzu kommt, dass nicht immer ersichtlich ist, welche Varianten einer Optimierung überhaupt möglich sind und welche unter Berücksichtigung unsicherer zukünftiger Entwicklungen der Kosten und Nutzen der Beteiligten optimal ist.

Die Bedürfnisse und Anforderungen der Nutzer und Besitzer der Infrastruktur können sich während der Lebensdauer drastisch verändern. Die Schwierigkeit ist daher, Infrastrukturen so zu gestalten, dass sie über ihre gesamte Lebensdauer ein angemessenes Leistungsniveau gewährleisten können. Es ist von allgemeinem Interesse, dass eine Intervention in eine bestehende Infrastruktur nur dann durchgeführt wird, wenn der Netto-Nutzen aller Beteiligten maximiert wird.

Im Rahmen dieser Projektarbeit habe ich für die bestehende Verkehrsproblematik von Uster verschiedene Varianten erarbeitet und eine optimale Lösung bestimmt. Dabei habe ich mich auf den Problemlösungsprozess aus dem Kurs Systems Engineering sowie auf die Theorie der Zielfunktion, des Entscheidungsbaumes und der Sensitivitätsanalysen gestützt. Um die Unsicherheiten zukünftiger Entwicklungen in den Entscheidungsprozess einfliessen zu lassen, wurden die entsprechenden Einflussfaktoren anhand von Prognosen geschätzt. Anschliessend wurde der Einfluss verschiedener Parametervariationen auf die Kostenberechnung modelliert und die ermittelten Werte mit der geschätzten Eintrittswahrscheinlichkeit der Szenarien gewichtet. Auf dieser Basis wurden die zur Auswahl stehenden Varianten beurteilt. Dabei hat sich die Variante 2 als die beste Option für die Zukunft von Uster herausgestellt.



%=============================================================================
% EOF
%

%%% Local Variables:
%%% mode: latex
%%% TeX-master: "../main"
%%% End:
