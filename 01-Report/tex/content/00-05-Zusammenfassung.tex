%=============================================================================
% Thesis Template in LaTex
%
% File:  00-05-Zusammenfassung.tex -- Zusammenfassung der Thesis
% Author(s): Juergen Hackl <hackl@ibi.baug.ethz.ch>
%            Clemens Kielhauser <kielhauser@ibi.baug.ethz.ch>
%
% Creation:  27 Jan 2014
% Time-stamp: <Tue 2013-08-13 20:14 juergen>
%
% Copyright (c) 2014 Infrastructure Management Group (IMG)
%               http://www.ibi.ethz.ch
%
% More information on LaTeX: http://www.latex-project.org/
%=============================================================================

\chapter*{Zusammenfassung}
\label{chap:Zusammen}

Die Optimierung bestehender Infrastruktursystem stellt eine der grossen Herausforderungen des Infrastruktur Management dar. So ist in den meisten Fällen einerseits nicht eindeutig zubestimmen was genau die Probleme sind und durch was sie verursacht werden und andereseits ist nicht immer vollständig festgelegt, was mit einer Veränderung erreicht werden soll. Hinzukommt, dass nicht immer ersichtlich ist, welche Variante einer Optimierung überhaupt möglich und schlussendlich effektiv die beste ist. Zusätzlich ist nicht immer eindeutig wie gezeigt werden kann, welche Option aus einer Gruppe von Varianten, unter Berücksichtigung der Unsicherheiten hinsichtlich der zukünftigen Entwicklung der Kosten und Nutzen der Beteiligten, die Optimale ist. \\
So können sich die Bedürfnisse und Anforderungen der Nutzer und Besitzer der Infrastruktur, während der Lebensdauer drastisch verändern. Die Schwierigkeit daher ist, eine  Infrastrukturen so zu gestalten, dass sie über ihre gesamte Lebensdauer ein angemessenes Leistungsniveau erbringen kann. Desweiteren ist es im Interesse der Allgemeinheit, dass eine Intervention in ein bestehendes Netz nur dann durchgeführt wird, wenn der netto Nutzen aller Beteiligten maximiert wird.

Im Rahmen dieser Projektarbeit habe ich durch die Bearbeitung der real bestehenden Verkehrsproblematik von Uster, mithilfe des Problemlösungsprozesses aus dem Kurs Systems Engineering sowie der Theorie der Zielfunktion, des Entscheidungsbaum und der Sensitivitätsanalyse, eine optimale Variante zur Verbesserung der Zukunft von Uster erarbeitet.
Um die Unsicherheiten die bei einer Vorhersage der Zukunft entstehen können, in den Entscheidungsprozess miteinfliessen zu lassen, wird die Entwicklung der wichtigsten zukünftigen Einflussfaktoren anhand von Prognosen geschätzt und anschliessend der Einfluss auf die Parameter der Kostenberchnung modelliert und schlussendlich der ermittelte Werte mit der geschätzten Eintrittswahrscheinlichkeit des Szenarios gewichtet.

Das Ziel dieser Arbeit ist, die Optimierung von Uster, durch das verbesseren eines Teilstücks der Veloinfrastruktur. Für diese Veloinfrastruktur habe ich eine optimale Variante erarbeitet, die, nach der Analyse der jetztigen Situation in Uster, die zukünftigen Bedürfnisse der Bevölkerung von Uster nach Mobilitat am besten befriedigen kann. So habe ich die Infrastruktur des Bahnübergangs Brunnenstrasse, in Abhängigkeit von unsicheren zukünftigen Nachfragebeziehung untersucht und Vorschläge zur Verbesserung der Situation erarbeitet und im nachfolgenden Schritt analysiert und bewertet, um die optimale Variante unter Berücksichtigung von veränderlichen zukünftigen Einflüssen, zu bestimmen. Dies geschieht aufgrund dessen, das durch das Berücksichtigen der veränderlichen Nachfrage nach Mobilität und durch die Berücksichtigung der Unsicherheiten, die beim modellieren dieser Einflüsse entstehen, im Enscheidungsprozess zur Generierung einer Infrastrukturintervention, zusätzliche Vorteile generiert und zukünftige Risiken beseitigt werden können.



%=============================================================================
% EOF
%

%%% Local Variables:
%%% mode: latex
%%% TeX-master: "../main"
%%% End:
