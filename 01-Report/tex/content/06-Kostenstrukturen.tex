%=============================================================================
% Thesis Template in LaTex
%
% File:  2-Theory.tex -- Basic Theory
% Author(s): Jürgen Hackl <hackl@ibi.baug.ethz.ch>
%            Clemens Kielhauser <kielhauser@ibi.baug.ethz.ch>
%
% Creation:  27 Jan 2014
% Time-stamp: <Tue 2013-08-13 20:14 juergen>
%
% Copyright (c) 2014 Infrastructure Management Group (IMG)
%               http://ibi.ethz.ch
%
% More information on LaTeX: http://www.latex-project.org/
%=============================================================================

\chapter{Kostenstrukturen}
\label{chap:Kosten}

Die Kosten der verschiedenen Interessensverbände setzen sich aus unterschiedlichsten Faktoren zusammen. In den folgenden Abschnitten werden die Kosten und ihre Berechnung dargestellt sowie die unsicheren Randbedingugen identifiziert. 

\begin{description}
\item[Unsicher Rahmenbedingungen]\hfill %\\
\begin{itemize}
\item Bevölkerungswachstum
\item Gebaute Variante
\item Bau von Uster West
\item Bau der Moosackerstrasse
\item Zentrumsentwicklung
\item Entwicklung der Nachfragebeziehung auf der Route Bahnhof - Sportanlage
\end{itemize}
\end{description}


\section{Besitzer}

Für die Besitzerin der Infrastruktur wird in dem betrachteten Zeitraum nur die Initialisierungskosten sowie die laufenden Betriebskosten von Bedeutung sein. Um ein vollständiges Bild der Interessen der Besitzer zu erhalten, müssten weiter Kosten in Betracht gezogen werden. 

Für unsere Untersuchungen, welche die Infrastruktur in den nächsten 40 Jahren betrachtet, spielen die Unterhaltskosten die bedeutendste Rolle. Einerseits übersteigt die W'keit, dass diese Kosten bezahlt werden müssen, die W'keit dass andere Kostentypen eintreten und andererseits ist der absolute Betrag der Unterhaltskosten über einen Zeitraum von $T$ Jahren deutlich grössers als der anderer Kostentypen. Aufgrund dieser Überlegungen brachten wir für die Besitzer nur die Unterhaltskosten. 

\subsection{Unterhaltskosten}

\begin{IMleftrightskip}
Die \textbf{Unterhaltskosten $K_{U}(t)$} werden in Formel \ref{equation:2} dargestellt. Sie setzen sich zusammen aus den einmaligen Investitionskosten für den Bau der Infrastruktur $K_{Inv}$. 
Die jährlich anfallenden Unterhaltskosten $K_{Unt,t}$ sowie die ausserordentlichen Interventionskosten pro Jahr $K_{Int,t}$. In den Jahren in denen an der Infrastruktur ausserplanmässige Unterhaltsarbeiten durchgeführt werden, wird $\delta_t = 1$ gesetzt ansonsten $0$. 
\end{IMleftrightskip}

\begin{equation}
\label{equation:2}
K_{U,i}(t) = K_{Bau,i} + \sum_{t=0}^T \ (K_{Unt,t} + \delta_{t} \cdot K_{Int,t})
\end{equation} 

{\setstretch{0.75}
wobei:
\begin{conditions}
 K_{U}(t)     	&  Totale Unterhaltskosten im Jahr $t$ \\
 K_{Inv}      	&  Investitionskosten für den Bau der Infrastruktur    \\
 K_{Unt,t}    	&  Unterhaltskosten im Jahr $t$  \\
 K_{Int,t}	  	&  Ausserordentliche Interventionskosten im Jahr $t$  \\
 $ $\delta_t$ $ &  $\in [0,1]$  
\end{conditions}
}

\begin{equation}
\label{equation:13}
K_{U,i}(t) = K_{\text{Bau,i}} + \sum_{t=0}^T \  EK_{U} \cdot s_{i} \cdot \lambda_{Unterhalt}
\end{equation}

\begin{align*}
      \lambda_{\mathrm{Unterhalt}} &=
      \begin{cases}
        \begin{aligned}
          & 0.8  \\
          & 1 \\
          & 2
        \end{aligned} &
        \begin{aligned}
         & \text{falls}\ \thinspace \\
         & \text{falls}\ \thinspace \\
         & \text{falls}\ \thinspace
        \end{aligned}
        \begin{aligned}
          & DTV < Kapazit"at_{i} \\
          & DTV = Kapazit"at_{i} \\
          & DTV > Kapazit"at_{i}
        \end{aligned}
      \end{cases}
\end{align*}

{\setstretch{0.75}
wobei:
\begin{conditions}
 K_{U}(t)     	     			&  Totale Unterhaltskosten im Jahr $t$ \\
 K_{Bau,i}           			&  Baukosten der Variante $i$    \\
 EK_{U}      	     			&  Kosten der Instandhaltung pro Kilometer im Jahr $t$  \\
 s_i	    	     			&  Länge der Variante $i$ \\  
 \lambda_{\mathrm{Unterhalt}} 	&  Auslastungsgrad
\end{conditions}
}



Die Einheitskosten pro Jahr, die in Tabelle \ref{tab:t-04-01-Interessensgruppen} angegeben werden, setzten sich nur aus den jährlich anfallenden Unterhaltskosten zusammen. Die Baukosten sowie die ausserordentlichen Interventionskosten müssen gesondert betrachtet werden. 

Die Einheitskosten für die Unterhaltsarbeiten pro Jahr setzen sich aus Referenzwerten der Stadt Zürich für ähnliche Infrastrukturprojekte zusammen. 
Die Länge der Infrastruktur sowie der gewählte Ausbaustandart sind in diesem Fall unsichere Variablen, die von der gebauten Variante abhängig sind. Desweiteren ist für die Abschätzung der jährlich anfallenden Unterhaltskosten die Anzahl an Nutzer zu bestimmen. 
Dies ist von Nöten, um Aussagen betreffend der Geschwindigkeit der Abnützung der Fahrbahn machen zu können. 
Die Bestimmung der Anzahl Nutzer ist Anhand von Schätzungen, basierend auf Messungen an vergelichbaren Infrastrukturobjekten, möglich jedoch nur mit einer gewissen Unsicherheit. Somit sind die Anzahl Nutzer sowie die gewählte Ausbauvariante die unsicheren Rahmenbedingugen die die Unterhaltskosten in Zufunft bestimmen. 
Die zukünftige Anzahl Nutzer dieser Infrastruktur ist desweiteren abhängig vom Bevölkerungswachstum und der Entwicklung von Uster bis 2035 gemäss dem STEK Schlussbericht.Die zu erwartende Bevölkerungsentwicklung kann der Abbildung 63 im STEK Schlussbericht Kapitel 10.3 \textit{Bevölkerungswachstum} entnommen werden.
\newpage

\section{Nutzer}
	
Für die nachfolgende Optimierung ist es von zentraler Bedeutung die massgebenden Kostenstrukturen zu definieren. In anbetracht der Verkehssituation, sind die  folgenden Kostentypen diejenigen die für den Nutzer massgebenden sind. 
Der Nutzer der Infrastruktur ist der Langsamverkehr der gemäss STEK sämtliche Verkehrsmitteln einschliesst die aus eigener Kraft angetrieben werden. Es muss geprüft werden ob auch Sonderbewilligungen zur Nutzung der Infrastruktur für E-Bikes und Hybridräder erteilt werden kann. Dies sollte nur geschehen fals die geplannte Kapazität der Infrastruktur das befahren mit deutlich unterschiedlichen Geschwindigekeiten erlaubt.

\subsection{Reisezeitkosten}

Die totalen \textbf{Reisezeitkosten $K_{TT}$} für $T$ Jahre werden gemäss Formel \ref{equation:3} berechnet. 
Die Reisezeitkosten geben die wirtschaftlichen Auswirkungen eines Zeitverlustes auf einen Verkehrsteilnehmer wieder. Es wird die zusätzlich benötigte Reisezeit verrechnet, in Anbetracht dessen, dass in dieser Zeit gearbeitet oder Freizeit hätte verbracht werden können. Sie sind somit die Kosten der Reise in Form des Zeitverlustes.    

\begin{equation}
K_{TT}(t) = \sum_{t=0}^T \ (S_{Nutzer,t} \cdot T_{Zeitverlust,t} \cdot K_{Zeitverlust})
\label{equation:3}
\end{equation}

{\setstretch{0.75}
wobei:
\begin{conditions}
 K_{TT}(t)		 &  Totale Kosten der verloreren Zeit im Jahr $t$  \\
 S_{Nutzer,t}    &  Totale Anzahl Nutzer im Jahr $t$ \\
 T_{Zeitverlust} &  Verlorene Zeit pro Nutzer auf der Infrastruktur, \newline in Abhängigkeit der Kapazität der Fahrbahn \\
 K_{Zeitverlust} &  Einheitskosten der verlorenen Zeit pro Stunde   
\end{conditions}
}

Der Zeitverlust ist abhängig vom Zustand der Infrastruktur, genauer von der Beschaffenheit des Oberflächenbelags der Fahrbahn. Diese Beziehung ist schwierig zu modelieren. Jedoch kann zwischen dem Strassenzustand, in Abhängigkeit von der gebauten Variante und der gefahrenen Geschwindigkeit eine Beziehung modeliert werden die es uns erlaubt die verlorene Zeit zu berechnen.  
Die Zeit die man benötigt eine bestimmte Strecke zurück zu legen ist abhängig von der gefahrenen Geschwindigkeit welche wiederum abhängig ist vom Zustand der Strasse sowie der Kapazität der Infrastruktur. Wird die Kapazität durch eine zu hohe Nachfrage überschritten, kann dies zu Verstopfungen und daraus resultierenden Verspätungen führen.  

Die Kapazitätsüberschreitung ist direkt abhängig von der Nachfragebeziehung auf der untersuchten Route. Diese Nachfrage ist wiederum abhängig von der Entwicklung der restlichen Infrastruktur in Uster sowie dem Bevölkerungswachstum.
 
Somit sind in diesem Falle die unsicheren Rahmenbedingungen die es für die Zukunft zu modellieren gilt, die Anzahl Verkehrsteilnehmer die auf der geplanten Veloroute zukünftig pro Tag, fahren werden, sowie der Ausbaustandart der Infrastruktur in Form des Querschnittsprofil der Strasse. 
Die Einheitskosten für den Zeitverlust repräsentiert den durchschnittlichen schweizer Lohn für eine Stunde Arbeit. Dies wurde hier mit 85 $\frac{CHF}{Stunde}$ angesetzt. 

Die Anzahl Nutzer kann wie in Formel \ref{equation:4} dargestellt, berechnet werden.

\begin{equation}
S_{Nutzer,t} = S_{Nutzer,0} + S_{Bev,Nutzer,t} + S_{SBB,Nutzer,t} + S_{Var,Nutzer,t}
\label{equation:4}
\end{equation}

{\setstretch{0.75}
wobei:
\begin{conditions}
 S_{Nutzer,t}  	   &  Totale Anzahl Nutzer im Jahr $t$ \\
 S_{Nutzer,0}      &  Initiale Anzahl Nutzer im Jahr $0$ \\
 S_{Bev,Nutzer,t}  &  Veränderung der Anzahl Nutzer aufgrund des Bevölkerungswachstum im Jahr $t$ \\
 S_{SBB,Nutzer,t}  &  Veränderung der Anzahl Nutzer aufgrund der Kapazitätserhöhung der SBB-Linie im Jahr $t$ \\
 S_{Var,Nutzer,t}  &  Veränderung der Anzahl Nutzer aufgrund der gebauten Variaten im Jahr $t$ 
\end{conditions}
}

Die Anzahl an Nutzer pro Jahr sowie die verlorene Zeit pro Nutzer und Jahr sind die unsicheren Variablen die es für die Berechnung dieser Kosten zu modelieren gilt. Sie sind abhänging von der Entwicklung der restlichen Infrastruktur in und um Uster. 


\subsection{Betriebskosten}

Die \textbf{Betriebskosten} $K_{B}$ des Fahrzeugs für den betrachteten Zeitraum von $T$ Jahren werden gemäss Formel \ref{equation:5} berechnet.
Diese sogenannten Fahrzeugsbetriebkosten werden im Rahmen der Optimierung der Kosten der verschiedenen Infrastruktur Interventionen als die jährlich pro Nutzer anfallenden Wartungskosten ermittelt. Diese setzen sich zusammen aus den Kosten für die Arbeitssstunden und der Kosten für die Ersatz- und Verschleissteile. Dies sind somit die Kosten für die Instandsetzung und den Betrieb eines Fahrzeugs. Diese Kosten sind abhängig von der Qualität der Fahrbahn sowie der Kapazität der Infrastruktur. Weiter ist ein entscheidender Faktor in der Bestimmung der Fahrzeugbetriebskosten die Strassengeometry. Diese beinhaltet die Anzahl und Form der Kurven, die Steigungen sowie die Breite der Strasse und die daraus resultierende Möglichkeit des sicheren Überholens. \newline
Diese Kosten sind abhängig von der Qualität des Fahrbahnbelags sowie von der genaueren Bauausführung der Infrastrukturvarianten. Die Anzahl an Kreuzungsstellen und die davon abhänginge Anzahl an Brems- und Beschleunigungsmanöver haben einen direkten Einfluss auf den Verschleiss der Mechanik des Fahrzeugs. So werden im Falle des Fahrrads die Kette und die Bremsbeläge durch vermehrtes Bremsen und Anfahren verstärkt abgenutzt.

\begin{equation}
K_{B}(t) =  \sum_{t=0}^T \ [(K_{Arbeit,t} + K_{Material,t}) \cdot S_{Nutzer,t}]
\label{equation:5}
\end{equation}

{\setstretch{0.75}
wobei:
\begin{conditions}
 K_{B}(t)		   &  Totale Fahrzeugbetriebskosten im Jahr $t$ \\
 K_{Arbeit,t}      &  Kosten für die Arbeitsstunden für die Instandsetzung pro Jahr \\
 K_{Material,t}    &  Kosten für die Ersatz- und Verschleissteile pro Jahr  \\
 S_{Nutzer,t}      &  Totale Anzahl Nutzer im Jahr $t$. Berechnet gemäss Formel \ref{equation:4}   \\
\end{conditions}
}

Die Kosten der Abeitsstunden sowie die Kosten der Materialien werden zusammengefasst als die Einheitskosten für den Fahrzeugbetrieb. Dieser wird in diesem Falle mit 100 $ \frac{CHF}{Nutzer}$ angesetzt. Die Totalen Kosten für die Instandsetzung und den Betrieb der Fahrzeuge pro Jahr setzt sich somit aus den Einheitskosten pro Nutzer multipliziert mit der Anzahl Nutzer pro Jahr zusammen.

\newpage

\section{Öffentliche Hand}
\label{sec:Public}

Die nachfolgenden Kosten sind die massgebenden Einwirkungen auf die Allgemeinheit. Die Öffentliche Hand setzt sich wie in Tabelle \ref{tab:t-04-01-Interessensgruppen} aufgeführt aus direkt und indirekt betroffener Personen zusammen. 

	
\subsection{Kosten durch Belastung der Umwelt}
\label{sec:Umweltkosten}

Die \textbf{Kosten die durch die Belastung der Umwelt} $K_{E}(t)$ (\textit{Englisch}: Environment) entstehen,
setzen sich auf den Kosten durch die Schadstoffbelastung $K_{S}(t)$ und der Kosten durch die Lärmbelastung $K_{L}(t)$ zusammen. 
Die Einheitskosten der Umweltbelastung wird aus der Summe der Einheitskosten für die Lärm- und Schadstoffbelastung gebildet.
Die Einheitskosten der Schadstoffbelastung pro Fahrzeug beträgt 500 $\frac{CHF}{Fahrzeug}$. Die Einheitskosten der Lärmbelastung liegen bei 500 $\frac{CHF}{Fahrzeug}$. 	 

\begin{equation}
K_{E}(t) = \sum_{t=0}^T \ [(K_{S,t} + K_{L,t}) \cdot S_{MIV,t}]
\label{equation:6}
\end{equation}

{\setstretch{0.75}
wobei:
\begin{conditions}
 K_{E}(t)	   &  Totale Umwelkosten im Jahr $t$ \\
 K_{S,t}       &  Kosten durch die Schadstoffbelastung pro Jahr \\
 K_{L,t}       &  Kosten durch die Lärmbelastung pro Jahr  \\
 S_{MIV,t}     &  Totale Anzahl Nutzer im Jahr $t$. Berechnet gemäss Formel \ref{equation:7}   \\
\end{conditions} 
}

Die Anzahl Fahrzeuge des motoriesierten Individualverkehrs (MIV), die pro Jahr auf der Infrastruktur unterwegs sind, wird gemäss Formel \ref{equation:7} berechnet.

\begin{equation}
S_{MIV,t} = S_{MIV,0} + S_{Bev,MIV,t} + S_{SBB,MIV,t} + S_{Var,MIV,t}
\label{equation:7}
\end{equation}

{\setstretch{0.75}
wobei:
\begin{conditions}
 S_{MIV,t}      &  Totale Anzahl Fahrzeuge im Jahr $t$ \\
 S_{MIV,0}      &  Initiale Anzahl Fahrzeuge im Jahr $0$ \\
 S_{Bev,MIV,t}  &  Veränderung der Anzahl Fahrzeuge aufgrund des Bevölkerungswachstum im Jahr $t$ \\
 S_{SBB,MIV,t}  &  Veränderung der Anzahl Fahrzeuge aufgrund der Kapazitätserhöhung der SBB-Linie im Jahr $t$ \\
 S_{Var,MIV,t}  &  Veränderung der Anzahl Fahrzeuge aufgrund der gebauten Variaten im Jahr $t$ \\
\end{conditions} 
}

Die Anzahl an Nutzer ist eine unsichere Variabel der Zukunft, welche abhänging ist von der Entwicklung der restlichen Infrastruktur in und um Uster. Die Totale Anzahl an Fahrzeugen pro Jahr $t$ ist die unsichere Rahmenbedingung die es, in Abhängigkeit der verschiedenen Szenarien zu modelieren gilt. 

Die \textbf{Kosten durch die Schadstoffbelastung} $K_{S}$ sind die Kosten der Schäden die aufgrund der Schadstoffbelastung entstehen. Die Schäden können neben gesundheitlichen Problemen für die Anwohner und Nutzer der Strasse auch die Beeinträchtigung des Pflanzenwachstums entlang der Infrastruktur sein. 

Die Luftverschmutzun wird duchr die Emissionen der motoriesierten Fahrzeuge verursacht. Somit kann die geplante Veloroute eine Reduktion der Belastung der Öffentliche Hand, in Form der Reduktion der Schadstoffemission, erreichen. 	
Die Schwierigkeit diese Kosten zu modelieren liegt in der nicht direkt messbaren Beziehung zwischen Schadstoffbelastung und daraus resultierenden Kosten. Jedoch ist zu vermerken das die Emissionen direkt von der Anzahl an motorisierten Fahrzeuge abhängt die auf der Infrastruktur verkehren. Somit ist die Anzahl an Fahrzeugen die täglich auf der Strasse fahren die unsichere Variable die es für die zukünftige Entwicklung zu modelieren gilt. 
Diese ist Abhänging vom Bevölkerungswachstum, der Entwicklung der Nachfragebeziehung auf der Route Bahnhof - Sportanlage, der Entwicklung des Stadtzentrums und dem Bau der Uster Westumfahrung sowie dem Bau der Moosackerstrasse. Ein weiterer Faktor den es zu erwähnen gilt, ist die Entwicklung der durchschnittlichen Fahrzeugsemissionswerte. Es muss berücksichtigt werden, welcher Trend sich hinsichtlich der Emissionswerte abzeichnet um dies in die Berechnung mit einzubeziehen. 
Die Schadstoffbelastung ist eine Funktion der durchschnittlich gefahrenen Geschwindigkeit sowie der Häufigkeit des \textit{Stopp and Go - Verkehrs}. So nimmt die Belastung der Luft durch Schadstoff deutlich zu, wenn vermehrt im \textit{Stopp and Go - Verkehr} gefahren wird. Diese Beziehung ist schwierig zu modelieren jedoch ist ersichtlich, dass die durchschnittlich gefahrene Geschwindigkeit vom Querschnittsprofil und somit von der Kapazität der Strasse abhängt. 

Die \textbf{Kosten durch Lärmbelastung} $K_{L}$ sind die Kosten die für die Allgemeinheit durch übermässigen Lärm entstehen, welcher von der Strasse verursacht wird. 
Die Kosten sind in diesem Falle die Störung und Beeinträchtigung der Anwohner in Form von Kopfschmerzen, Bluthochdruck, Schlafstörrungen sowie psychischer Belastung. \\
Der Lärm entsteht mehrheitlich durch Motorengeräusche sowie der Abrollgeräusche der Reifen. 

\newpage

\subsection{Unfallkosten}
\label{sec:Unfallkosten}

Die totalen \textbf{Unfallkosten} $K_{A}$ für den betrachteten Zeitraum von $T$ Jahren werden gemäss Formel \ref{equation:8} berechnet. 
In Betracht gezogen werden drei Komponenten. Einerseits der Verlust an Einnahmen für 30 Tage pro Unfall. Dies wird mit $11'500 \thinspace \frac{CHF}{Unfall}$ verrechnet. 
Weiter die Kosten für die Gesundheit bzw. die Pflege der verunfallten Person für 30 Tage. Die geschätzten Kosten belaufen sich auf $100'000 \thinspace \frac{CHF}{Unfall}$. 
Und zuletzt die Kosten für einen Unfall mit Todesfolge. Hierfür werden $3.7mio \thinspace \frac{CHF}{Unfall}$ angesetzt. 

\begin{equation}
K_{A}(t) = \sum_{t=0}^T \ [(K_{A,Einnahmen}  + K_{A,Pflege}) \cdot i_t + K_{A,Tod} \cdot j_t]
\label{equation:8}
\end{equation}

{\setstretch{0.75}
wobei:
\begin{conditions}
 K_{A}(t) 		 &  Totale Unfallkosten im Jahr $t$ \\
 K_{A,Einnahmen} &  Kosten aufgrund Einnahmeverlust für 30 Tage pro Unfall und Jahr  \\
 K_{A,Pflege} 	 &  Kosten für die Gesundheit bzw. Pflege für 30 Tage pro Unfall und Jahr \\
 K_{A,Tod}		 &  Kosten für Unfall mit Todesfolge pro Unfall und Jahr \\
 i_t        	 &  Anzahl Unfälle mit Ausfall der Einnahmen sowie Pflege für 30 Tage pro Jahr \\
 j_t        	 &  Anzahl Unfälle mit Todesfolge pro Jahr \\   
\end{conditions}
}

Hier ist zu Vermerken, dass die Anzahl Unfälle pro Jahr mit einer Unsichererheit behaftet ist, welche aus der Ungewissheit resultiert, um wie viel die Nachfrage auf der Route steigen wird. \\ [2ex]
So ist in diesem Fall die Anzahl Unfälle pro Jahr abhängig von der Anzahl Nutzer pro Jahr. Die Anzahl Nutzer wiederum ist abhängig von der gebauten Variante sowie vom progonstizierten Bevölkerungswachstum. \\
Für die Unfallkosten sind somit die Anzahl Unfälle und demnach die Anzahl Nutzer die unsicheren Randbedingungen die im Falle einer Optimierung modeliert werden müssten.



% ===========================================================================
% EOF
%

%%% Local Variables:
%%% mode: latex
%%% TeX-master: "../main"
%%% End:
