%=============================================================================
% Thesis Template in LaTex
%
% File:  2-Theory.tex -- Basic Theory
% Author(s): Jürgen Hackl <hackl@ibi.baug.ethz.ch>
%            Clemens Kielhauser <kielhauser@ibi.baug.ethz.ch>
%
% Creation:  27 Jan 2014
% Time-stamp: <Tue 2013-08-13 20:14 juergen>
%
% Copyright (c) 2014 Infrastructure Management Group (IMG)
%               http://ibi.ethz.ch
%
% More information on LaTeX: http://www.latex-project.org/
%=============================================================================

\chapter{Varianten}
\label{chap:Varianten}

Die Varianten beziehen sich auf das Kapitel 7.2.1 \textit{Die Velostadt verbessern}. Es wurde im STEK bereits definiert welche Routen Hauptverbindungen darstellen und daher besondere Aufmerksamkeit, im Rahmen der Entwicklung Usters zu einer Velostadt verdienen.
Die Abbildung 45 \textit{Räumliche Umsetzung der Strategie "die Velostadt verbessern"} im STEK Kap. 7 zeigt die Haupt- und Nebenrouten für den Veloverkehr im Raum Uster.
Auf dieser Abbildung wird die zentrale Bedeutung der Route Bahnhof - Sportanlage als wichtige Nord-Süd Achse verdeutlicht.
Die hier ausgeabeitet Varianten bestimmen die Rahmenbedingungen die für die Berechnung der Kosten der Optimierung von nöten sind. 
Die verschiedenen Möglichkeiten zur Ausgestaltung der Fahrbahn kann der Tabelle 1 im STEK Schlussbericht Kapitel 7.2 \textit{Uster fördert den Fuss- und Veloverkehr} entnommen werden.  
Die Tabelle 1 im STEK Schlussbericht Kapitel 7 basiert auf den kantonalen Vorgaben. 

Die Kapazität für den ÖV ist in keiner Variante beeinträchtigt. Einen Ausbau der Kapazität ist nicht geplannt. Neubauten oder umstrukturierungen verschiedener Bushaltestellen ist im Rahmen der genaueren Bauausführung zu prüfen. \\
Der Bau der Westerschliessung ist im Rahmen der Entwicklung Ustes zur Velostadt unerlässlich und ist zentral für die geplante umstrukturierung entlang der Route Bahnhof - Sportanlage.

\subsection{Variante: \ 1}
\label{chap:V1}
	
Diese Variante beinhalte den Bau der geplanten Velo- und Fussgängerunterführung im Bereich der Gleisquerung bei der Bahnhof-/ Brunnenstrasse. Weitere Investitionen sind in diesem Rahmen nicht geplant. Die Auführung der Velounterführung ist in verschiedenen Kapazitätsgrössen zu untersuchen. Es soll in Betracht gezogen werden, dass die Velounterführung als Flaschenhals in der Nord-Süd Achse fungiert und es ist somit zu Untersuchen welche Querschnittskapazitäten für welche Nachfragebeziehungen von nöten sind um verstopfungen aufgrund von Überbelastungen zu vermeiden. Die Bestimmung der optimalen Kapazität ist abhängig von der zukünftigen Verkehrsmenge und dadurch mit einer Unsicherheit behaftet die es zu modelieren gilt. 

\subsection{Variante: \ 2}
\label{chap:V2}
	
Die zweite von mir untersuchte Variante beinhaltet neben der Velounterführung einen Ausbau der Infrastruktur auf der Brunnenstrasse und der Pfäffikerstrasse bis zur Sportanlage Buchenholz. Dies ist im Rahmen der gängigen Ausbaustandarts für Velowege als Hauptrouten gemäss STEK Kap.7 Tab.1 auszuführen.

\subsection{Variante: \ 3}
\label{chap:V3}

Die dritte Variante ist eine geplante Veloschnellroute. 
Diese zeichnet sich durch separate und baulich von der restlichen Fahrbahn getrennte Radwege aus. Dies ermöglicht gemäss STEK eine komfortable, durchgehende und direkte Verindung mit allenfals zu prüfender Vortrittsberechtigung. Die Infrastruktur dieser Radwege sollte ausschliesslich für diese Art von Verkehr zugelassen sein und gemäss STEK Kap.7 Tab. 1 mind. 4.8m breit sein. Um eine kreuzungsfreie Durchfahrt zu gewährleisten ist eine Ausführung in Anlehnung an den Aufbau einer Autobahn zu prüfen.  
Das Ziel dieser Infrastruktur ist es grössere Distanzen in kurzer Zeit und mit hoher Geschwindigkeit unmotorisiert zurück legen zu könne. Dies setzt eine kreuzungsfreie Ausführung vorraus was mit baulichen Massnahmen oder Vortrittsberechtigungen erreicht werden kann.
Die dritte Variante ist an ein Beispiel aus dem Ausland angelehnt. Die Infrastrukturintervention wäre ein Ausbau der Nord-Süd Achse entlang der Pfäffikerstrasse, Brunnen- und Bahnhofstrasse zu einer Veloschnellstrasse. 
Das Vorbild dieser Infrastruktur ist die C99, Kopenhagen-Albertslund eine sogenannten \textit{Super-Radschnellroute} auch bekannt als \textit{Cykelsuperstier}. Diese Art von Veloinfrastruktur hat sich bereits in Dänemark bewehrt um das Pendeln leichter und sicherer zu machen. 
In Anbetracht der enormen Kapazitätssteigerung wäre ein Ausbau der Parkplätze auf der Sportanlage Buchenholz zu prüfen. Dies wäre im Rahmen eines Programmes gedacht das sich \textit{Ride and Bike} nennen könnte, welches zum Ziel hat die Kombination von Autobahnanschluss und Parkiermöglichkeit mit einer bequemen, sicheren und schnellen Veloverbindung ins Stadtzentrum zu fördern. 

Der geplanten Stadterschliessungen West und Süd-Ost sind für die Förderung des Langsamverkehrs in der Stadt Uster von zentraler Bedeutung. So kann eine Entlastung des Zentrums und der Nord-Süd Achse entlang der Bahnhofstrasse nur realisiert werden wenn der MIV entlang der Westumfahrung um das Stadtzentrum herum geführt wird. 





% ===========================================================================
% EOF
%

%%% Local Variables:
%%% mode: latex
%%% TeX-master: "../main"
%%% End:
