%=============================================================================
% Thesis Template in LaTex
%
% File:  01-Einleitung.tex -- Einleitung
% Author(s): Cyrano Golliez <golliezc@student.ethz.ch>
%            
%
% Creation:  27 Jan 2014
% Time-stamp: <Tue 2013-08-13 20:14 juergen>
%
% Copyright (c) 2014 Infrastructure Management Group (IMG)
%               http://ibi.ethz.ch
%
% More information on LaTeX: http://www.latex-project.org/
%=============================================================================

\chapter{Einleitung}
\label{chap:Einleitung}

Die Optimierung bestehender Verkehrssystem in städtischen Gebieten stellt insofern eine grosse Herausforderung dar, da die zukünftige Veränderung der Nachfrage nach Mobilität einerseits in der Erarbeitung und andererseits in der Analyse und der Bewertung von möglichen Lösungsvarianten berücksichtigt werden muss. 

Infrastrukturen müssen so gebaut werden, dass sie die Interessen aller beteiligten über einen langen Zeithorizont befriedigen können.

(NOCH NICHT LESEN -> IST NOCH MÜLL bzw. nur SATZBAUSTEIN)


Das Hauptziel dieser Arbeit ist die Optimierung des Velonetz von Uster, durch das Verbessern eines Teilstücks der Veloinfrastruktur. Für diese Veloinfrastruktur soll eine optimale Variante erarbeitet werden, die, nach der Analyse der momenaten Situation in Uster die zukünftigen Bedürfnisse der Bevölkerung von Uster nach Mobilitat am besten befriedigen kann. 

Ich habe mich im Rahmen dieser Fallstudie mit der Frage einer beschäftigt, die Langsamverkehrsinfrastruktur im Zentrum von Uster, trotz unsicherer Zukunft, zu optimieren. Im diese Zielsetzung erreichen zu können, habe ich die Infrastruktur des Bahnübergangs Brunnenstrasse in Abhängigkeit von unsicheren zukünftigen Nachfragebeziehung untersucht und Vorschläge zur Verbesserung der Situation erarbeitet.

Diese wichtige Route verbindet die südlichen Stadteile sowie das Zentrum ideal mit der Sportanalge Buchenhold sowie mit den Institutionen der Gesundheitsmeile auf dem Weg dahin. In der Gegenrichtung verbindet sie die nördlich der Bahngleise gelegenen Quartiere mit dem Zentrum. Dies ist somit eine der zentralen Achsen Usters und ihr ist im Rahmen der Stadtentwicklung besonderes Augenmerk zu schenken, insbesondere im Zuge der Zentrumsentwicklung und der Mobilitässtrateige 2035. 



 



% ===========================================================================
% EOF
%

%%% Local Variables:
%%% mode: latex
%%% TeX-master: "../main"
%%% End:
