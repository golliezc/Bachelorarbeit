%=============================================================================
% Thesis Template in LaTex
%
% File:  01-Einleitung.tex -- Einleitung
% Author(s): Cyrano Golliez <golliezc@student.ethz.ch>
%            
%
% Creation:  27 Jan 2014
% Time-stamp: <Tue 2013-08-13 20:14 juergen>
%
% Copyright (c) 2014 Infrastructure Management Group (IMG)
%               http://ibi.ethz.ch
%
% More information on LaTeX: http://www.latex-project.org/
%=============================================================================

\chapter{Einleitung}
\label{chap:Einleitung}

(NOCH NICHT LESEN -> IST NOCH MÜLL bzw. nur SATZBAUSTEIN)

Problemstellung gemäss IBI
Was ist Aufgabenstellung und Problemstellung

Das Hauptziel dieser Arbeit ist die Optimierung des Velonetz von Uster, durch das Verbessern eines Teilstücks der Veloinfrastruktur. Für diese Veloinfrastruktur soll eine optimale Variante erarbeitet werden, die, nach der Analyse der momenaten Situation in Uster die zukünftigen Bedürfnisse der Bevölkerung von Uster nach Mobilitat am besten befriedigen kann. Infrastrukturen müssen so gebaut werden, dass sie die Interessen aller beteiligten über einen langen Zeithorizont befriedigen können.

Ich habe mich im Rahmen dieser Fallstudie mit der Frage einer beschäftigt, die Langsamverkehrsinfrastruktur im Zentrum von Uster, trotz unsicherer Zukunft, zu optimieren. Im diese Zielsetzung erreichen zu können, habe ich die Infrastruktur des Bahnübergangs Brunnenstrasse in Abhängigkeit von unsicheren zukünftigen Nachfragebeziehung untersucht und Vorschläge zur Verbesserung der Situation erarbeitet.

Diese wichtige Route verbindet die südlichen Stadteile sowie das Zentrum ideal mit der Sportanalge Buchenhold sowie mit den Institutionen der Gesundheitsmeile auf dem Weg dahin. In der Gegenrichtung verbindet sie die nördlich der Bahngleise gelegenen Quartiere mit dem Zentrum. Dies ist somit eine der zentralen Achsen Usters und ihr ist im Rahmen der Stadtentwicklung besonderes Augenmerk zu schenken, insbesondere im Zuge der Zentrumsentwicklung und der Mobilitässtrateige 2035. 

In anbetracht der Bestrebungen aus Uster ein urbanes Regionalzentrum zu machen, ist die Förderung des ökologischen und zukunftsorientierten Langsamverkehr essenziell, insbesondere die Veloförderung. Hinsichtlich der zukünftigen Entwicklung der Mobilität wird das Fahrrad auch auf Strecken bis zu 30km eine entscheidende Rolle bei der Verkehrsmittelwahl spielen. 
Im Rahmen der Entwicklung des Bahnhofzentrum mit besonderem Augenmerk auf den Fahrbeziehungen der Buslinien, der geplanten Kapazität und Lage der Veloparkieranlagen und der Kapazitäten der Wintethurerstrasse sowie der geplanten Umfahrung Uster West wird die Infrastruktur auf der Nord-Süd Achse zum zentralen Element zur Förderung des Langsamverkehrs. Um diese zentrale Achse und auch das Stadtzentrum vom MIV Durchgangsverkehr zu entlasten ist der Ausbau der Uster Westumfahrung unumgänglich. 



% ===========================================================================
% EOF
%

%%% Local Variables:
%%% mode: latex
%%% TeX-master: "../main"
%%% End:
