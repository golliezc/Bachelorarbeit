%=============================================================================
% Thesis Template in LaTex
%
% File:  01-Einleitung.tex -- Einleitung
% Author(s): Cyrano Golliez <golliezc@student.ethz.ch>
%            
%
% Creation:  27 Jan 2014
% Time-stamp: <Tue 2013-08-13 20:14 juergen>
%
% Copyright (c) 2014 Infrastructure Management Group (IMG)
%               http://ibi.ethz.ch
%
% More information on LaTeX: http://www.latex-project.org/
%=============================================================================

\chapter{Einleitung}
\label{chap:Einleitung}

Im Rahmen dieser Bachelorarbeit im Bereich Infrastruktur Management habe ich geeignete Verfahren und Methoden zur Erarbeitung kreativer Lösungsansätze auf die Verkehrsproblematik von Uster angewendet.

Die Optimierung bestehender Verkehrssystem in städtischen Gebieten stellt eine grosse Herausforderung dar, da die zukünftige Veränderung der Nachfrage nach Mobilität einerseits in der Erarbeitung und andererseits in der Analyse und der Bewertung von möglichen Lösungsvarianten berücksichtigt werden muss. So sind bei der Planung von Strasseninfrastrukturen die aktuelle Ausgangslage sowie die wahrscheinlichsten zukünftigen Szenarien zu berücksichtigen, damit die Infrastruktur die Interessen aller Beteiligten über einen angemessenen Zeitraum befriedigen kann. \\
Da Infrastrukturen über einen sehr langen Zeitraum bestehen, muss eine Interventionen in ein bestehendes System zukunftsorientiert sein und die Gesamtkosten aller beteiligten Personen über diesen Zeitraum minimieren. Infolge dessen muss man, um eine Intervention zu erarbeiten die zukünftigen Einflüsse auf die Infrastruktur anhand von Prognosen modellieren. Die Anforderungen der Nutzer und Besitzer der Infrastruktur können sich während der Lebensdauer einer Infrastruktur aufgrund verschiedener Einflussfaktoren, wie zum Beispiel der Implementierung neuer Technologien oder dem Bevölkerungswachstum, drastisch verändern. So ist zum Beispiel ungewiss ob die Nachfrage nach Mobilität über den betrachteten Zeitraum steigt oder sinkt. \\
Das Modellieren dieser Einflüsse auf die Situation muss einerseits mit grösster Sorgfalt geschehen. Andererseits ist dies nur mit einer gewissen Unsicherheit möglich, was bei der Bestimmung der optimalen Intervention berücksichtigt werden muss. Eine optimale Variante in Anbetracht der unsicheren zukünftigen Gegebenheiten zu erarbeiten, ist das Ziel dieser Projektarbeit. 

Das sternförmige Strassennetz von Uster hat zur Folge, dass die Verkehrsnetze insbesondere in Hauptverkehrszeiten überlastet sind. Hinzu kommt die mittig durch die Stadt führende Bahnlinie, welche die Stadt zerschneidet und aufgrund der langen Schliesszeit der Bahnschranken von bis zu 40’/h lange Wartezeiten an den Bahnübergängen verursacht. \\
Um Uster nachhaltig zu verbessern, habe ich die in dieser Arbeit vorgestellten Lösungsansätze generiert, indem ich dem Ablauf des Problemlösungsprozesses folgte. Dieser systematische Prozess erlaubt es, jede Art von Problem zu lösen. Durch die Optimierung der Zielfunktion wird die beste Variante ermittelt. Mithilfe des Entscheidungsbaumes wird der Bewertungs- und Entscheidungsprozess graphisch dargestellt. Mithilfe der Sensitivitätsanalyse der rechten Seite der Zielfunktion wird die optimale Lösung, auf ihre Belastbarkeit überprüft. \\

Die von mir erarbeitet Variante 2, stellt nach dem durchgeführten Risikovergleich die optimale Variante über den betrachteten Zeitraum von vierzig Jahren dar. Die durchgeführten Sensitivitätsanalysen bestätigen die getroffene Wahl. Somit wird durch den Bau der Variante 2 die Zukunft von Uster optimal und nachhaltig verbessert sowie der Nutzen aller Beteiligten erhöht. 



 



 



% ===========================================================================
% EOF
%

%%% Local Variables:
%%% mode: latex
%%% TeX-master: "../main"
%%% End:
