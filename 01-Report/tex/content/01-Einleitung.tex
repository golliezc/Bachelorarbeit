%=============================================================================
% Thesis Template in LaTex
%
% File:  01-Einleitung.tex -- Einleitung
% Author(s): Cyrano Golliez <golliezc@student.ethz.ch>
%            
%
% Creation:  27 Jan 2014
% Time-stamp: <Tue 2013-08-13 20:14 juergen>
%
% Copyright (c) 2014 Infrastructure Management Group (IMG)
%               http://ibi.ethz.ch
%
% More information on LaTeX: http://www.latex-project.org/
%=============================================================================

\chapter{Einleitung}
\label{chap:Einleitung}

Die von mir untersuchte Infrastruktur Interventionen beziehen sich auf die Nord-Süd Verbindung
entlang der Pfäffikerstrasse, Brunnenstrasse und Bahnhofstrasse mit speziellem Augenmerk auf der geplanten Velo- und Fussgängerunterführung im Bereich der Gleisquerung Bahnhofstrasse/Brunnenstrasse. 

Diese wichtige Route verbindet die südlichen Stadteile sowie das Zentrum ideal mit der Sportanalge Buchenhold sowie mit den Institutionen der Gesundheitsmeile auf dem Weg dahin. In der Gegenrichtung verbindet sie die nördlich der Bahngleise gelegenen Quartiere mit dem Zentrum. Dies ist somit eine der zentralen Achsen Usters und ihr ist im Rahmen der Stadtentwicklung besonderes Augenmerk zu schenken, insbesondere im Zuge der Zentrumsentwicklung und der Mobilitässtrateige 2035. 

In enbracht der Bestrebungen aus Uster ein urbanes Regionalzentrum zu machen, ist die Förderung des ökologischen und zukunftsorientierten Langsamverkehr essenziell, insbesondere die Veloförderung. Hinsichtlich der zukünftigen Entwicklung der Mobilität wird das Fahrrad auch auf Strecken bis zu 30km eine entscheidende Rolle bei der Verkehrsmittelwahl spielen. 
Im Rahmen der Entwicklung des Bahnhofzentrum mit besonderem Augenmerk auf den Fahrbeziehungen der Buslinien, der geplanten Kapazität und Lage der Veloparkieranlagen und der Kapazitäten der Wintethurerstrasse sowie der geplanten Umfahrung Uster West wird die Infrastruktur auf der Nord-Süd Achse zum zentralen Element zur Förderung des Langsamverkehrs. Um diese zentrale Achse und auch das Stadtzentrum vom MIV Durchgangsverkehr zu entlasten ist der Ausbau der Uster Westumfahrung unumgänglich. 

Das Ziel der von mir untersuchten Infrastruktur Investitionen ist es den Gesamtnutzen der Interessensverbände zu maximieren. \newline Dies mit speziellem Augenmerk auf der Vermehrung des Nutzens der Nutzer dieser Infrastruktur, d.h. der Langsamverkehr. 

Die Nutzen die durch die Anwendung einer Infrastruktur Intervention entstehen sind hier definiert als die Verminderung der Kosten. 

Die von mir verwendete Grundlage für die Erstellung dieser Auflistung der Steakholder sowie der Nutzen der jeweiligen Interessensverbände, basiert auf dem Dokument:


\noindent\hspace*{10mm}\textit{Structure and Infrastructure Engineering: IM1-HS2019-HO-Adey-et-al-2012-f.pdf} 


Ich möchte den Nutzen also die Reduktion der Kosten gemäss der Funktionen in diesem Dokument berechnen. 
Die Funktionen werden ich anhand der Angaben aus dem \textit{STEK} im Rahmen der von mir definierten unsicheren Variablen modellieren. \\ [2ex]
Die von mir untersuchten Infrastruktur Investitionen werden sich hinsichtlich ihrer Kapazitäten unterscheiden. Einerseits möchte ich untersuchen welche Variante die optimale Vermehrung des Gesamtnutzens ermöglicht und andererseits welche die optimale Kapazität für eine geplannte Lebensdauer von mind. 40 Jahren ist. 


% ===========================================================================
% EOF
%

%%% Local Variables:
%%% mode: latex
%%% TeX-master: "../main"
%%% End:
