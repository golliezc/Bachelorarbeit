%=============================================================================
% Thesis Template in LaTex
%
% File:  00-03-Abstract.tex -- Abstract of the Thesis
% Author(s): Juergen Hackl <hackl@ibi.baug.ethz.ch>
%            Clemens Kielhauser <kielhauser@ibi.baug.ethz.ch>
%
% Creation:  27 Jan 2014
% Time-stamp: <Tue 2013-08-13 20:14 juergen>
%
% Copyright (c) 2014 Infrastructure Management Group (IMG)
%               http://www.ibi.ethz.ch
%
% More information on LaTeX: http://www.latex-project.org/
%=============================================================================

\chapter*{Abstract}
\label{chap:abstract}


The aim of this project was to optimize the transport system of Uster by improving a section of the bicycle infrastructure. Therefore the infrastructure of the level crossing Brunnenstrasse as a function of uncertain future demand relations and developed various proposals for improving the situation has been investigated. From this, an optimal variant was derived, which can best satisfy the future mobility needs of the population of Uster.

The optimization of existing infrastructure systems is one of the major challenges of the future of infrastructure management. In most cases, it is not possible to clearly determine what exactly the problems are and what causes them. On the other hand, it is not always completely clear what is to be achieved by a change. In addition, it is not always clear which variant of an optimization is possible at all and which is optimal, taking into account uncertain future developments of the costs and benefits of those involved.

The needs and requirements of the users and owners of the infrastructure can change drastically during its lifetime. The difficulty is therefore to design infrastructure in such a way that it can guarantee an adequate level of performance throughout its life cycle. It is in the general interest that any intervention in existing infrastructure should only be undertaken if the net benefit to all parties involved is maximised.

Within the scope of this project, I have worked out different variants for the existing traffic problems of Uster and determined an optimal solution. In doing so, I drew on the problem-solving process from the Systems Engineering course and on the theory of the objective function, the decision tree and sensitivity analyses. In order to allow the uncertainties of future developments to flow into the decision-making process, the corresponding influencing factors were estimated using forecasts. Subsequently, the influence of various parameter variations on the cost calculation was modelled and the values determined were weighted with the estimated probability of the scenarios occurring. On this basis, the variants available for selection were assessed. Variant 2 proved to be the best option for the future of Uster.




%=============================================================================
% EOF
%

%%% Local Variables:
%%% mode: latex
%%% TeX-master: "../main"
%%% End:
