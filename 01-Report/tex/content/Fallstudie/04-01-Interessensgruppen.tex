%=============================================================================
% Thesis Template in LaTex
%
% File:  Interessensgruppen -- Fallstudie
% Author(s): Jürgen Hackl <hackl@ibi.baug.ethz.ch>
%            Clemens Kielhauser <kielhauser@ibi.baug.ethz.ch>
%
% Creation:  27 Jan 2014
% Time-stamp: <Tue 2013-08-13 20:14 juergen>
%
% Copyright (c) 2014 Infrastructure Management Group (IMG)
%               http://ibi.ethz.ch
%
% More information on LaTeX: http://www.latex-project.org/
%=============================================================================

% Unterkapitel Interessengruppen
% ---------
\label{subsec:Gruppen}

Der Nutzen einer Intervention in ein bestehendes Infrastruktursystem, ist definiert als die positive oder negative Auswirkung, die das Ausführen einer solchen Intervention auf die beteiligten Personen hat. Demzufolge entspicht der Nutzen einer Intervention, dem Effekt der Veränderung der aktuellen Situation. Um den Mehrwert einer Intervention ermitteln zu können, muss in einem ersten Schritt die beteiligten Personen bestimmt werden. Die sogenannte Interessensgruppen, werden gemäss Abschnitt \ref{sec:Inter.gruppen} bestimmt und im nachfolgenden Abschnitt kurz erläutert, sowie die wichtigsten Kosten der beteiligten Parteien vorgestellt.

\paragraph{Besitzer}

Die Interessensgruppe der Eigentümer der Infrastruktur setzt sich aus verschiedene Parteien zusammen. Die wichtigsten involvierten Parteien sind einerseits die Stadt Uster und der Kanton Zürich für die Strasseninfrastruktur und andererseits die SBB als Inhaberin der Bahninfrastruktur. Sie werden im Rahmen dieser Betrachtung durch die Baudirektion der Stadt Uster vertreten. 
Diese Parteien haben einen grossen Einfluss darauf, welche Variante gebaut wird. Sie sind somit die wichtigsten Akteure im politischen Diskurs über die Notwendigkeit einer Veränderung der Infrastruktur. \\ 
Der Besitzer der Infrastruktur ist Verantwotlich für den Bau der Intervention und für die Wartung der Infrastruktur. Er bezahlt einerseits die Veränderung der bestehenden Infrastruktur und ist andererseits dafür verantwortlich, dass die Seviceleistung der Infrastruktur über den betrachteten Zeithorizont auf einem angemessenen Niveus, gewährleistet ist. Das bedeuted, dass er für die Kosten der Wartung und Instandhaltung der Infrastruktur aufkommen muss. Die Kosten die dem Besitzer über den betrachteten Zeitraum entstehen, setzten sich aus den Arbeits- und Materialstunden für die jährliche Wartung und den Kosten den der Bau einer Variante verursacht, zusammen. 


\paragraph{Nutzer}

Die Nutzer der Infrastruktur sind die Velo- und Autofahrer welche den Bahnübergang passieren. Der Velofahrer repräsentiert im Rahmen dieser Untersuchung gemäss der STEK, sämtliche Verkehrsteilnehmer des Langsamverkehrs. Langsamverkehr bedeuted, dass der Antrieb, ausschliesslich durch die eigene Muskelkraft erfolgt. Davon ausgenommen sind E-Bikes mit einer zulässigen Höchstgeschwindigkeit von unter 35 $km/h$. \\
Das befahren einer Infrastruktur kann dem Nutzer Kosten verursachen. In anbetracht der momentanen Verkehssituation am Bahnübergang sin die massgebenden Kosten die dem Nutzer entstehen, zum einen die Kosten durch verlängerte Reisezeiten, die sogenannten Reisezeitkosten ($TT$) (engl. Travel time cost) und die Kosten die durch den Betrieb des Fahrzeugs beim befahren der Infrastruktur entstehen.  
Der Nutzer hat das grösste Interesse an einer Veränderung der aktuellen Verkehrssituation und ist am stärksten von einer Veränderung betroffen ist. Demnach ist es unerlässlich die Kosten die dem Nutzer entstehen, in der Berechnung der Kosten einer Intervention, zu berücksichtigen.


\paragraph{Öffentliche Hand}

Die Öffentliche Hand setzt sich aus den direkt und indirekt von der Infrastruktur betroffenen Personen zusammen. 
Die direkt betroffenen, sind zum einen die Anwohner am Bahnübergang und zum anderen die Inhaber von Geschäften und Restaurants, in unmittelbarer Umgebung der Infrastruktur. Sie nutzen die Infrastruktur zum betrachteten Zeitpunkt nicht, befinden sich aber in unmittelbarer Nähe und werden durch die Schadstoffemissionen und die Lärmemission, die von der Infrastruktur verursacht werden, beeinträchtig. 
Ihnen entstehen Kosten, die von der Allgemeinheit getragen werden müssen.\\
Die indirekt betroffene Öffentlichkeit repräsentiert im Rahmen dieser Untersuchung, die gesamte Bevölkerung der Stadt Uster. 
Sie tragen einerseits die Kosten durch die Belastung der Umwelt mit und zusätzlich die entstehenden Gesundheitskosten durch Unfälle auf der Infrastruktur. \\
Diese Unfallkosten gehen zulasten der Allgemeinheit, in Form der Belastung des Gesundheitssystems. Die Allgemeinheit nutzt in diesem Sinne die Infrastruktur nicht und ist auch nicht in ihrer Nähe zuhause oder bei der Arbeit, sondern wird durch die Nutzung der Infrastruktur indirekt beeinträchtigt. \\
Zur Vereinfachung werden die direkt und indirekt betroffenen Personen unter dem Stichpunkt \textit{Öffentliche Hand} zusammengefasst.
 


% ===========================================================================
% EOF
%

%%% Local Variables:
%%% mode: latex
%%% TeX-master: "../main"
%%% End:
