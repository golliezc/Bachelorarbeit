%=============================================================================
% Thesis Template in LaTex
%
% File:  Interessensgruppen -- Fallstudie
% Author(s): Jürgen Hackl <hackl@ibi.baug.ethz.ch>
%            Clemens Kielhauser <kielhauser@ibi.baug.ethz.ch>
%
% Creation:  27 Jan 2014
% Time-stamp: <Tue 2013-08-13 20:14 juergen>
%
% Copyright (c) 2014 Infrastructure Management Group (IMG)
%               http://ibi.ethz.ch
%
% More information on LaTeX: http://www.latex-project.org/
%=============================================================================

% Unterkapitel Interessengruppen
% ---------
\label{subsec:Gruppen}

Der Nutzen einer Intervention, ist als die positive oder negative Auswirkung einer Intervention auf die beteiligten Personen definiert. Um den Mehrwert einer Intervention ermitteln zu können, müssen in einem ersten Schritt die beteiligten Personen bestimmt werden. Die sogenannten Interessensgruppen werden gemäss Abschnitt \ref{sec:Inter.gruppen} bestimmt und im nachfolgenden Abschnitt kurz erläutert. Die wichtigsten Kosten der beteiligten Parteien werden vorgestellt.

\paragraph{Besitzer}

Die Interessensgruppe der Besitzer setzt sich aus verschiedene Parteien zusammen. Die wichtigsten involvierten Parteien sind die Stadt Uster und der Kanton Zürich als Eigentümer der Strasseninfrastruktur und die SBB als Besitzerin der Bahninfrastruktur. 
Ausserdem sind sie die wichtigsten Akteure im politischen Diskurs über die Notwendigkeit einer Veränderung der Infrastruktur und haben dementsprechend einen grossen Einfluss auf den Entscheid, welche Variante gebaut werden soll. \\ 
Der Besitzer der Infrastruktur bezahlt einerseits den Bau der Infrastrukturintervention und ist andererseits dafür verantwortlich, dass die Serviceleistung der Infrastruktur über den betrachteten Zeithorizont auf einem angemessenen Niveau gewährleistet ist. Das bedeutet, dass er für die Kosten der Wartung und Instandhaltung der Infrastruktur aufkommen muss. Die Kosten, die dem Besitzer über den betrachteten Zeitraum entstehen, setzten sich aus den Arbeits- und Materialstunden für die jährliche Wartung und den Kosten für den Bau einer Variante zusammen. 


\paragraph{Nutzer}

Die Nutzer der Infrastruktur sind zum einen die Velofahrer, wobei der Velofahrer im Rahmen dieser Untersuchung sämtliche Verkehrsteilnehmer des Langsamverkehrs repräsentiert und zum anderen der MIV. Langsamverkehr bedeutet in diesem Fall, dass der Antrieb ausschliesslich durch Muskelkraft erfolgt. Davon ausgenommen sind E-Bikes mit einer zulässigen Höchstgeschwindigkeit von maximal 35 $km/h$. \\
Das Befahren einer Infrastruktur kann dem Nutzer Kosten verursachen. In Anbetracht der momentanen Verkehrssituation am Bahnübergang sind die massgebenden Kosten, die dem Nutzer entstehen, zum einen die Kosten durch verlängerte Reisezeiten, die sogenannten Reisezeitkosten ($TT$) (engl. Travel Time Cost) und die Kosten die durch den Betrieb des Fahrzeugs beim Befahren der Infrastruktur entstehen.  \\
Der Nutzer hat das grösste Interesse an einer Veränderung der aktuellen Verkehrssituation und ist am stärksten von einer Veränderung betroffen. Demnach ist es unerlässlich, die Kosten, die dem Nutzer entstehen, in der Berechnung der Kosten einer Intervention zu berücksichtigen.


\paragraph{Öffentlichkeit}

Die Öffentlichkeit setzt sich aus den direkt oder indirekt von der Infrastruktur betroffenen Personen zusammen. 
Die direkt Betroffenen sind zum einen die Anwohner am Bahnübergang und zum anderen die Inhaber von Geschäften und Restaurants in unmittelbarer Umgebung der Infrastruktur. Sie werden durch die Schadstoffemissionen und die Lärmemission, die von der Infrastruktur verursacht werden, beeinträchtigt und tragen die Kosten, die durch diese Belastung entstehen. \\
Die indirekt betroffene Öffentlichkeit stellt im Rahmen dieser Untersuchung die gesamte Bevölkerung der Stadt Uster dar. Diese wird durch die Nutzung der Infrastruktur indirekt beeinträchtigt und trägt einerseits die Kosten durch die Belastung der Umwelt mit und zusätzlich die entstehenden Gesundheitskosten durch Unfälle auf der Infrastruktur. \\
Zur Vereinfachung werden die direkt und indirekt betroffenen Personen unter der Bezeichnung \textit{Öffentlichkeit} zusammengefasst.
 


% ===========================================================================
% EOF
%

%%% Local Variables:
%%% mode: latex
%%% TeX-master: "../main"
%%% End:
