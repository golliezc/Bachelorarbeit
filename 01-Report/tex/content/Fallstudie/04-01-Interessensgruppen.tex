%=============================================================================
% Thesis Template in LaTex
%
% File:  Interessensgruppen -- Fallstudie
% Author(s): Jürgen Hackl <hackl@ibi.baug.ethz.ch>
%            Clemens Kielhauser <kielhauser@ibi.baug.ethz.ch>
%
% Creation:  27 Jan 2014
% Time-stamp: <Tue 2013-08-13 20:14 juergen>
%
% Copyright (c) 2014 Infrastructure Management Group (IMG)
%               http://ibi.ethz.ch
%
% More information on LaTeX: http://www.latex-project.org/
%=============================================================================

% Unterkapitel Interessengruppen
% ---------
\label{subsec:Gruppen}

\begin{description}
\item[Besitzer]\hfill \\
Die Interessensgruppe der Eigentümer setzt sich aus verschiedene Parteien zusammen. Die wichtigsten involvierten Parteien sind die Stadt Uster und der Kanton Zürich sowie die Eigentümerin der Sportanlage Buchholz. Sie werden durch die Baudirektion der Stadt Uster vertreten.
Für die Besitzerin der Infrastruktur wird in dem betrachteten Zeitraum nur die Initialisierungskosten sowie die laufenden Betriebskosten von Bedeutung sein. Um ein vollständiges Bild der Interessen der Besitzer zu erhalten, müssten weiter Kosten in Betracht gezogen werden. 
Für unsere Untersuchungen, welche die Infrastruktur in den nächsten 40 Jahren betrachtet, spielen die Unterhaltskosten die bedeutendste Rolle. Einerseits übersteigt die W'keit, dass diese Kosten bezahlt werden müssen, die W'keit dass andere Kostentypen eintreten und andererseits ist der absolute Betrag der Unterhaltskosten über einen Zeitraum von $T$ Jahren deutlich grössers als der anderer Kostentypen. Aufgrund dieser Überlegungen brachten wir für die Besitzer nur die Unterhaltskosten. 
\item[Nutzer]\hfill \\ 
Für die nachfolgende Optimierung ist es von zentraler Bedeutung die massgebenden Kostenstrukturen zu definieren. In anbetracht der Verkehssituation, sind die  folgenden Kostentypen diejenigen die für den Nutzer massgebenden sind. 
Der Nutzer der Infrastruktur ist der Langsamverkehr der gemäss STEK sämtliche Verkehrsmitteln einschliesst die aus eigener Kraft angetrieben werden. Es muss geprüft werden ob auch Sonderbewilligungen zur Nutzung der Infrastruktur für E-Bikes und Hybridräder erteilt werden kann. Dies sollte nur geschehen fals die geplannte Kapazität der Infrastruktur das befahren mit deutlich unterschiedlichen Geschwindigekeiten erlaubt.
Die Infrastruktur ist für jegliche Art von nichtmotorisiertem Verkehr ausgelegt. Dieser beinhaltet Fahrräder, Inlineskater, Skateboarder und auch Rollstuhlfahrer.Wichtig für die planung der Intervention ist die Unterteilung der Verkehrsteilnehmer anhand ihrer durchschnittlichen Geschwindigkeit.
\item[Öffentliche Hand]\hfill \\
Die in Tabelle \ref{tab:t-04-01-Interessensgruppen} aufgeführten Kosten sind die massgebenden Einwirkungen auf die Allgemeinheit. Die Öffentliche Hand setzt sich aus direkt und indirekt betroffener Personen zusammen. 
Die Anwohner sowie in einem entfernteren Sinne auch die Nutzer selbst, zählen zur direkt betroffenen Öffentlichkeit. Sie nutzen die Infrastruktur nicht direkt, befinden sich aber in ihrer unmittelbaren Nähe. Diese sind die Hauptträger der Kosten die durch Lärm- und Schadstoffbelastung enstehen. Eine Reduktion des MIV Anteil und die damit einhergehende verkehrsberuhigung sind die besten Mittel zur Reduktion dieser Kosten.   \\
Die Unfallkosten gehen zulasten der Allgemeinheit in Form der Belastung des Gesundheitssystems. Die Allgemeinheit nutzt in diesem Sinne die Infrastruktur nicht und ist auch nicht in ihrer Nähe zuhause oder bei der Arbeit sondern wird durch die Benützung der Infrastruktur von ihr indirekt betroffen.
\end{description}




% ===========================================================================
% EOF
%

%%% Local Variables:
%%% mode: latex
%%% TeX-master: "../main"
%%% End:
