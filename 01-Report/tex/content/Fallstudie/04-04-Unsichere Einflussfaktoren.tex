%=============================================================================
% Thesis Template in LaTex
%
% File:  Unsichere Parameter -- Fallstudie
% Author(s): Jürgen Hackl <hackl@ibi.baug.ethz.ch>
%            Clemens Kielhauser <kielhauser@ibi.baug.ethz.ch>
%
% Creation:  27 Jan 2014
% Time-stamp: <Tue 2013-08-13 20:14 juergen>
%
% Copyright (c) 2014 Infrastructure Management Group (IMG)
%               http://ibi.ethz.ch
%
% More information on LaTeX: http://www.latex-project.org/
%=============================================================================

% Unterkapitel Unsichere Parameter
% ---------

%\subsubsection*{Tägliches Verkehrsaufkommen}
%\label{subsubsec:DTV}

Um einen nachhaltige Verbesserung der Verkehrsproblematik in Uster zu erreichen, muss die optimale Lösung, die Situation für die nächsten vierzig Jahre verbessern. Damit ein Zeitrum von vierzig Jahren untersucht werden kann, müssen die unsicheren zukünftigen Entwicklungen der wichtigsten Einflussfaktoren berücksichtigt werden. Die nachfolgenden Auflistung stellt die wichtigsten Einflüsse auf die Verkehrssituation am Bahnübergang und somit auf das DTV dar. 

Alle diese Einflussfaktoren haben gemeinsam, dass einerseits ihre zukünftige Entwicklung und andererseits das Ausmass, in dem sie den DTV in der Zukunft beeinflussen, ungewiss ist. Diese Einflüsse müssen, um die Unsicherheiten hinsichtlich der zukünfiten Mobilitätssituation am Bahnübergang berücksichtigen und um eine optimale Lösung für die nächsten vierzig Jahre finden zukönnen, im Rahmen dieser Optimierung modelliert werden. 

{\setstretch{0.6}
\begin{itemize}
\item Bevölkerunswachstum
\item Zentrumsentwicklung und Verkehrsberuhigung
\item Ausbau der Veloparkieranlagen am Bahnhof 
\item Aufwertung der Quartiere nördlich des Bahnhofs
\item Urbane Strassenraumgestaltung im Zentrumsgebiet
\item Förderung des Langsamverkehrs gemäss STEK 
\item Ausbau des Spital und der Sportanlage Buchholz
\end{itemize}
}

Da der Verkehr am Bahnübergang hauptsächlich aus Ziel- und Quellverkehr des Zentrums besteht, hat das Bevölkerungswachstum den grössten Einfluss auf das DTV am Bahnübergang. Gemäss der Angaben des Stadtentwicklungskonzept, kurz STEK, leben in Uster 35'000 Einwohner. Die zu erwartende Entwicklung der Bevölkerung, ist abhängig von verschiedenen Faktoren und demnach nur anhand von Prognossen vorhersagbar. Gemäss der Prognossen im STEK, wird der Wachstumstrend in der Zukunft anhalten und die kantonalen Vorgaben, sehen für Uster bis 2035 eine Bevölkerungszunahme um 7000 Einwohner vor.  (\cite{STEK})

Der Bau der Uster Westumfahrung, sowie der Bau der Moosackerstrasse haben gemäss STEK keinen nennenswerten Einfluss auf die Menge an Autos, die den Bahnübergang Brunenstrasse, in Zukunft passieren werden. Dies folgt, wie in Abschnitt \ref{chap:Fallstudie} erläutert, der Annahme dass die Umleitung des Durchgangsverkehr über die Oberlandstrasse bereits nahezug vollständig vollzogen ist und das der gemessene DTV hauptsächlich aus Quell- und Zielverkehr ins Zentrums besteht.  (\cite{STEK})

\newpage


% ===========================================================================
% EOF
%

%%% Local Variables:
%%% mode: latex
%%% TeX-master: "../main"
%%% End:
