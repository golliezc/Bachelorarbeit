%=============================================================================
% Thesis Template in LaTex
%
% File:  04-07-Kosten -- Analyse der Lösungen - Fallstudie
% Author(s): Jürgen Hackl <hackl@ibi.baug.ethz.ch>
%            Clemens Kielhauser <kielhauser@ibi.baug.ethz.ch>
%
% Creation:  27 Jan 2014
% Time-stamp: <Tue 2013-08-13 20:14 juergen>
%
% Copyright (c) 2014 Infrastructure Management Group (IMG)
%               http://ibi.ethz.ch
%
% More information on LaTeX: http://www.latex-project.org/
%=============================================================================

% Unterkapitel Kosten
% ---------

Die Abbildung \ref{img:Kostenberechnung} zeigt die Kostenberechnung der ersten 8 Jahre am Beispiel der Variante 1 im Szenario SB1/SU1. 
Die jährlichen DTV Werte sind gemäss dem Abschnitt \ref{subsec:Modellierung} berechnet und die Reisezeitverluste $t$ gemäss Abschnitt \ref{sub:Reisezeit}. 

Nach \ref{sub:Unterhalt} werden die Unterhaltskosten, bestehend aus Bau- und jährlichen Wartungskosten berechnet. Die Berechnung der Wartungskosten erfolgt mit den in Abschnitt ref{•} beschriebenen Abmessungen der Variante. Die im Jahr 2020 anfallenden Interventionskosten belaufen sich gemäss Abschnitt ref{•} auf 68'000 CHF.

Die Betriebskosten der Nutzer sind gemäss Abschnitt \ref{sub:Betrieb} berechnet. Die Berechnung erfolgt duch die Mulitplikation des DTV mit der Länge der jeweiligen Fahrbahn und den Einheitskosten des Fahrzeugbetriebs. Um die Kosten eines Jahres zu ermitteln, wird der berechnet Wert mit 365 multipliziert. Die weiteren Kosten der Nutzer sind die Reisezeitkosten, die nach Abschnitt \ref{sub:Reisezeit} berechnet werden. Hierfür wird der im oberen Bereich der Tabelle dargestellte Zeitverlust pro Nutzer, berechnet aus dem Zeitverlust, der durch das befahren der Infrastruktur, gemäss Formel \ref{eq.5}, und durch die durchschnittliche Wartezeit aufgrund der Bahnschranke gemäss Abschnitt \ref{sub:Reisezeit}, entsteht, mit dem DTV und den Einheitskosten des Zeitverlust, multipliziert. Die Berechnung des Zeitverlust nach Formel \ref{eq.5} ist abhängig von der Kapazität, der Annahme zur gefahrenen Durchschnittsgeschwindigkeit und der Länger der Variante.

Die Berechnung der Umweltkosten erfolgt nach Abschnitt \ref{subsec:Environment}. Diese setzten sich zusammen aus, den Lärm- und Schadstoffbelastungskosten. Die Berechnug erfolgt jeweils, durch die multiplikation des DTV mit den jeweiligen Einheitskosten und der Länge der Fahrbahn. Im Fall der Schadstoffbelastungskosten wird, vom MIV DTV der jährliche E-Auto Anteil abgezogen. \\
Die Berechnung der Unfallkosten erfolgt durch die Multiplikation des DTV mit der Länger der Fahrbahn den Unfallriskien gemäss Abschnitt \ref{subsubsec:Unfallrisiko}. Dies erfolgt pro Unfallkategorie. Daraus ergibt sich die jeweilige Unfallanzahl nach Unfallart. Diese werden mit den Einheitskosten der jeweiligen Unfallart mulipliziert und die berechneten Kosten, um die totalen Unfallkosten zu ermitteln, aufsummiert.
 
\begin{figure}[h!]
	\centering
	\includegraphics[width=\textwidth]{figures/04-06-04-Kostenberechnung}
	\caption[Kostenberechnung]{Beispiel der Kostenberechnung}
	\label{img:Kostenberechnung}
\end{figure}



% ===========================================================================
% EOF
%

%%% Local Variables:
%%% mode: latex
%%% TeX-master: "../main"
%%% End:
