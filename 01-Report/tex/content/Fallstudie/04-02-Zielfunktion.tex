%=============================================================================
% Thesis Template in LaTex
%
% File:  Zielfunktion -- Fallstudie
% Author(s): Jürgen Hackl <hackl@ibi.baug.ethz.ch>
%            Clemens Kielhauser <kielhauser@ibi.baug.ethz.ch>
%
% Creation:  27 Jan 2014
% Time-stamp: <Tue 2013-08-13 20:14 juergen>
%
% Copyright (c) 2014 Infrastructure Management Group (IMG)
%               http://ibi.ethz.ch
%
% More information on LaTeX: http://www.latex-project.org/
%=============================================================================

% Unterkapitel Zielfunktion
% ---------
\label{subsec:Funktion}


Das Ziel meiner Optimierung ist es den Gesamtnutzen zu steigern mit speziellem Augenmerk auf der Vermehrung des Nutzens der Fahrradfahrer.

Die geplanten Infrastruktur Interventionen sollen die Kapazität und somit das Angebot auf der Route Bahnhof - Sportanlage erhöhen. 

Mithilfe der Optimierung und der anschliessenden Analyse soll diejenige Intervention bestimmt werden, die den totalen Nutzen über den betrachteten Zeitraum am meisten steigert. 


Die Gleichung \ref{equation:1} stellt das Optimierungsproblem als mathematische Funktion dar.
Die totalen Kosten $TK$ einer Interventionsstrategie sind definiert als die netto Kosten aller Steakholder über einen untersuchten Zeitraum $[0,T]$ 

Da in userem Fall die Erlöse, während einer Zeitperiode $[0,T]$ generiert werden können, nicht in Betracht gezogen werden, ist die minimierung der Gesamtkosten äquivalent zur maximierung des netto Nutzens aller beteiligter Interessensverände. 
Die Zeit $0$ kennzeichnet den Startpunkt der Untersuchung wobei die Zeit $T$ das Ende der Untersuchungsperiode ist. 

\begin{equation}
Min. \thinspace TK_{i} = Min. \thinspace [K_{U}^i + K_{TT}^i + K_{B}^i + K_{E}^i + K_{A}^i]
\label{equation:1}
\end{equation} 

{\setstretch{0.75}
wobei:
\begin{conditions}
\renewcommand{\arraystretch}{0.7}
 TK_{i}   	    &  Totale Kosten der Variante $i$ für den betrachteten Zeitraum von $T$ Jahren \\
 K_{U}^i		&  Totale Unterhalts- und Baukosten der Variante $i$ \\
 K_{TT}^i       &  Totale Reisezeitkosten der Variante $i$    \\
 K_{B}^i        &  Totale Betriebskosten der Variante $i$ \\
 K_{E}^i	    &  Totale Umweltbelastungskosten der Variante $i$  \\
 K_{A}^i        &  Totale Unfallkosten der Variante $i$ 
\end{conditions}
}

Tabelle \ref{tab:t-04-01-Interessensgruppen} listet die Interessensgruppen sowie die Kostenstrukturen auf. Die Kostenstrukturen in Form der Kostentypen und der modelierten Einheitskosten werden in den folgenden Kapitel erläutert .

%=============================================================================
% Thesis Template in LaTex
%
% File:  t-05-01-IsingModel.tex -- Table for the Ising
% Author(s): Juergen Hackl <hackl@ibi.baug.ethz.ch>
%            Clemens Kielhauser <kielhauser@ibi.baug.ethz.ch>
%
% Creation:  27 Jan 2014
% Time-stamp: <Tue 2013-08-13 20:14 juergen>
%
% Copyright (c) 2014 Infrastructure Management Group (IMG)
%               http://ibi.ethz.ch
%
% More information on LaTeX: http://www.latex-project.org/
%=============================================================================

\begin{table}[ht!]
\flushleft
\renewcommand{\arraystretch}{1.4}
%\small\renewcommand{\arraystretch}{1.2} 
%
%
\begin{tabular}{@{}p{3.3cm} p{4cm} p{1.2cm} r l @{}} \\   
\toprule
\textbf{Interessensgruppe} & \textbf{Kostentyp} & \textbf{Symbol} & \multicolumn{2}{c}{\textbf{Einheitskosten}} 			\\
\midrule
Besitzer                   & Unterhaltkosten (\textit{U})                    & $K_{U}(t)$    & 5 - 30 		&	$\frac{CHF}{m^2 \ Jahr}$              \\
Nutzer		               & Reisezeitkosten (\textit{TT})                   & $K_{TT}(t)$   & 35 - 56 		&	$\frac{CHF}{Stunde \ DTV_{k}}$         \\
                           & Betriebskosten (\textit{B})            		 & $K_{B}(t)$    & 0.15 - 0.7 	&	$\frac{CHF}{km \ DTV_{k}}$              \\
Öffentliche Hand           & Kosten durch Belastung \newline der Umwelt \newline (Environment) (\textit{E})   & $K_{E}(t)$    & 0.05  & $\frac{CHF}{Fahrzeugkilometer}$      \\
                           & Unfallkosten (\textit{A})                       & $K_{A}(t)$    & 15'000 - 3.7mio & $\frac{CHF}{Unfall_{n}}$    \\
\bottomrule

\end{tabular}
\caption{Tabelle der Interessensgruppen und Kostenstrukturen}
\label{tab:t-04-01-Interessensgruppen}
\end{table}


%=============================================================================
% EOF
%

%%% Local Variables:
%%% mode: latex
%%% TeX-master: "../guidelines"
%%% End:



\newpage

% ===========================================================================
% EOF
%

%%% Local Variables:
%%% mode: latex
%%% TeX-master: "../main"
%%% End:
