%=============================================================================
% Thesis Template in LaTex
%
% File:  Zielfunktion -- Fallstudie
% Author(s): Jürgen Hackl <hackl@ibi.baug.ethz.ch>
%            Clemens Kielhauser <kielhauser@ibi.baug.ethz.ch>
%
% Creation:  27 Jan 2014
% Time-stamp: <Tue 2013-08-13 20:14 juergen>
%
% Copyright (c) 2014 Infrastructure Management Group (IMG)
%               http://ibi.ethz.ch
%
% More information on LaTeX: http://www.latex-project.org/
%=============================================================================

% Unterkapitel Zielfunktion
% ---------
\label{subsec:Funktion}

Bei einer Optimierung eines Systems sind mathematische Modelle sehr hilfreich. Ein besonders nützliches Modell in der Analyse von Varianten ist ein Lineares Programm. Dies setzt vorraus das alle relevanten Informationen aus der Situationsanalyse und der Zielformulierung in ein mathematisches Modell umgewandelt werden. Im Rahmen dieser Arbeit, erfolgt dies durch die nachfolgend beschriebene Zielfunktion. Die Zielfunktion beinhaltet alle relevanten Kosten der Interessensgruppen der Infrastruktur. (\cite{Adey2019})

Die Gleichung \ref{equation:1} stellt das zu optimierende Problem als mathematische Funktion dar.
Die totalen Kosten $TK$ einer Interventionsstrategie sind definiert als, die netto Kosten aller Steakholder über einen untersuchten Zeitraum $[0,T]$  \\
Da in userem Fall die Erlöse, während einer Zeitperiode $[0,T]$ generiert werden können, nicht in Betracht gezogen werden, ist die Minimierung der Gesamtkosten äquivalent zur Maximierung des netto Nutzens aller beteiligter Interessensverände. 
Die Zeit $0$ kennzeichnet den Startpunkt der Untersuchung wobei die Zeit $T$ das Ende der Untersuchungsperiode ist. 

\begin{equation}
Min. \thinspace TK_{i} = Min. \thinspace [K_{U}^i + K_{B}^i + K_{TT}^i + K_{E}^i + K_{A}^i]
\label{equation:1}
\end{equation} 

{\setstretch{0.75}
wobei:
\begin{conditions}
\renewcommand{\arraystretch}{0.7}
 TK_{i}   	    &  Totale Kosten der Variante $i$ für den betrachteten Zeitraum von $T$ Jahren \\
 K_{U}^i		&  Totale Unterhalts- und Baukosten der Variante $i$ \\
 K_{B}^i        &  Totale Betriebskosten der Variante $i$ \\
 K_{TT}^i       &  Totale Reisezeitkosten der Variante $i$    \\
 K_{E}^i	    &  Totale Umweltbelastungskosten der Variante $i$  \\
 K_{A}^i        &  Totale Unfallkosten der Variante $i$ 
\end{conditions}
}

Mithilfe der Optimierung dieser Zielfunktion und der anschliessenden Analyse der Resultate, wird diejenige Intervention bestimmt, die den totalen Nutzen über den betrachteten Zeitraum am meisten steigert. 
Zur Übersicht fasst die Tabelle \ref{tab:t-04-01-Interessensgruppen} die Interessensgruppen, sowie die jeweiligen Kosten zusammen. Die Kosten, der verschienden Interessensverbände im Zusammenhang mit der Infrastruktur, werden im nachfolgenden Kapitel erläutert.

%=============================================================================
% Thesis Template in LaTex
%
% File:  t-05-01-IsingModel.tex -- Table for the Ising
% Author(s): Juergen Hackl <hackl@ibi.baug.ethz.ch>
%            Clemens Kielhauser <kielhauser@ibi.baug.ethz.ch>
%
% Creation:  27 Jan 2014
% Time-stamp: <Tue 2013-08-13 20:14 juergen>
%
% Copyright (c) 2014 Infrastructure Management Group (IMG)
%               http://ibi.ethz.ch
%
% More information on LaTeX: http://www.latex-project.org/
%=============================================================================

\begin{table}[hbt!]
\flushleft
%\small\renewcommand{\arraystretch}{1.2} 
%
%
\begin{tabular}{@{}p{3.5cm} p{4cm} p{1.35cm} p{4.65cm}@{}} \\   
\toprule
\textbf{Interessensgruppe} & \textbf{Kostentyp} & \textbf{Symbol} & \textbf{Einheitskosten} \\
\midrule
Besitzer                   & Unterhaltkosten (\textit{U})                    & $K_{U}(t)$    & 15'000 $\frac{CHF}{m^2 \ Jahr}$              \\
Nutzer		               & Reisezeitkosten (\textit{TT})                   & $K_{TT}(t)$   & 85 $\frac{CHF}{Stunde \ Jahr}$               \\
                           & Betriebskosten (\textit{B})            		 & $K_{B}(t)$    & 100 $\frac{CHF}{Nutzer \ Jahr}$              \\
Öffentliche Hand           & Kosten durch Belastung \newline der Umwelt \newline (Environment) (\textit{E})   & $K_{E}(t)$    & 5000 $\frac{CHF}{Fahrzeug \ Jahr}$      \\
                           & Unfallkosten (\textit{A})                       & $K_{A}(t)$    & 15'000 - 3.7mio \ $\frac{CHF}{Unfall \ Jahr}$    \\
\bottomrule

\end{tabular}
\caption{Tabelle der Interessensgruppen und Kostenstrukturen}
\label{tab:t-04-01-Interessensgruppen}
\end{table}


%=============================================================================
% EOF
%

%%% Local Variables:
%%% mode: latex
%%% TeX-master: "../guidelines"
%%% End:



\pagebreak

% ===========================================================================
% EOF
%

%%% Local Variables:
%%% mode: latex
%%% TeX-master: "../main"
%%% End:
