%=============================================================================
% Thesis Template in LaTex
%
% File:  Zielfunktion -- Fallstudie
% Author(s): Jürgen Hackl <hackl@ibi.baug.ethz.ch>
%            Clemens Kielhauser <kielhauser@ibi.baug.ethz.ch>
%
% Creation:  27 Jan 2014
% Time-stamp: <Tue 2013-08-13 20:14 juergen>
%
% Copyright (c) 2014 Infrastructure Management Group (IMG)
%               http://ibi.ethz.ch
%
% More information on LaTeX: http://www.latex-project.org/
%=============================================================================

% Unterkapitel Zielfunktion
% ---------
\label{subsec:Funktion}

Bei der Optimierung eines Systems sind mathematische Modelle sehr hilfreich. Ein besonders nützliches Modell in der Analyse von Varianten ist ein Lineares Programm. Dies setzt vorraus das alle relevanten Informationen aus der Situationsanalyse und der Zielformulierung in ein mathematisches Modell umgewandelt werden und mithilfe der Zielfunktion das Problem mathematisch formuliert wird.
Im Rahmen dieser Arbeit, erfolgt dies durch die nachfolgend beschriebene Zielfunktion \ref{equation:1}, welche alle relevanten Kosten der Interessensgruppen der Infrastruktur beinhaltet. (\cite{Adey2019})

\pagebreak

Die totalen Kosten $K$ einer Interventionsstrategie sind definiert als, die Kosten aller Steakholder über den untersuchten Zeitraum $[0,T]$  \\
Da in meinem Fall die Erlöse nicht in Betracht gezogen werden, ist die Minimierung der Gesamtkosten äquivalent zur Maximierung des netto Nutzens aller beteiligten Interessensgruppen. Zusätzlich setzt die Bedingungen \ref{eq.1.1} voraus, dass keine negativen Werte berücksichtigt werden.
Die Zeit $0$ kennzeichnet den Startpunkt der Untersuchung wobei die Zeit $T$ das Ende der Untersuchungsperiode ist. 

\begin{equation}
Min. \thinspace K = Min. \thinspace [K_{U} + K_{B} + K_{TT} + K_{E} + K_{A}]
\label{equation:1}
\end{equation} 

\begin{equation}
K_{U} \geq 0, K_{B} \geq 0, K_{TT} \geq 0, K_{E} \geq 0, K_{A} \geq 0  \label{eq.1.1}
\end{equation}

{\setstretch{0.7}
wobei:
\begin{conditions}
\renewcommand{\arraystretch}{0.7}
 K   	      &  Totale Kosten über den betrachteten Zeitraum von $T$ Jahren \\
 K_{U}		  &  Totale Unterhalts- und Baukosten  \\
 K_{B}        &  Totale Betriebskosten \\
 K_{TT}       &  Totale Reisezeitkosten   \\
 K_{E}	      &  Totale Umweltbelastungskosten \\
 K_{A}        &  Totale Unfallkosten 
\end{conditions}
}

Mithilfe der Optimierung der Zielfunktion und der anschliessenden Analyse der Resultate, wird diejenige Intervention bestimmt, die den totalen Nutzen über den betrachteten Zeitraum am meisten steigert. 
Zur Übersicht fasst die Tabelle \ref{tab:t-04-01-Interessensgruppen} die Interessensgruppen, sowie die jeweiligen Kosten zusammen. Die Kosten, der verschienden Interessensgruppen werden im nachfolgenden Kapitel erläutert.

%=============================================================================
% Thesis Template in LaTex
%
% File:  t-05-01-IsingModel.tex -- Table for the Ising
% Author(s): Juergen Hackl <hackl@ibi.baug.ethz.ch>
%            Clemens Kielhauser <kielhauser@ibi.baug.ethz.ch>
%
% Creation:  27 Jan 2014
% Time-stamp: <Tue 2013-08-13 20:14 juergen>
%
% Copyright (c) 2014 Infrastructure Management Group (IMG)
%               http://ibi.ethz.ch
%
% More information on LaTeX: http://www.latex-project.org/
%=============================================================================

\begin{table}[ht!]
\flushleft
\renewcommand{\arraystretch}{1.4}
%\small\renewcommand{\arraystretch}{1.2} 
%
%
\begin{tabular}{@{}p{3.3cm} p{4cm} p{1.2cm} r l @{}} \\   
\toprule
\textbf{Interessensgruppe} & \textbf{Kostentyp} & \textbf{Symbol} & \multicolumn{2}{c}{\textbf{Einheitskosten}} 			\\
\midrule
Besitzer                   & Unterhaltkosten (\textit{U})                    & $K_{U}(t)$    & 5 - 30 		&	$\frac{CHF}{m^2 \ Jahr}$              \\
Nutzer		               & Reisezeitkosten (\textit{TT})                   & $K_{TT}(t)$   & 35 - 56 		&	$\frac{CHF}{Stunde \ DTV_{k}}$         \\
                           & Betriebskosten (\textit{B})            		 & $K_{B}(t)$    & 0.15 - 0.7 	&	$\frac{CHF}{km \ DTV_{k}}$              \\
Öffentliche Hand           & Kosten durch Belastung \newline der Umwelt \newline (Environment) (\textit{E})   & $K_{E}(t)$    & 0.05  & $\frac{CHF}{Fahrzeugkilometer}$      \\
                           & Unfallkosten (\textit{A})                       & $K_{A}(t)$    & 15'000 - 3.7mio & $\frac{CHF}{Unfall_{n}}$    \\
\bottomrule

\end{tabular}
\caption{Tabelle der Interessensgruppen und Kostenstrukturen}
\label{tab:t-04-01-Interessensgruppen}
\end{table}


%=============================================================================
% EOF
%

%%% Local Variables:
%%% mode: latex
%%% TeX-master: "../guidelines"
%%% End:



\pagebreak

% ===========================================================================
% EOF
%

%%% Local Variables:
%%% mode: latex
%%% TeX-master: "../main"
%%% End:
