%=============================================================================
% Thesis Template in LaTex
%
% File:  Unsichere Parameter -- Fallstudie
% Author(s): Jürgen Hackl <hackl@ibi.baug.ethz.ch>
%            Clemens Kielhauser <kielhauser@ibi.baug.ethz.ch>
%
% Creation:  27 Jan 2014
% Time-stamp: <Tue 2013-08-13 20:14 juergen>
%
% Copyright (c) 2014 Infrastructure Management Group (IMG)
%               http://ibi.ethz.ch
%
% More information on LaTeX: http://www.latex-project.org/
%=============================================================================

% Unterkapitel Unsichere Parameter
% ---------

Die zukünftige Ausprägung einiger Parameter der Kostenstruktur ist abhängig von der Entwicklung ihrer Einflussfaktoren. Nachfolgend aufgelistet sind die wichtigsten Einflussfaktoren für die Situation am Bahnhof Uster. Diese Abhängigkeit bedingt, dass der Wert der die Paramter der Kostenstruktur in Zukunft annehmen werden, nur mit einer gewissen Unsicherheit vorausgesagt werden kann.  \\

{\setstretch{0.6}
\begin{itemize}
\item Bevölkerunswachstum
\item Zentrumsentwicklung
\item Ausbau der Veloparkieranlagen am Bahnhof 
\item Steigerung der Nachfragebeziehung auf der Route Bahnhof - Sportanlage Buchenholz
\item Erhöhte Nachfrage aufgrund der Erweiterung des Spitals
\item Förderung des Langsamverkehrs gemäss STEK 
\end{itemize}
}

Alle diese Einflussfaktoren haben gemeinsam, dass das Ausmass des Einfluss dieser Faktoren auf die Parameter der Kostenstruktur in der Zukunft ungewiss ist. In einfachen Worten: Es kann nicht vorher gesagt werden, um wie viel sich ein Parameter verändert aufgrund der gestiegenen Nachfrage durch den Ausbau des Spitals. Dieser Einfluss muss, um die Unsicherheit hinsichtlich der zukünfiten Mobilitätssituatuon zuberücksichtigen, im Rahmen dieser Projektarbeit modelliert werden. 
 
Aus Zeitgründen betrachte ich den Grossteil der Paramter meiner Kostenstruktur als konstant und beschränke mich auf die wichtigste Variable  meiner Kostenstruktur. Das tägliche Verkehrsaufkommen. Der Einfluss dieser Faktoren auf andere Parameter meiner Kostenstruktur, wird aus Zeitgründen nicht untersucht.Im nachfolgenden Abschnitt wird der Einfluss dieser Faktoren auf den täglichen Verkehr erläutert.

\subsubsection*{Tägliches Verkehrsaufkommen}
\label{subsubsec:DTV}

Das tägliche Verkehrsaufkommen in Uster ist einerseits abhängig von der demographischen Entwicklung, dass heisst vom Wachstum der Bevölkerung.
Weiter beeinflusst das Verkehrsaufkommen die Entwicklung der Nachfragebeziehung auf der Route Bahnhof - Sportanlage, sowie die Entwicklung des Stadtzentrums. Der Ausbau der Veloparkieranalgen am Bahnhof und die Erweiterung des Spitals im Norden von Uster, können das Verkehrsaufkommen am Bahnübergang zusätzlich beeinflussen. Der Bau der Uster Westumfahrung sowie der Bau der Moosackerstrasse haben gemäss STEK keinen Einfluss auf die Menge an Autos die den Bahnübergang passieren. Dies folgt der Annahme, dass die Umleitung des Durchgangsverkehr über die Oberlandstrasse bereits nahezug vollständig erfolgt ist und dass der gemessene tägliche Verkehr hauptsächlich aus Quell-/Zielverkehr des Zentrums besteht. Somit haben diese Bauprojekte keinen Einfluss auf die Berechnung des täglichen Verkehrsaufkommen. (\cite{STEK})

Meines erachtes hat das Bevölkerungswachstum den grössten Einfluss auf das tägliche Verkehrsaufkommen. Aus diesem Grund modelliere ich den Effekt der die demographischen Entwicklung auf die Anzahl Fahrzeuge, die täglich den Bahnübergang passieren, haben wird, mit den drei unter Abschnitt \ref{subsec:Szenarien} beschriebenen Szenarien.

Die weiteren aufgelisteten Einflussfaktoren werden unter dem Stickpunkt \textit{Umsetzung der STEK} zusammengefasst. Die geschieht unter der Annahme, dass diese Einflussfaktoren, hauptsächlich die Nachfrage der Langsamverkehrsteilnehmer steigert.
Unter diesem Punkt zusammengefasst sind somit die Effekte der Entwicklung des Zentrums, der Einfluss des Ausbaus der Veloparkieranlage am Bahnhof und die Massnahmen der Stadt Uster zur Förderung des Langsamverkehrs.

Die Anzahl Fahrzeuge die im Jahr $t$ über den Bahnübergang Brunnenstrasse fahren werden, muss in Abhängigkeit der zukünftige Entwicklung der Einflussfaktoren, modeliert werden.  Dies geschieht mithilfe der in  unter Abschnitt \ref{subsec:Szenarien} erläuterten Szenarien. 

\newpage


% ===========================================================================
% EOF
%

%%% Local Variables:
%%% mode: latex
%%% TeX-master: "../main"
%%% End:
