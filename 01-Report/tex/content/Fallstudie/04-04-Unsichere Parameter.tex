%=============================================================================
% Thesis Template in LaTex
%
% File:  Unsichere Parameter -- Fallstudie
% Author(s): Jürgen Hackl <hackl@ibi.baug.ethz.ch>
%            Clemens Kielhauser <kielhauser@ibi.baug.ethz.ch>
%
% Creation:  27 Jan 2014
% Time-stamp: <Tue 2013-08-13 20:14 juergen>
%
% Copyright (c) 2014 Infrastructure Management Group (IMG)
%               http://ibi.ethz.ch
%
% More information on LaTeX: http://www.latex-project.org/
%=============================================================================

% Unterkapitel Unsichere Parameter
% ---------
\label{subsec:Uncertain}

Die Unsicherheiten der Kostenstrukturen defienieren die unsichren Parameter... diese müssen geschätzt werden :)

Die Länge der Infrastruktur sowie der gewählte Ausbaustandart sind in diesem Fall unsichere Variablen, die von der gebauten Variante abhängig sind. Desweiteren ist für die Abschätzung der jährlich anfallenden Unterhaltskosten die Anzahl an Nutzer zu bestimmen. 
Dies ist von Nöten, um Aussagen betreffend der Geschwindigkeit der Abnützung der Fahrbahn machen zu können. 
Die Bestimmung der Anzahl Nutzer ist Anhand von Schätzungen, basierend auf Messungen an vergelichbaren Infrastrukturobjekten, möglich jedoch nur mit einer gewissen Unsicherheit. Somit sind die Anzahl Nutzer sowie die gewählte Ausbauvariante die unsicheren Rahmenbedingugen die die Unterhaltskosten in Zufunft bestimmen. 
Die zukünftige Anzahl Nutzer dieser Infrastruktur ist desweiteren abhängig vom Bevölkerungswachstum und der Entwicklung von Uster bis 2035 gemäss dem STEK Schlussbericht.Die zu erwartende Bevölkerungsentwicklung kann der Abbildung 63 im STEK Schlussbericht Kapitel 10.3 \textit{Bevölkerungswachstum} entnommen werden.


Die Kapazitätsüberschreitung ist direkt abhängig von der Nachfragebeziehung auf der untersuchten Route. Diese Nachfrage ist wiederum abhängig von der Entwicklung der restlichen Infrastruktur in Uster sowie dem Bevölkerungswachstum.
Somit sind in diesem Falle die unsicheren Rahmenbedingungen die es für die Zukunft zu modellieren gilt, die Anzahl Verkehrsteilnehmer die auf der geplanten Veloroute zukünftig pro Tag, fahren werden, sowie der Ausbaustandart der Infrastruktur in Form des Querschnittsprofil der Strasse. 
Die Anzahl an Nutzer pro Jahr sowie die verlorene Zeit pro Nutzer und Jahr sind die unsicheren Variablen die es für die Berechnung dieser Kosten zu modelieren gilt. Sie sind abhänging von der Entwicklung der restlichen Infrastruktur in und um Uster. 

Die Anzahl an Nutzer ist eine unsichere Variabel der Zukunft, welche abhänging ist von der Entwicklung der restlichen Infrastruktur in und um Uster. Die Totale Anzahl an Fahrzeugen pro Jahr $t$ ist die unsichere Rahmenbedingung die es, in Abhängigkeit der verschiedenen Szenarien zu modelieren gilt. \\
Somit ist die Anzahl an Fahrzeugen die täglich auf der Strasse fahren die unsichere Variable die es für die zukünftige Entwicklung zu modelieren gilt. 
Diese ist Abhänging vom Bevölkerungswachstum, der Entwicklung der Nachfragebeziehung auf der Route Bahnhof - Sportanlage, der Entwicklung des Stadtzentrums und dem Bau der Uster Westumfahrung sowie dem Bau der Moosackerstrasse. Ein weiterer Faktor den es zu erwähnen gilt, ist die Entwicklung der durchschnittlichen Fahrzeugsemissionswerte. Es muss berücksichtigt werden, welcher Trend sich hinsichtlich der Emissionswerte abzeichnet um dies in die Berechnung mit einzubeziehen.

Hier ist zu Vermerken, dass die Anzahl Unfälle pro Jahr mit einer Unsichererheit behaftet sind. Diese resultiert aus der Ungewissheit, um wie viel die Nachfrage auf der Route steigen wird und ab welchem Zeitpunkt die Infrastruktur an ihre Grenzen stossen wird.\\ [2ex]
So ist in diesem Fall die Anzahl Unfälle pro Jahr abhängig von der Anzahl Nutzer pro Jahr. Die Anzahl Nutzer wiederum ist abhängig von der Verkehrstechnischen Entwicklung Usters sowie vom progonstizierten Bevölkerungswachstum. \\
Für die Unfallkosten sind somit die Anzahl Unfälle und demnach die Anzahl Nutzer die unsicheren Randbedingungen die im Falle einer Optimierung modeliert werden müssten.

% ===========================================================================
% EOF
%

%%% Local Variables:
%%% mode: latex
%%% TeX-master: "../main"
%%% End:
