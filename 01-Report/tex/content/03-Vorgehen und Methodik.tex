%=============================================================================
% Thesis Template in LaTex
%
% File:  03-Vorgehen und Methodik -- Vorgehen
% Author(s): Cyrano Golliez <golliezc@student.ethz.ch>
%            
%
% Creation:  27 Jan 2014
% Time-stamp: <Tue 2013-08-13 20:14 juergen>
%
% Copyright (c) 2014 Infrastructure Management Group (IMG)
%               http://ibi.ethz.ch
%
% More information on LaTeX: http://www.latex-project.org/
%=============================================================================

\chapter{Vorgehen und Methodik}
\label{chap:Vorgehen}


Um eine Verbesserung der Verkehrssituation in Uster zu erreichen, müssen die unsicheren zukünftigen Gegebenheiten in der Generierung der Lösungsvarianten berücksichtigt werden. Eine optimale Variante zu erarbeiten, erfordert ein systematisches Vorgehen. 
Für das Bearbeiten der Problemstellung habe ich die Schritte und Überlegungen des Problemlösungsprozesses und der Real Option Methodology verwendet. Die nachfolgende Beschreibung, fast mein Vorgehen kurz zusammen.

\begin{IMleftrightskip}
Zur Bestimmung der Systemgrenzen wird eine Situationsanalyse durchgeführt. Mit dieser wird zum einen die Infrastruktur und zum anderen der Zeithorizont über den die Intervention untersucht wird, ermittelt. Zusätzlich werden die Faktoren, welche die zukünftigen Gegebenheiten in Uster am stärksten beeinflussen, definiert. Dies geschieht unter Berücksichtigung der momentanen Situation und der Annahmen, wie Uster in Zukunft funktionieren wird. Die Analyse schafft die Basis für die Formulierung der Ziele, welche in einem nächsten Schritt festgelegt werden.

Die Ziele legen fest, was mit der Intervention erreicht werden soll. Um zu bestimmen, was effektiv erreicht werden soll, müssen die betroffenen Parteien und ihre Bedürfnisse definiert werden. Für diese sogenannten Interessensgruppen werden alle relevanten Kosten ermittelt. Anhand dieser Kosten wird in einem nächsten Schritt die zu optimierende Zielfunktion definiert. Dann werden die relevanten Kosten definiert und die wichtigsten Einflussfaktoren, deren zukünftige Entwicklung ungewiss ist, bestimmt. 

Die nächste Phase umfasst das kreative Erschaffen von möglichen Lösungsvarianten, welche unter Berücksichtigung der jetzigen Situation erschaffen, werden. Um die optimale Variante bestimmen zu können, müssen diese analysiert, bewertet und verglichen werden. Dazu wird in einem ersten Schritt der Effekt der als unsicher definierten Einflussfaktoren, anhand von Szenarien modelliert. Um die Lösungsvorschläge unter dem Einfluss der verschiedenen Szenarien untersuchen zu können, müssen die Wahrscheinlichkeiten definiert werden, nach welcher ein Szenario eintreten wird. 

Der letzte Schritt zu Bestimmung der optimalen Variante beinhaltet die Bewertung der Lösungen. In dieser Phase wird mithilfe des Entscheidungsbaums das jeweilige Risiko der Varianten ermittelt. Um aufzuzeigen, wie die Veränderung der Parameter der Kostenstruktur das Ergebnis beeinflusst sowie um zu untersuchen, inwiefern sich das Ergebnis ändert, wenn die geschätzte Eintrittswahrscheinlichkeit der Szenarien variiert, wird eine Sensitivitätsanalyse durchgeführt. Eine Sensitivitätsanalyse erfolgt jeweils nur für einen Parameter, um deutlicher aufzeigen zu können, welchen Effekt diese Veränderung auf den Entscheidungsprozess hat. 
\end{IMleftrightskip}

Mit dem hier vorgestellten Vorgehen wird im Rahmen dieser Projektarbeit eine optimale Lösungsvariante zur Verbesserung der Verkehrssituation in Uster aufgezeigt. Nach dem Betrachten der zur Verfügung gestellten Literatur komme ich zum Schluss, dass eine solche Untersuchung bisher so noch nicht durchgeführt wurde. In den betrachteten Arbeiten wird jeweils nur ein Objekt hinsichtlich Flexibilität des Designs auf unsichere zukünftige Gegebenheiten untersucht. Das Bearbeiten eines gesamten Infrastruktursystems mit mehreren, sich gegenseitig beeinflussenden Komponenten wie zum Beispiel verschiedenen Verkehrsteilnehmern, Fussgängerzonen sowie Grün- und Parkanlagen, ist meines Erachtens, so noch nicht durchgeführt worden.

% ===========================================================================
% EOF
%

%%% Local Variables:
%%% mode: latex
%%% TeX-master: "../main"
%%% End:
