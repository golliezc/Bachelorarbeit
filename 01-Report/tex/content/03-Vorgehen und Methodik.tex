%=============================================================================
% Thesis Template in LaTex
%
% File:  03-Vorgehen und Methodik -- Vorgehen
% Author(s): Cyrano Golliez <golliezc@student.ethz.ch>
%            
%
% Creation:  27 Jan 2014
% Time-stamp: <Tue 2013-08-13 20:14 juergen>
%
% Copyright (c) 2014 Infrastructure Management Group (IMG)
%               http://ibi.ethz.ch
%
% More information on LaTeX: http://www.latex-project.org/
%=============================================================================

\chapter{Vorgehen und Methodik}
\label{chap:Vorgehen}


Um eine Verbesserung der Verkehrssituation in Uster zu erreichen, müssen die unsicheren zukünftigen Gegebenheiten in der Generierung der Lösungsvarianten berücksichtigt werden. Eine optimale Variante zuerarbeiten, erfordert ein systematisch Vorgehen. 
Für das bearbeiten der Problemstellung habe ich die Schritte und Überlegungen des Problemlösungsprozess und der Real Option Methodology verwendet. Die nachfolgende Beschreibung, fast mein Vorgehen kurz zusammen.

\begin{IMleftrightskip}
Zur Bestimmung der Systemgrenzen wird eine Situationsanalyse durchgeführt. Mit dieser wird, zum einen die Infrastruktur und zum anderen der Zeithorizont über den die Intervention untersucht wird, ermittelt. Zusätzlich werden die Faktoren welche die zukünftigen Gegebenheiten in Uster am stärksten beeinflussen, definiert. Dies geschieht unter berücksichtigung der momentanen Situation und der Annahmen wie Uster in Zukunft funktionieren wird. Die Analyse schafft die Basis für die Formulierung der Ziele, welche in einem nächsten Schritt festgelegt werden.

Die Ziele legen fest, was mit der Intervention erreicht werden soll. Um zu bestimmen was effektiv erreicht werden soll, müssen die betroffenen Parteien und ihre Bedürfnisse definiert werden. Für diese sogenannten Interessensgruppen werden alle relevanten Kosten und Nutzen ermittelt und anhand dieser relevanten Kosten, wird in einem nächsten Schritt die zu optimierende Zielfunktion definiert.
Darauf folgt die ausführliche Beschreibung der Kosten und die Bestimmung der Einflussfaktoren auf die zukünftigen unsicheren Gegebenheiten. 

Als nächstes folgt die Phase in der mögliche Lösungen generiert werden. In dieser Phase werden die Infrastrukturinterventionen erarbeitet welche die Problemstellung beheben könnten. Dies erfolgt unter Berücksichtigung der jetztigen Situation und der erschaffung von möglichst flexibel designten Lösungsvarianten.

Der nächste Schritt beinhaltet die Analyse der im vorangehenden Schritt erarbeiten Lösungsvarianten. Zuerst gilt es die als unsicher definierten Parameter, anhand von Szenarien zu modellieren. Um die Lösungsvorschläge unter dem Einfluss der verschiedenen Szenarien untersuchen zu können, müssen die Wahrscheinlichkeiten definiert werden, nach welchen ein Szenario eintretten wird. Die Eintrittswahrscheinlichkeit einer Kombination zweier Szenarien, ergibt sich auf der multiplikation der Wahrscheinlichkeiten der einzelnen Szenarien.

Der letzte Schritt beinhaltet die Bewertung der Lösungen. In dieser Phase wird, mithilfe des Entscheidungsbaums das jeweilige Risiko der Varianten ermittelt und die optimale Lösung bestimmt. Diese optimale Lösung wird mit den Grundannahmen bestimmt, dies bedeuted, dass die gefundene Lösung von den getroffenen Annahmen abhängig sein kann. Um aufzuzeigen, wie die Veränderung der Parameter der Kostenstruktur das Ergebniss beeinflusst, sowie um zu untersuchen, infwiefern sich das Ergebniss ändert, wenn die geschätzte Eintrittswahrscheinlichkeit der Szenarien variiert, wird eine Sensitivitätsanalyse durchgeführt.
Eine Sensititvitätsanalyse erfolgt jeweils nur für einen Parameter, um deutlicher aufzeigen zu können, welchen Effekt diese Veränderung auf den Entscheidungsprozess hat. 
\end{IMleftrightskip}

Mit dem hier vorgestellten Vorgehen, wird im Rahmen dieser Projektarbeit eine optimale Lösungsvariante, zur Verbesserung der Verkehrssituation in Uster erschaffen. Nach dem betrachten, der zur verfügung gestellten Literatur, komme ich zum Schluss, dass eine solche Untersuchung, so noch nicht durchgeführt wurde. In den betrachteten Arbeiten wird jeweils nur ein Objekt hinsichtlich der Flexibilität des Designs auf unsichere zukünftige Gegebenheiten untersucht. Das bearbeiten eines gesamten Infrastruktursystems, mit mehreren, sich gegenseitig beeinflussenden Komponenten, wie zum Beispiel verschiedenen Fahrbahnen, Fussgängerzonen sowie Grün- und Parkanlage, ist meines erachtens, so noch nicht durchgeführt worden.

% ===========================================================================
% EOF
%

%%% Local Variables:
%%% mode: latex
%%% TeX-master: "../main"
%%% End:
