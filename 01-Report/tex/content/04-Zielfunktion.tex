%=============================================================================
% Thesis Template in LaTex
%
% File:  2-Theory.tex -- Basic Theory
% Author(s): Jürgen Hackl <hackl@ibi.baug.ethz.ch>
%            Clemens Kielhauser <kielhauser@ibi.baug.ethz.ch>
%
% Creation:  27 Jan 2014
% Time-stamp: <Tue 2013-08-13 20:14 juergen>
%
% Copyright (c) 2014 Infrastructure Management Group (IMG)
%               http://ibi.ethz.ch
%
% More information on LaTeX: http://www.latex-project.org/
%=============================================================================

\chapter{Objective funtion and Steakholder}
\label{chap:background}
Was ist ein Label? \newline
Wie kann ich das ändern? \\ [2ex]
Das stinkt :=)
\section{\textbf{{Einleitung}}}

Das Ziel der von mir untersuchten Infrastruktur Investitionen ist es den Gesamtnutzen der Interessensverbände zu maximieren. \newline Dies mit speziellem Augenmerk auf der Vermehrung des Nutzens der Nutzer dieser Infrastruktur, d.h. der Fahhradfahrer. \newline
Die Nutzen die durch die Anwendung einer Infrastruktur Intervention entstehen sind hier definiert als die Verminderung der Kosten. \newline
Die von mir verwendete Grundlage für die Erstellung dieser Auflistung der Steakholder sowie der Nutzen der jeweiligen Interessensverbände, basiert auf dem Dokument: \\ [2ex]
\noindent\hspace*{10mm}\textit{Structure and Infrastructure Engineering: IM1-HS2019-HO-Adey-et-al-2012-f.pdf} \\ [2ex]
Ich möchte den Nutzen also die Reduktion der Kosten gemäss der Funktionen in diesem Dokument berechnen. 
Die Variablen dieser Funktionen werden ich anhand der Angaben aus dem \textit{STEK} sowie der von mir definierten Unsicherheiten modellieren. \\ [2ex]
Die von mir untersuchten Infrastruktur Investitionen werden sich hinsichtlich ihrer Kapazitäten unterscheiden und ich möchte untersuchen welche Variante die optimale Vermehrung des Gesamtnutzens ermöglicht. \newline
Ist es möglich die Nachfragebeziehungen für das Fahrrad, zwischen dem Bahnhof Uster und der Sportanlage Buchholz mit Hilfe des GIS-Browsers zu modellieren bzw. für die Zukunft zu schätzen? \\ [2ex]
So wäre eine meiner unsicheren Rahmenbedingungen die Anzahl Fahrräder die in Zukunft, \newline (ca. 30-40 Jahre), pro Tag die Strecke Uster Bahnhof - Sportanlage Buchholz befahren.

\newpage

\section{\textbf{{Steakholder}}}

Die Tabelle \ref{table:1} listet die Interessensgruppen sowie die Kostenstrukturen auf. \\ [2ex]


\begin{table}[htbp]
\centering
\begin{tabular}{ |p{3cm} p{4cm} p{2cm} p{3cm}|  }
 \hline
 Interessensgruppen & Art der Nutzen & Symbol & Beschreibung\\ [0.5ex]
 \hline
 
Besitzer der Infrastruktur 
& Reduktion der \newline Unterhaltskosten (U)  & $N_{U}^i(t)$ & Menge an Arbeitsstunden \newline Menge an Arbeitsmaterial\\
\hline
    
Nutzer 
& Reduktion der Reisezeitkosten (TT) & $N_{TT,Velo}^i(t)$ & Kosten     der Reise in Anbetracht      des  Zeitverlust. \newline 
Verlust von Arbeitszeit und Freizeit. \\ 
    
& Reduktion der Fahrzeugbetriebskosten (B) & $N_{B,Velo}^i(t)$ & Arbeitsstunden für Instandhaltung. \newline 
Materialmenge für Instandhaltung.\\ 

& Reduktion der Unfallkosten (A) & $N_{A,Velo}^i(t)$ & Anzahl Materialschäden \newline Anzahl und Art von Personenschäden \newline Anzahl Todefälle \\

& Reduktion der Kosten durch Unbehagen \newline (Discomfort) (D) & $N_{D,Velo}^i(t)$ & Menge der Kosten durch Veränderung des Komforts. \newline Physisch durch holprige Strassen und 
psychologisch aufgrund erhöhter Gefahrenlage \\
\hline

Öffentliche Hand
& Reduktion der Umweltbelastung \newline (Environment) (E)  & $N_E^i(t)$ & Menge an Lärmbelastung \newline Mänge an Luftverschmutzung \newline \textit{Durch Verminderung des MIV-Anteil} \\

& Erhöhung der Fahrkomforts ÖV (K) & $N_{K,\ddot{O}V}^i(t)$ & Reduktion der Anzahl Passagiere im Bus in der Rushhour \\
\hline

\end{tabular}
\caption{Tabelle der Interessengruppen und Kostenstrukturen}
\label{table:1}
\end{table}


Die Gleichung (1) beschreibt den Totalen Nutzen $TN$ einer Interventionsstrategie $I$. Der Totale Nutzen ist definiert als der netto Nutzen aller Steakholder über einen untersuchten Zeitraum [0,T] \newline
Wobei Nutzen hier definiert ist als die Diffenrenz der Kosten vor und nach dem Durchführen einer Intervention. 

\begin{equation}
    TN_{i} = \int_{0}^{T} (N_{U}^i(t) + N_{TT,Velo}^i(t) + N_{B,Velo}^i(t) +
    N_{A,Velo}^i(t) + N_{D,Velo}^i(t) + N_E^i(t) + N_{K,\ddot{O}V}^i(t)) \cdot e^{-\gamma t} dt
\end{equation} \\ [2ex]

Die Zeit $0$ kennzeichnet den Startpunkt der Untersuchung wobei die Zeit $T$ das Ende der Untersuchungsperiode ist. $\gamma$ ist der Diskontsatz.


\newpage





 
\section{\textbf{{Zielfunktion}}}

Das Ziel meiner Optimierung ist es den Gesamtnutzen zu steigern mit speziellem Augenmerk auf der Vermehrung des Nutzens der Fahrradfahrer. \newline
Die geplanten Infrastruktur Interventionen sollen die Kapazität und somit das Angebot auf der Route Bahnhof - Sportanlage erhöhen. \newline
Mithilfe der Optimierung und der anschliessenden Analyse soll diejenige Intervention bestimmt werden, die den totalen Nutzen über den betrachteten Zeitraum am meisten steigert. \\ [4ex]




Formel (2) beschreibt die mathematische Formulierung des Optimierungsproblems.
\\ [2ex]
\begin{equation}
    Max.\thinspace\sum_{a=1}^{A} (TN_{a}^{Max}) \cdot {y_a}
\end{equation} \\ [2ex]

Wobei ${a}$ die betrachtete Intervention ist und ${A}$ die Gesamtanzahl an Interventionen die in betracht gezogen werden. \\ [2ex]
$TB_a^{Max}$ ist der maximale totale Nutzen der generiert werden kann, wenn nur Intervention ${a}$ durchgeführt wird.



% ===========================================================================
% EOF
%

%%% Local Variables:
%%% mode: latex
%%% TeX-master: "../main"
%%% End:
