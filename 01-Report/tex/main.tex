%=============================================================================
% Thesis Template in LaTex
%
% Time-stamp: <Mon 2016-04-04 08:52 juergen>
% File:  main.tex -- Main file of the template
% Author(s): Juergen Hackl <hackl@ibi.baug.ethz.ch>
%            Clemens Kielhauser <kielhauser@ibi.baug.ethz.ch>
%
% Creation:  27 Jan 2014
%
% Copyright (c) 2014 Infrastructure Management Group (IMG)
%               http://ibi.ethz.ch
%
% More information on LaTeX: http://www.latex-project.org/
%=============================================================================

% **************
% Basic settings
% **************

\newcommand{\mylaterality}{twoside}
% "oneside" or "twoside"
% Either you are creating a document which is printed on both, left pages
% and right pages (twoside) or you create a document which is printed
% on right pages only (oneside).

\newcommand{\mylanguage}{english,ngerman}
% "english,ngerman", "ngerman,english", ...
% NOTE: The *last* language is the active one!
% See babel documentation for further details.

\newcommand{\mycolorlinks}{false}  %% "true" or "false"
% Enables or disables colored links (hyperref package).

\newcommand{\mylinenumbering}{false}  %% "true" or "false"
% Enables or disables linenumbers (lineno package).

\newcommand{\myspacing}{false}  %% "true" or "false"
% Enables or disables doublespacing (setspace package).

% General metadata for the PDF
\newcommand{\myauthor}{Cyrano Golliez}  %% also used for PDF metadata (hyperref)
\newcommand{\myethnr}{15-914-609}
\newcommand{\mytitle}{Bedarfsgerechte Optimierung \\ der Veloinfrastruktur Uster}  %% also used for PDF metadata (hyperref)
\newcommand{\mysubject}{SUBJECT was kommt hier}  %% also used for PDF metadatay (hyperref)
\newcommand{\mykeywords}{KEYWORDS}  %% also used for PDF metadata (hyperref)
\newcommand{\myprofessor}{Prof. Dr. Bryan T. Adey}
\newcommand{\mysupervisor}{Dr. Claudio Martani}

% ********
% Preamble
% ********

% Load main settings for document preamble
% ========================================
%=============================================================================
% Thesis Template in LaTex
%
% Time-stamp: <Mon 2016-04-04 07:49 juergen>
% File:  main.tex -- Main file of the template
% Author(s): Juergen Hackl <hackl@ibi.baug.ethz.ch>
%            Clemens Kielhauser <kielhauser@ibi.baug.ethz.ch>
%
% Creation:  27 Jan 2014
%
% Copyright (c) 2014 Infrastructure Management Group (IMG)
%               http://ibi.ethz.ch
%
% More information on LaTeX: http://www.latex-project.org/
%=============================================================================

% ********
% Preamble
% ********

% Documentclass using KOMA-script
% ===============================

\documentclass[
  paper=a4,                         % Paper format
  fontsize=11pt,                    % Fontsize
  DIV=12,                           % Divided page horizontally and vertically
  BCOR=10mm,                        % Binding correction
  \mylaterality,                    % Double or one-sided typesetting
  parskip=half,                     % Parskip
]{scrreprt}                         % KOMA - class


% Usepackages
% ===========

\usepackage[utf8]{inputenc}         % Input caracters
\usepackage[T1]{fontenc}            % Font encodings
\usepackage[ngerman]{babel}         % Language
\usepackage{scrpage2}               % Headers and footers
\pagestyle{scrheadings}             % Headers and footers
\usepackage{graphicx}               % Enhanced support for graphics
\usepackage[export]{adjustbox}
\usepackage{xcolor}                 % Color extensions
\usepackage{enumitem}               % Layout of itemize, enumerate, ...
\usepackage{multicol}               % Multiple columns
\usepackage{subfig}                 % Figures broken into subfigures
\usepackage{rotating}               % Rotation tools
\usepackage{longtable}              % Long tables
\usepackage{array,tabularx}               % Adjustable Tabulars
\usepackage{booktabs}               % Tabular rules
\usepackage{float}                  % Float environment
\usepackage{amsmath}                % AMS mathematical facilities
\usepackage{amssymb}                % AMS symbol fonts
\usepackage{makeidx}                % Index
\usepackage[intoc]{nomencl}         % Nomenclature
\usepackage{acronym}                % Acronyms
\usepackage{tikz}                   % TikZ
\usepackage{pgfplots}               % PGF plots
\usepackage{units}                  % Typeset units
\usepackage{csquotes}               % quotes
\usepackage{wasysym}                % Additional Symbols
\usepackage{listings}
\usepackage{pgfgantt}               % Gantt Chart
\usepackage{pdfpages}               % Includes PDF files
\usepackage{blindtext}              % Some random text
\usepackage{ifthen}                 % If commands



% my own packages
% ---------------
\usepackage{setspace} 				% for {\setstretch{0.75}
									%	  ...(Text/Inhalt)...
									%	  }

\newenvironment{conditions}
  {\par\vspace{\abovedisplayskip}\noindent
   \tabularx{\columnwidth}{>{$}l<{$} @{${}={}$} >{\raggedright\arraybackslash}X}}
  {\endtabularx\par\vspace{\belowdisplayskip}}

%\usepackage[document]{ragged2e}

%\setlength{\arrayrulewidth}{1mm}
%\setlength{\tabcolsep}{18pt}
%\renewcommand{\arraystretch}{1.5}

% BibLaTeX using biber as backend (recommended)
% ---------------------------------------------

%\usepackage[authordate,backend=biber]{biblatex-chicago}
%\bibliography{literature}
%\setlength\bibitemsep{1.5\itemsep}


% BibTeX
% ------

%\usepackage{cite}                   % Citations
\usepackage{natbib}                 % Citation Style
% \usepackage{apacite}                % APA cite Style
% \renewcommand{\bibsection}{}        % No auto headings


% Line spacing
% ------------
\ifthenelse{\equal{\myspacing}{true}}
{
\usepackage[doublespacing]{setspace} % Set line space to 2 
}{}
% \usepackage[onehalfspacing]{setspace} % Set line space to 1.5


% Line numbering
% --------------

\usepackage{lineno}    % numbering the lines
\ifthenelse{\equal{\mylinenumbering}{true}}{\linenumbers}{}


% Font settings
% -------------

%\usepackage[scaled]{helvet}
%\renewcommand*\familydefault{\sfdefault}


% PDF settings
% ------------

\usepackage[
  pdftitle={\mytitle},
  pdfauthor={\myauthor},
  pdfsubject={\mysubject},
  pdfcreator={Accomplished with: pdfLaTeX, biber, tikz and hyperref-package},
  pdfproducer={\myauthor},
  pdfkeywords={\mykeywords},
  a4paper=true,
  pdftex=true,
  backref,
  pagebackref=false,                % creates backward references too
  bookmarks=false,                  %
  bookmarksopen=false,              % Bookmarkcolumn is opened
  pdfpagemode=None,                 % None, UseOutlines, UseThumbs, FullScreen
  plainpages=false,                 %
  pdfdisplaydoctitle=true,          % show documenttitle
  colorlinks=\mycolorlinks,                  % turn on/off colored links
]{hyperref}

% Header
% ======

\automark[section]{chapter}


% Table rule
% ----------

\renewcommand{\arraystretch}{2}
\setlength{\arrayrulewidth}{1pt}
\lightrulewidth=1pt
\heavyrulewidth=.5pt

\usepackage[singlelinecheck=false]{caption}

%\newenvironment{conditions}{\par\vspace{\abovedisplayskip}\noindent\begin{tabular}{>{$}l<{$} @{${}={}$} l}}{\end{tabular}\par\vspace{\belowdisplayskip}}

%\newenvironment{conditions}
%  {\par\vspace{\abovedisplayskip}\noindent
 %  \tabularx{\columnwidth}{>{$}l<{$} @{${}={}$} >{\raggedright\arraybackslash}X}}
  %{\endtabularx\par\vspace{\belowdisplayskip}}




% Colors
% ======

\definecolor{ibiBlue}{RGB}{0,106,175}
\definecolor{ethBlue}{RGB}{31,64,122}
\definecolor{ethLightBlue}{RGB}{18,105,176}
\definecolor{ethGreen}{RGB}{72,90,44}
\definecolor{ethLightGreen}{RGB}{154,197,15}
\definecolor{ethGray}{RGB}{111,111,100}
\definecolor{ethRed}{RGB}{168,50,45}
\definecolor{ethLogo}{RGB}{0,0,0}%{255, 255, 255}
\definecolor{DBaugGreen}{RGB}{169,200,61}
\definecolor{DBaugBlue}{RGB}{42,105,182}

% Hyphenations
% =============

\hyphenation{ex-am-ple hy-phen-ate}


% MISC
% ====
\raggedbottom
\makeindex

\newenvironment{IMleftskip}{\par\begingroup\leftskip=2em}{\par\endgroup}
\newenvironment{IMleftrightskip}{\par\begingroup\leftskip=2em\rightskip=2em}{\par\endgroup}

%=============================================================================
%%% Local Variables:
%%% mode: latex
%%% TeX-master: "main"
%%% End: %% DO NOT REMOVE THIS LINE!

%=============================================================================

% *************
% Main Document
% *************

% Document
% ========

\begin{document}

% roman page numbers
% ------------------
%\frontmatter
\pagenumbering{roman}

% Titlepage
% ---------
%=============================================================================
% Thesis Template in LaTex
%
% Time-stamp: <Mon 2016-04-04 08:51 juergen>
% File:  00-01-b-Titlepage.tex -- Titlepage for a thesis
% Author(s): Juergen Hackl <hackl@ibi.baug.ethz.ch>
%            Clemens Kielhauser <kielhauser@ibi.baug.ethz.ch>
%
% Creation:  27 Jan 2014
%
% Copyright (c) 2014 Infrastructure Management Group (IMG)
%               http://www.ibi.ethz.ch
%
% More information on LaTeX: http://www.latex-project.org/
%=============================================================================


\begin{titlepage}

\input{./figures/f-00-00-Background.tex} %% DO NOT REMOVE THIS LINE!

{
\vspace{5mm}
{\Huge\textcolor{white}{Bachelor Arbeit}}\\[4mm]
{\Huge\textcolor{white}{\textbf{\mytitle}}}

{\color{white} 
%\vspace{40mm}
\begin{tabbing}
Autor: \hspace{20mm} \= \myauthor \hspace{20mm} \= ETH-Nr. \myethnr \\[3mm]
%\> Max Mustermann \> ETH-Nr. XX-XXX-XXX
%\\[3mm]
Supervisors: \> \myprofessor\\[1mm]
\> \mysupervisor\\[3mm]
Date: 29.05.2020
\end{tabbing}
}
}
\vspace{7mm}
%\begin{figure}[ht!]
%  \centering
%  \input{./figures/f-00-01-cover.tex}
%\end{figure}

%\> \today


\end{titlepage}


% ===========================================================================
% EOF
%

%%% Local Variables:
%%% mode: latex
%%% TeX-master: "../main"
%%% End:

% Declaration
% -----------
\cleardoublepage

\begin{figure}[h!]
	\centering
	\includegraphics[width=\textwidth]{content/f-00-02-a-Declaration}
\end{figure}

\cleardoublepage

\begin{figure}[h!]
	\centering
	\includegraphics[width=\textwidth]{content/f-00-02-b-Declaration}
\end{figure}

%\includepdf[pagecommand={\thispagestyle{empty}},fitpaper=true,pages=-]{content/00-02-a-Declaration}

% Acknowledgment
% --------------
\cleardoublepage
\addcontentsline{toc}{chapter}{Acknowledgment}
\input{content/00-03-Acknowledgment}

% Abstract
% --------
\cleardoublepage
\addcontentsline{toc}{chapter}{Abstract}
%=============================================================================
% Thesis Template in LaTex
%
% File:  00-03-Abstract.tex -- Abstract of the Thesis
% Author(s): Juergen Hackl <hackl@ibi.baug.ethz.ch>
%            Clemens Kielhauser <kielhauser@ibi.baug.ethz.ch>
%
% Creation:  27 Jan 2014
% Time-stamp: <Tue 2013-08-13 20:14 juergen>
%
% Copyright (c) 2014 Infrastructure Management Group (IMG)
%               http://www.ibi.ethz.ch
%
% More information on LaTeX: http://www.latex-project.org/
%=============================================================================

\chapter*{Abstract}
\label{chap:abstract}


The aim of this project was to optimize the transport system of Uster by improving a section of the bicycle infrastructure. Therefore the infrastructure of the level crossing Brunnenstrasse as a function of uncertain future demand relations and developed various proposals for improving the situation has been investigated. From this, an optimal variant was derived, which can best satisfy the future mobility needs of the population of Uster.

The optimization of existing infrastructure systems is one of the major challenges of the future of infrastructure management. In most cases, it is not possible to clearly determine what exactly the problems are and what causes them. On the other hand, it is not always completely clear what is to be achieved by a change. In addition, it is not always clear which variant of an optimization is possible at all and which is optimal, taking into account uncertain future developments of the costs and benefits of those involved.

The needs and requirements of the users and owners of the infrastructure can change drastically during its lifetime. The difficulty is therefore to design infrastructure in such a way that it can guarantee an adequate level of performance throughout its life cycle. It is in the general interest that any intervention in existing infrastructure should only be undertaken if the net benefit to all parties involved is maximised.

Within the scope of this project, I have worked out different variants for the existing traffic problems of Uster and determined an optimal solution. In doing so, I drew on the problem-solving process from the Systems Engineering course and on the theory of the objective function, the decision tree and sensitivity analyses. In order to allow the uncertainties of future developments to flow into the decision-making process, the corresponding influencing factors were estimated using forecasts. Subsequently, the influence of various parameter variations on the cost calculation was modelled and the values determined were weighted with the estimated probability of the scenarios occurring. On this basis, the variants available for selection were assessed. Variant 2 proved to be the best option for the future of Uster.




%=============================================================================
% EOF
%

%%% Local Variables:
%%% mode: latex
%%% TeX-master: "../main"
%%% End:


% Zusammenfassung
% --------
\cleardoublepage
\addcontentsline{toc}{chapter}{Zusammenfassung}
%=============================================================================
% Thesis Template in LaTex
%
% File:  00-05-Zusammenfassung.tex -- Zusammenfassung der Thesis
% Author(s): Juergen Hackl <hackl@ibi.baug.ethz.ch>
%            Clemens Kielhauser <kielhauser@ibi.baug.ethz.ch>
%
% Creation:  27 Jan 2014
% Time-stamp: <Tue 2013-08-13 20:14 juergen>
%
% Copyright (c) 2014 Infrastructure Management Group (IMG)
%               http://www.ibi.ethz.ch
%
% More information on LaTeX: http://www.latex-project.org/
%=============================================================================

\chapter*{Zusammenfassung}
\label{chap:Zusammen}

Same as Abstrac but german

%=============================================================================
% EOF
%

%%% Local Variables:
%%% mode: latex
%%% TeX-master: "../main"
%%% End:



% Table of Contents
% -----------------
\cleardoublepage
\tableofcontents

% List of Figures
% ---------------
\cleardoublepage
\listoffigures

% List of Tables
% --------------
\cleardoublepage
\listoftables

% Nomeclature
% -----------
%\addcontentsline{toc}{chapter}{Nomeclature}

% Glossary
% --------
%\addcontentsline{toc}{chapter}{Glossary}

%\cleardoublepage
%\printglossary[type=main]


% arabic page numbers
% ------------------
% \mainmatter
\cleardoublepage
\pagenumbering{arabic}% \setcounter{page}{0}

% Chapter 1
% ---------
%=============================================================================
% Thesis Template in LaTex
%
% File:  01-Einleitung.tex -- Einleitung
% Author(s): Cyrano Golliez <golliezc@student.ethz.ch>
%            
%
% Creation:  27 Jan 2014
% Time-stamp: <Tue 2013-08-13 20:14 juergen>
%
% Copyright (c) 2014 Infrastructure Management Group (IMG)
%               http://ibi.ethz.ch
%
% More information on LaTeX: http://www.latex-project.org/
%=============================================================================

\chapter{Einleitung}
\label{chap:Einleitung}

Problemstellung gemäss IBI
Das Hauptziel dieser Arbeit ist die Optimierung des Velonetz von Uster, durch das Verbessern eines Teilstücks der Veloinfrastruktur. Für diese Veloinfrastruktur soll eine optimale Variante erarbeitet werden, die, nach der Analyse der momenaten Situation in Uster die zukünftigen Bedürfnisse der Bevölkerung von Uster nach Mobilitat am besten befriedigen kann.

Infrastrukturen müssen so gebaut werden, dass sie die Interessen aller beteiligten über einen langen Zeithorizont befriedigen können.

Was ist Aufgabenstellung und Problemstellung

Ich habe mich im Rahmen dieser Fallstudie mit der Frage einer beschäftigt, die Langsamverkehrsinfrastruktur im Zentrum von Uster, trotz unsicherer Zukunft, zu optimieren. Im diese Zielsetzung erreichen zu können, habe ich die Infrastruktur des Bahnübergangs Brunnenstrasse in Abhängigkeit von unsicheren zukünftigen Nachfragebeziehung untersucht und Vorschläge zur Verbesserung der Situation erarbeitet.

Diese wichtige Route verbindet die südlichen Stadteile sowie das Zentrum ideal mit der Sportanalge Buchenhold sowie mit den Institutionen der Gesundheitsmeile auf dem Weg dahin. In der Gegenrichtung verbindet sie die nördlich der Bahngleise gelegenen Quartiere mit dem Zentrum. Dies ist somit eine der zentralen Achsen Usters und ihr ist im Rahmen der Stadtentwicklung besonderes Augenmerk zu schenken, insbesondere im Zuge der Zentrumsentwicklung und der Mobilitässtrateige 2035. 

In enbracht der Bestrebungen aus Uster ein urbanes Regionalzentrum zu machen, ist die Förderung des ökologischen und zukunftsorientierten Langsamverkehr essenziell, insbesondere die Veloförderung. Hinsichtlich der zukünftigen Entwicklung der Mobilität wird das Fahrrad auch auf Strecken bis zu 30km eine entscheidende Rolle bei der Verkehrsmittelwahl spielen. 
Im Rahmen der Entwicklung des Bahnhofzentrum mit besonderem Augenmerk auf den Fahrbeziehungen der Buslinien, der geplanten Kapazität und Lage der Veloparkieranlagen und der Kapazitäten der Wintethurerstrasse sowie der geplanten Umfahrung Uster West wird die Infrastruktur auf der Nord-Süd Achse zum zentralen Element zur Förderung des Langsamverkehrs. Um diese zentrale Achse und auch das Stadtzentrum vom MIV Durchgangsverkehr zu entlasten ist der Ausbau der Uster Westumfahrung unumgänglich. 

Das Ziel der von mir untersuchten Infrastruktur Investitionen ist es den Gesamtnutzen der Interessensverbände zu maximieren. \newline Dies mit speziellem Augenmerk auf der Vermehrung des Nutzens der Nutzer dieser Infrastruktur, d.h. der Langsamverkehr. 

Ich möchte den Nutzen also die Reduktion der Kosten gemäss der Funktionen in diesem Dokument berechnen. 
Die Funktionen werden ich anhand der Angaben aus dem \textit{STEK} im Rahmen der von mir definierten unsicheren Variablen modellieren. \\ [2ex]
Die von mir untersuchten Infrastruktur Investitionen werden sich hinsichtlich ihrer Kapazitäten unterscheiden. Einerseits möchte ich untersuchen welche Variante die optimale Vermehrung des Gesamtnutzens ermöglicht und andererseits welche die optimale Kapazität für eine geplannte Lebensdauer von mind. 40 Jahren ist. 


% ===========================================================================
% EOF
%

%%% Local Variables:
%%% mode: latex
%%% TeX-master: "../main"
%%% End:


% Chapter 2
% ---------
%=============================================================================
% Thesis Template in LaTex
%
% File:  02-Grundlagen und Theorie -- Grundlagen
% Author(s): Cyrano Golliez <golliezc@student.ethz.ch>>
%            
%
% Creation:  27 Jan 2014
% Time-stamp: <Tue 2013-08-13 20:14 juergen>
%
% Copyright (c) 2014 Infrastructure Management Group (IMG)
%               http://ibi.ethz.ch
%
% More information on LaTeX: http://www.latex-project.org/
%=============================================================================

\chapter{Grundlagen und Theorie}
\label{chap:Grundlage}

In diesem Abschnitt wird die theoretische Grundlage erläutert. für das erarbeiten meines methodische Vorgehens, um die Problemstellung zu bearbeiten

\section{Problemlösungszyklus}
\label{sec:Problemzyklus}



\section{Zielfunktion}
\label{sec:Zielf}


\section{Entscheidungsbaum}
\label{sec:Decisiontree}


\section{Sensitivitätsanalyse}
\label{sec:Sensitivität}


\section{Real option methodology}
\label{sec:RealOption}


% ===========================================================================
% EOF
%

%%% Local Variables:
%%% mode: latex
%%% TeX-master: "../main"
%%% End:


% Chapter 3
% ---------
%=============================================================================
% Thesis Template in LaTex
%
% File:  03-Vorgehen und Methodik -- Vorgehen
% Author(s): Cyrano Golliez <golliezc@student.ethz.ch>
%            
%
% Creation:  27 Jan 2014
% Time-stamp: <Tue 2013-08-13 20:14 juergen>
%
% Copyright (c) 2014 Infrastructure Management Group (IMG)
%               http://ibi.ethz.ch
%
% More information on LaTeX: http://www.latex-project.org/
%=============================================================================

\chapter{Vorgehen und Methodik}
\label{chap:Vorgehen}


Um eine Verbesserung der Verkehrssituation in Uster zu erreichen, müssen die unsicheren zukünftigen Gegebenheiten in der Generierung der Lösungsvarianten berücksichtigt werden. Eine optimale Variante zuerarbeiten, erfordert ein systematisch Vorgehen. 
Für das bearbeiten der Problemstellung habe ich die Schritte und Überlegungen des Problemlösungsprozess und der Real Option Methodology verwendet. Die nachfolgende Beschreibung, fast mein Vorgehen kurz zusammen.

\begin{IMleftrightskip}
Zur Bestimmung der Systemgrenzen wird eine Situationsanalyse durchgeführt. Mit dieser wird, zum einen die Infrastruktur und zum anderen der Zeithorizont über den die Intervention untersucht wird, ermittelt. Zusätzlich werden die Faktoren welche die zukünftigen Gegebenheiten in Uster am stärksten beeinflussen, definiert. Dies geschieht unter berücksichtigung der momentanen Situation und der Annahmen wie Uster in Zukunft funktionieren wird. Die Analyse schafft die Basis für die Formulierung der Ziele, welche in einem nächsten Schritt festgelegt werden.

Die Ziele legen fest, was mit der Intervention erreicht werden soll. Um zu bestimmen was effektiv erreicht werden soll, müssen die betroffenen Parteien und ihre Bedürfnisse definiert werden. Für diese sogenannten Interessensgruppen werden alle relevanten Kosten und Nutzen ermittelt und anhand dieser relevanten Kosten, wird in einem nächsten Schritt die zu optimierende Zielfunktion definiert.
Darauf folgt die ausführliche Beschreibung der Kosten und die Bestimmung der Einflussfaktoren auf die zukünftigen unsicheren Gegebenheiten. 

Als nächstes folgt die Phase in der mögliche Lösungen generiert werden. In dieser Phase werden die Infrastrukturinterventionen erarbeitet welche die Problemstellung beheben könnten. Dies erfolgt unter Berücksichtigung der jetztigen Situation und der erschaffung von möglichst flexibel designten Lösungsvarianten.

Der nächste Schritt beinhaltet die Analyse der im vorangehenden Schritt erarbeiten Lösungsvarianten. Zuerst gilt es die als unsicher definierten Parameter, anhand von Szenarien zu modellieren. Um die Lösungsvorschläge unter dem Einfluss der verschiedenen Szenarien untersuchen zu können, müssen die Wahrscheinlichkeiten definiert werden, nach welchen ein Szenario eintretten wird. Die Eintrittswahrscheinlichkeit einer Kombination zweier Szenarien, ergibt sich auf der multiplikation der Wahrscheinlichkeiten der einzelnen Szenarien.

Der letzte Schritt zu Bestimmung der optimalen Variante, beinhaltet die Bewertung der Lösungen. In dieser Phase wird, mithilfe des Entscheidungsbaums das jeweilige Risiko der Varianten ermittelt. Um aufzuzeigen, wie die Veränderung der Parameter der Kostenstruktur das Ergebniss beeinflusst, sowie um zu untersuchen, infwiefern sich das Ergebniss ändert, wenn die geschätzte Eintrittswahrscheinlichkeit der Szenarien variiert, wird eine Sensitivitätsanalyse durchgeführt. Eine Sensititvitätsanalyse erfolgt jeweils nur für einen Parameter, um deutlicher aufzeigen zu können, welchen Effekt diese Veränderung auf den Entscheidungsprozess hat. 
\end{IMleftrightskip}

Mit dem hier vorgestellten Vorgehen, wird im Rahmen dieser Projektarbeit eine optimale Lösungsvariante, zur Verbesserung der Verkehrssituation in Uster erschaffen. Nach dem betrachten, der zur verfügung gestellten Literatur, komme ich zum Schluss, dass eine solche Untersuchung, so noch nicht durchgeführt wurde. In den betrachteten Arbeiten wird jeweils nur ein Objekt hinsichtlich der Flexibilität des Designs auf unsichere zukünftige Gegebenheiten untersucht. Das bearbeiten eines gesamten Infrastruktursystems, mit mehreren, sich gegenseitig beeinflussenden Komponenten, wie zum Beispiel verschiedenen Fahrbahnen, Fussgängerzonen sowie Grün- und Parkanlage, ist meines erachtens, so noch nicht durchgeführt worden.

% ===========================================================================
% EOF
%

%%% Local Variables:
%%% mode: latex
%%% TeX-master: "../main"
%%% End:


% Chapter 4
% ---------
%=============================================================================
% Thesis Template in LaTex
%
% File:  2-Theory.tex -- Basic Theory
% Author(s): Jürgen Hackl <hackl@ibi.baug.ethz.ch>
%            Clemens Kielhauser <kielhauser@ibi.baug.ethz.ch>
%
% Creation:  27 Jan 2014
% Time-stamp: <Tue 2013-08-13 20:14 juergen>
%
% Copyright (c) 2014 Infrastructure Management Group (IMG)
%               http://ibi.ethz.ch
%
% More information on LaTeX: http://www.latex-project.org/
%=============================================================================

%Diese Verteilung hat zu unterschiedlichen Mikrostandortqualitäten der einzelnen Stadtteile geführt, wobei insbesondere das Zentrum und die Quartiere nördlich des Bahnhofes grosses Wachstumspotential haben. Da eine Erweiterung des Siedlungsgebiets nicht vorgesehen ist, wird ein Wachstum nur im Bestand möglich sein. Aufgrund dessen wird die Zukunft von Uster mehrheitlich durch die Entwicklung des Zentrums geprägt.
 
\chapter{Fallstudie}
\label{chap:Fallstudie}

Uster ist die dritt grösste Stadt im Kanton Zürich und liegt östlich des Greifensees. Mit dem Bau des S-Bahnnetz hat die ehemalige, aus mehreren Dörfern zusammen gewachsenen, Industriestradt eine Entwicklung zur Wohn- und Arbeitsstadt durchlaufen.
Aufgrund der angrenzenden Waldstücke und der Nähe zum Greifensee hat Uster einen hohen Freizeit und Erholungswert.
Die Entwicklung der Stadt aus mehreren Dörfern heraus, hat zur Folge, dass viele Einkaufsmöglichkeiten sowie Arbeitsplatzstandorte dezentral verteilt sind. Diese historisch bedingte Verteilung, sowie die verschiedenen Standortqualitäten haben zu unterschiedlich geprägten Stadtteilen geführt. Da keine Erweiterung des Siedlungsgebiets vorgesehen ist, wird ein weiteres Wachstum nur im Bestand möglich sein. (\cite{STEK}))

Um für die Zukunft von Uster ein optimales System zu entwickeln, müssen die Bedürfnisse der Bevölkerung berücksichtigt werden. 
So hat die Aufwertung des Stadtzentrum für die Bevölkerung von Uster oberste Priorität, jedoch soll insbesondere die Poststelle mit dem Auto, trotzt Zentrumsaufwertung, erreichbar bleiben. Zusätzlich wünscht sich die Bevölkerung, dass die nördlich des Bahnofs gelegenen Quartiere besser an das Zentrum angebunden und die Gewerbeflächen in diesen Quartieren erweitert, werden. Dies setzt voraus, dass die Gleisquerung verbessert und die Verbindungen der Stadtteile optimiert wird. Desweiteren hat eine Bevölkerungsbefragung im Jahr 2015 ergeben, dass sich die Stadtbewohner eine zukunftsorientierte Veloinfrastruktur, mit besonderem Augenmerk auf die Optimierung der Situation an den Bahnübergängen wünscht, insbesonder unter Berücksichtigung der neuen Velotypen wie Lastenräder und schnellen E-Bikes.

Um diese Bedürfnisse befriedigen zu könne, wurde im Stadtentwicklungskonzet, kurz STEK, die Ziele und strategischen Stossrichtungen der räumlichen Stadtentwicklung von Uster bis 2035 festgelegt. Der Stadtrat von Uster legt fest, dass auf Grundlage des STEK, die kommunalen Richt- und Nutzungspläne bis 2025 revidiert werden. Folglich nehme ich an, dass, um eine Prognosse für die Zukunft von Uster machen zu können, die Leitziele des STEK berücksichtigt werden müssen. Die Leitziele lauten gemäss (\cite{STEK}) wie folgt: 

\begin{description}
	\item[Stadtidentität]	\textit{Bewahrung und Weitereintwicklung der Vielseitigkeit} Die Stadt soll ihre polyzentrale Struktur behalten und die Vielseitigkeit der Innenstadt soll bewahrt werden. Uster soll in seiner Rolle als Regionalzentrum gestärkt werden, in dem das Wachstum auf das Zentrum und die gut erschlossenen Gebiete von Nänikon beschränkt wird.
\end{description}

\pagebreak

\begin{description}
	\item[Stadtentwicklung]	\textit{Wohnen und Arbeiten finden statt} Das Arbeitsplatzangebot soll sich im Gleichschritt mit dem Wohnungsmarkt entwickeln, um das Verhältniss von zwei Einwohnern auf einen Arbeitsplatz beizubehalten. Im Rahmen der Stadtentwicklung 2035, möchte die Stadt Uster, die zentrumsnahen Gebiete und die Bahnhofsumgebung in ein, mit Wohnungen durchmischtes Arbeitsplatgebiet umgestalten. 
	\item[Landschaft und Erholung] \textit{Grün- und Freiräume vor der Haustüre} Die Uster umgebenden Landschaften sollen erhalten und wo nötig aufgewertet werden und durch attraktive gestaltete Freiräume im Siedlungsgebiet, sowie durch gezieltes aufwerten der Erholungsräume, den Nutzungsdruck auf die Naturräume gemildert werden. 
	\item[Mobilität] \textit{Uster steigt um!} Um die Kapazitätsengpässe im bestehenden Verkehrsnetz zu mildern, erwägt Uster einen Umstieg vom motorisierten Individualverkehr kurz. MIV auf den öffentlichen Verkehr, kurz. ÖV, und auf den Langsamverkehr, sprich Velo- und Fussgängerverkehr. Die Stadt Uster setzt sich zum Ziel, eine Reduktion des MIV Anteil am Modalsplit des innerstädtischen Verkehr zuerreichen und den Langsam- sprich Veloverkehr nachhaltig zu fördern. Dies geschieht durch die Verbesserung der Routen und Fahrbedingungen des Veloverkehrs. Insbesondere im Zentrum, wird die Verkehrsführung angepasst, um einerseits die Erreichbarkeit mit dem Velo zu verbessern und andererseits die Aufenthaltsqualität durch die lokale Verkehrsberuhigungen zu erhöhen.
\end{description}

Im Rahmen des Leitbild \textit{Stadtraum Uster 2035} werden im STEK sogenannte Schlüsselprojekte definiert. Als Schlüssprojekte bezeichnet, werden Interventionen, die durch ihre Ausführung, in ihrer Umgebung eine weitere Entwicklung auslösen sollen. Die wichtigsten Schlüsselprojekte sind; das Bahnhofsgebiet, das verkehrsberuhigte Zentrum, das Zeughausareal, die Erhohlungsachse Aabach, die urbane Strassenraumgestaltung im Zentrumsgebiet und die Fuss- und Velounterführung Brunnen-/Bahnhofstrasse, sowie die beiden kantonalen Projekte zur Stadterschliessung; Usterwest und Umfahrung Moosackerstrasse. Das Ausmass, der aufgrund der Ausführung dieser Projekte entstehenden Auswirkungen auf die Verkehrsleistung in Uster, kann nicht mit absoluter Sicherheit vorher gesagt werden. (\cite{STEK}) \\

Die Stadtentwicklung sieht vor, dass das Zentrum von Uster attraktiver gestaltet werde soll, um dadurch die Nachfrage vor Ort zu steigern. Durch die Aufwertung des Strassenraum und durch die Massnahmen zur Verkehrsberuhigung, wird das Zentrum für den Langsamverkehr besser erreichbar, wobei die Erreichbarkeit mit dem Auto gewahrt bleiben soll. Die Umgestaltung der Strassenräume der Innenstadt zu urbanen Begegnungszonen, erfordert eine Anpassung der Verkehrsregime. \\
Aufgrund dessen, dass die Versorgungslage im Stadtzentrum die Standortqualitäten von Uster als Wohn- und Arbeitsstadt beeinflusst, hat diese Aufwertung höchste Priorität. \\
Gemäss des STEK haben die bahnhofsnahen Grundstücke, insbesondere die Quartiere nördlich des Bahnhof, das grösste Wachstumspotential. Die Umnutzung dieser Grundstücke infolge der Aufwertung des Stadtzentrums wird, mit grösster Wahrscheinlichkeit zu einer Erhöhung der Verkehrsbelastung am Bahnhof und den zentrumsnahen Vekehrsnetzen, führen. Insbesondere auf dem Velonetz, ist infolge dieser Entwicklung, mit einer zukünftig erhöhten Belastung zurechnen. 
(\cite{STEK})

\pagebreak

\paragraph{Verkehr} ist in Uster ein politisch äusserst umstrittenes Thema. Das Zentrum ist stark geprägt durch den MIV und ein nahezu flächendeckendes Tempo 30 Regime. Der Quell- und Zielverkehr in Zentrum und der hauptsächlich in Nord-Süd Richtung erfolgende Durchgangsverkehr, haben ein hohes Verkehrsaufkommen im Zentrum zur Folge. Die Aufenthaltsqualität auf den Hauptrouten des Zentrums, wie zum Beispiel die Bankstrasse oder die Bahnhofstrasse, ist durch die hohe Verkehrsbelastung reduziert. Insbesondere die Bankstrasse ist zu Spitzenbelastungszeiten ein Nadelöhr im ÖV-Netz, da das grosse Verkehrsaufkommen, das An- und Abfahren der Busse erschwert. Zusätzlich erreichen alle Buslinien, um möglichst kurze Wartzeiten zu ermöglichen, den Bahnhof zur selben Zeit. Dies hat eine hohe Belastung der bahnhofsnahen Verkehrsinfrastruktur zur Folge. \\
Durch die Kombination aus S-Bahn und sechs Stadtbuslinien, erschliesst Uster durch den öffentlichen Verkehr in grossen Teilen. Jedoch ist die Fahrplanstabilität des Busse, durch den Stau, infolge des grosse Anteil des MIV am innerstädtischen Verkehr, beeinträchtigt. (\cite{STEK})

Der MIV-Anteil am innerstädtischen Ziel-, Quell- und Binnenverkehr beträgt gemäss STEK 57\%. Demnach sind die Verkehrsprobleme von Uster mehrheitlich selbst verursacht. Zusätzlich hat das sternförmig angelegte Strassennetz zur Folge, dass der Nord-Süd Durchrgangsverkehr durch das Stadtzentrum geführt wird. Die wichtigsten Knotenpunkte, wie zum Beispiel der Nüsslikreisel, der Nashornkreisel sowie die Seefeldstrasse, geraten in den Spitzenbelastungszeiten an ihre Kapazitätsgrenzen. \\
Die Situation wird in Zukunft durch die Zunahme des Verkehrsaufkommens aufgrund des Bevölkerungs- und Arbeitsplatzwachstums, noch weiter verschärft. 

Um eine ausreichende Mobiliät gewährleisten zu können sowie die Umweltbelastung in der Stadt zu reduzieren, hat Uster im Rahmen des Gesamtverkehrskonzept des Kanton Zürichs, zwei Ziele formuliert. Zum einen soll die Erreichbarkeit der urbanen Räume verbessert und zum anderen, durch gezielte Eingriffe eine Erhöhung des Langsamverkehrsanteil am Gesamtverkehrsaufkommen erwirkt werden. Das Langsamverkehrsnetz, sprich Velo- und Fusswegenetz, ist gemäss dem STEK, vorallem auf den kurzen und zentrumsnahen Hauptrouten zu stärken. Gemäss kantonalem Richtplan soll der Anteil des Langsamverkehrs am Gesamtverkehrsaufkommen, von 20\% (2011) auf 22\% (2030) erhöht werden. Dies bedingt, dass der Modalsplit des Innerstädtischenverkehr zugunsten des Langsamverkehrs verändert wird, jedoch sollen die Kapazitäten des MIV weder erhöht, noch merklich reduziert werden. 

\begin{figure}[h!]
	\centering
	\includegraphics[width=0.6\textwidth]{figures/f-04-04-UsterWest-Moosackerstr}
	\caption[Strassenprojekte Uster]{Strassenprojekte Moosackerstrasse und Usterwest gemäss (\cite{STEK})}
	\label{img:Strassenprojekte}
\end{figure}

Die Abbildung \ref{img:Strassenprojekte} zeigt die geplanten kantonalen Strassenprojekte Usterwest und Moosackerstrasse. Durch diese Strassenprojekte, sollen einerseits die Stadterschliessung verbessert und andererseits das Zentrum vom Durchgangsverkehr entlastet werden. Insbesondere die Verkehrsbelastung des Nüsslikreis, soll durch den Bau der Moosackerstrasse reduziert werden. Gemäss dem STEK ist die Realisierung der Uster Westumfahrung in näherer Zukunft nicht absehbar. Mit dem Bau der Moosackerstrasse hingegen, kann in näherer Zukunft gerechnet werden, was zu einer Reduktion des Durchgangsverkehr im Zentrum und einer Entlastung des Nüsslikreisel führen wird. Laut dem STEK wird die Situation an den bestehenden Bahnübergängen, durch den Bau der Moosackerstrasse nicht verbessert. Um die Situation an den Bahnübergängen nachhaltig zu verbessern, muss ein anderer Lösungsansatz gefunden werden. 

\begin{figure}[h!]
	\centering
	\includegraphics[width=\textwidth]{figures/f-04-01-Veloweg-Alltag}
	\caption[Velonetz Alltag]{Velonetz Alltag (\cite{GIS})}
	\label{img:Velonetz}
\end{figure}
 
Die Abbildung \ref{img:Velonetz} zeigt das Velonetz der Innenstadt von Uster. In grün sind die Hauptverbindungen, in blau die Nebenverbindungen und in violett die Veloschnellroute dargestellt. Wie in der Abbildung ersichtlich, ist der Bahnübergang Brunnenstrasse der zentrale Knotenpunkt des Velonetz. Die Gleisquerung ist für den Langsamverkehr, aufgrund des dichten S-Bahn Fahrplans, nur mit langen Wartezeiten möglich. 

Gemäss der Leitziele, die sich Uster im Rahmen der Stadtenticklung 2035 gegeben hat, soll Uster zur Velostadt ausgebaut werden. Die von der SP eingereichte Velointiative wiederspiegelt zusätzlich das Bedürfnis der Bevölkerung nach einer Förderung des Langsamverkehrs. Insbesondere die, im kantonalen Städtevergleich, als unterdurchschnittlich erachtete Dichte des Velonetz, stellt für die Bevölkerung ein Mangel dar, sowie ist die Verkehrssicherheit, aufgrund der stark am MIV ausgerichten Strassenräume, auf dem bestehenden Velonetz ungenüngend. \\

\paragraph{Fokus} 

Nach der Analyse des STEK bin ich zum Schluss gekommen, dass die Zerschneidung der Stadt durch die Gleisanlage eines der grössten Probleme von Uster darstellt. Die Verbesserung der Gleisquerung und Erhöhung der Sicherheit auf dem bestehenden Velonetz und der damit einhergehenden Erhöhung der Erreichbarkeit des Zentrums, erachte ich optimaler Ansatz, um eine Verbesserung der Verkehrssituation in Uster zu erreichen. 

Unter Anbetracht der ersten Bestrebungen der Stadt Uster den Verkehr im Zentrum zu reduzieren, sprich die Umleitung des Nord-Süd Durchgangsverkehr über die Oberlandstrasse zur Unterführung Dammstrasse, und der Leitziele des STEK "den Langsamverkehr zu fördern", habe ich mich dazu entschieden, die Situation am Bahnübergang Brunnenstrasse zu optimieren.


\begin{figure}[h!]
	\centering
	\includegraphics[width=0.6\textwidth]{figures/f-04-01-Bahnübergang-FOTO}
	\caption[Bahnübergang Brunnenstrasse]{Bahnübergang Brunnenstrasse und Velonetz Alltag (\cite{GIS})}
	\label{img:Brunnenstrasse}
\end{figure}

Die Abbildung \ref{img:Brunnenstrasse} zeigt den Bahnübergang Brunnenstrasse, die die südlich des Bahnhofs gelegenen Stadteile mit dem Spital und der Sportanlage Buchholz, sowie die nördlich des Bahnhofs gelegenen Quartiere mit dem Stadtzentrum und dem Greifensee, verbinden. 

Die im Rahmen dieser Projektarbeit generierten Optimierungsvorschläge, sollen die Verkehrssituation nachhaltig verbessern. Deshalb untersuche ich den Effekt dieser Varianten auf die Verkehrssituation über einen Zeitraum von vierzig Jahren. Um das beste Investment für die Stadt Uster, ermitteln zu können, definiere ich im nachfolgenden Abschnitt, die Ziele die mit der Intervention erreicht werden sollen.

\pagebreak

\section{Formulierung der Ziele und Rahmenbedingungen}
\label{sec:Zielformulierung}

In Anbetracht der dezentralen Struktur von Uster und der geplanten Aufwertung der Innenstadt, ist eine Verbesserung der Erreichbarkeit des Zentrums unabdingbar. Eine Verbesserung der Erreichbarkeit setzt voraus, dass die Reisezeit verkürzt und die Verkehrssicherheit erhöht wird. Durch die Umgestaltung des Bahnübergangs und der Verbesserung der Signalisation, soll der Teilung der Stadt durch die Gleisanlagen entgegengewirkt werden und so ein wichtiger Beitrag zur Stadtentwicklung geleistet werden. Dies soll eine Verkehrsberuhigung der zentrumsnahen Verkehrsnetze fördern und durch die punktuelle Aufwertung die Aufenthaltsqualität steigern.

Die von mir geplante Intervention in das bestehende Infrastrukturnetz, soll die Verkehrssituation am Bahnhof nachhaltig optimieren, sowie die Attraktivität und die Standortqualitäten von Uster stärken. Das Ziel ist, unter Berücksichtigung der unsicheren zukünftigen Gegebenheiten, der gesteigerten Nachfrage nach Mobilität Rechnung zu tragen, sowie den Langsamverkehrsanteil am Modalsplit des Innenstadtverkehrs zu beeinflussen.
 
Diese Optimierung soll, durch die minimierung der Kosten, den Gesamtnutzen aller Interessensgruppen steigern. Aufgrund dessen, dass in der nachfolgenden Berechnung, nur die Kosten der beteiligten Personen berücksichtigt werden, ist die Minimierung der Kosten mit der Maximierung des Nutzens gleichzusetzen. \\
So sollen einerseits, durch die Verkürzung der Reisezeit die Kosten der Nutzer gesenkt und durch die Erhöhung der Verkehrssicherheit die Kosten der Öffentlichkeit, reduziert werden und andererseits durch die Bestimmung der für den Zeitraum der nächsten vierzig Jahre optimalen Varianten, für die Besitzer das optimale Investment ermittelt werden.


	\subsection{Interessensgruppen}
	%=============================================================================
% Thesis Template in LaTex
%
% File:  Interessensgruppen -- Fallstudie
% Author(s): Jürgen Hackl <hackl@ibi.baug.ethz.ch>
%            Clemens Kielhauser <kielhauser@ibi.baug.ethz.ch>
%
% Creation:  27 Jan 2014
% Time-stamp: <Tue 2013-08-13 20:14 juergen>
%
% Copyright (c) 2014 Infrastructure Management Group (IMG)
%               http://ibi.ethz.ch
%
% More information on LaTeX: http://www.latex-project.org/
%=============================================================================

% Unterkapitel Interessengruppen
% ---------
\label{subsec:Gruppen}

Der Nutzen der von einer Intervention ausgeht, ist definiert, als die positive oder negative Auswirkung, die das Ausführen einer solchen Intervention auf die beteiligten Personen hat. Demzufolge entspicht der Nutzen einer Intervention, dem Effekt der Veränderung der aktuellen Situation. Um den Mehrwert einer Intervention ermitteln zu können, muss in einem ersten Schritt die beteiligten Personen bestimmt werden. Die sogenannte Interessensgruppen, werden gemäss Abschnitt \ref{sec:Inter.gruppen} bestimmt und im nachfolgenden Abschnitt kurz erläutert, sowie die wichtigsten Kosten der beteiligten Parteien vorgestellt.

\paragraph{Besitzer}

Die Interessensgruppe der Besitzer setzt sich aus verschiedene Parteien zusammen. Die wichtigsten involvierten Parteien sind die Stadt Uster und der Kanton Zürich als Eigentümer der Strasseninfrastruktur und die SBB als Besitzerin der Bahninfrastruktur. 
Ausserdem sind Sie die wichtigsten Akteure im politischen Diskurs, über die Notwendigkeit einer Veränderung der Infrastruktur und haben dementsprechend einen grossen Einfluss auf den Entscheid, welche Variante gebaut werden soll. \\ 
Der Besitzer der Infrastruktur bezahlt einerseits den Bau der Infrastrukturintervention und ist andererseits dafür verantwortlich, dass die Seviceleistung der Infrastruktur, über den betrachteten Zeithorizont auf einem angemessenen Niveus, gewährleistet ist. Das bedeuted, dass er für die Kosten der Wartung und Instandhaltung der Infrastruktur aufkommen muss. Die Kosten die dem Besitzer über den betrachteten Zeitraum entstehen, setzten sich demnach aus den Arbeits- und Materialstunden für die jährliche Wartung und den Kosten den der Bau einer Variante verursacht, zusammen. 


\paragraph{Nutzer}

Die Nutzer der Infrastruktur sind die Velo- und Autofahrer welche den Bahnübergang passieren. Der Velofahrer repräsentiert im Rahmen dieser Untersuchung gemäss der STEK, sämtliche Verkehrsteilnehmer des Langsamverkehrs. Langsamverkehr bedeuted, dass der Antrieb ausschliesslich durch Muskelkraft erfolgt. Davon ausgenommen sind E-Bikes mit einer zulässigen Höchstgeschwindigkeit von maximal 35 $km/h$. \\
Das befahren einer Infrastruktur kann dem Nutzer Kosten verursachen. In anbetracht der momentanen Verkehssituation am Bahnübergang sind die massgebenden Kosten die dem Nutzer entstehen, zum einen die Kosten durch verlängerte Reisezeiten, die sogenannten Reisezeitkosten ($TT$) (engl. Travel time cost) und die Kosten die durch den Betrieb des Fahrzeugs beim befahren der Infrastruktur entstehen.  
Der Nutzer hat das grösste Interesse an einer Veränderung der aktuellen Verkehrssituation und ist am stärksten von einer Veränderung betroffen. Demnach ist es unerlässlich die Kosten die dem Nutzer entstehen, in der Berechnung der Kosten einer Intervention zu berücksichtigen.


\paragraph{öffentlichkeit}

Die Öffentlichkeit setzt sich aus den direkt oder indirekt von der Infrastruktur betroffenen Personen zusammen. 
Die direkt betroffenen sind zum einen die Anwohner am Bahnübergang und zum anderen die Inhaber von Geschäften und Restaurants in unmittelbarer Umgebung der Infrastruktur. Sie werden durch die Schadstoffemissionen und die Lärmemission, die von der Infrastruktur verursacht werden, beeinträchtig und tragen die Kosten die durch diese Belastung entstehen. 
Die indirekt betroffene Öffentlichkeit stellt im Rahmen dieser Untersuchung die gesamte Bevölkerung der Stadt Uster dar. 
Sie trägt einerseits die Kosten durch die Belastung der Umwelt mit und zusätzlich die entstehenden Gesundheitskosten durch Unfälle auf der Infrastruktur. \\
Die indirekt betroffene Öffentlichkeit nutzt die Infrastruktur nicht und ist auch nicht in ihrer Nähe zuhause oder bei der Arbeit, sondern wird indirekt durch die Nutzung der Infrastruktur beeinträchtigt. \\
Zur Vereinfachung werden die direkt und indirekt betroffenen Personen unter dem Stichpunkt \textit{Öffentlichkeit} zusammengefasst.
 


% ===========================================================================
% EOF
%

%%% Local Variables:
%%% mode: latex
%%% TeX-master: "../main"
%%% End:


	\subsection{Zielfunktion}
	%=============================================================================
% Thesis Template in LaTex
%
% File:  Zielfunktion -- Fallstudie
% Author(s): Jürgen Hackl <hackl@ibi.baug.ethz.ch>
%            Clemens Kielhauser <kielhauser@ibi.baug.ethz.ch>
%
% Creation:  27 Jan 2014
% Time-stamp: <Tue 2013-08-13 20:14 juergen>
%
% Copyright (c) 2014 Infrastructure Management Group (IMG)
%               http://ibi.ethz.ch
%
% More information on LaTeX: http://www.latex-project.org/
%=============================================================================

% Unterkapitel Zielfunktion
% ---------
\label{subsec:Funktion}

Bei der Optimierung eines Systems sind mathematische Modelle sehr hilfreich. Ein besonders nützliches Modell in der Analyse von Varianten ist ein Lineares Programm. Dies setzt vorraus das alle relevanten Informationen aus der Situationsanalyse und der Zielformulierung in ein mathematisches Modell umgewandelt werden und mithilfe der Zielfunktion das Problem mathematisch formuliert wird.
Im Rahmen dieser Arbeit, erfolgt dies durch die nachfolgend beschriebene Zielfunktion \ref{equation:1}, welche alle relevanten Kosten der Interessensgruppen der Infrastruktur beinhaltet. (\cite{Adey2019})

\pagebreak

Die totalen Kosten $K$ einer Interventionsstrategie sind definiert als, die Kosten aller Steakholder über den untersuchten Zeitraum $[0,T]$  \\
Da in meinem Fall die Erlöse nicht in Betracht gezogen werden, ist die Minimierung der Gesamtkosten äquivalent zur Maximierung des netto Nutzens aller beteiligten Interessensgruppen. Zusätzlich setzt die Bedingungen \ref{eq.1.1} voraus, dass keine negativen Werte berücksichtigt werden.
Die Zeit $0$ kennzeichnet den Startpunkt der Untersuchung wobei die Zeit $T$ das Ende der Untersuchungsperiode ist. 

\begin{equation}
Min. \thinspace K = Min. \thinspace [K_{U} + K_{B} + K_{TT} + K_{E} + K_{A}]
\label{equation:1}
\end{equation} 

\begin{equation}
K_{U} \geq 0, K_{B} \geq 0, K_{TT} \geq 0, K_{E} \geq 0, K_{A} \geq 0  \label{eq.1.1}
\end{equation}

{\setstretch{0.7}
wobei:
\begin{conditions}
\renewcommand{\arraystretch}{0.7}
 K   	      &  Totale Kosten über den betrachteten Zeitraum von $T$ Jahren \\
 K_{U}		  &  Totale Unterhalts- und Baukosten  \\
 K_{B}        &  Totale Betriebskosten \\
 K_{TT}       &  Totale Reisezeitkosten   \\
 K_{E}	      &  Totale Umweltbelastungskosten \\
 K_{A}        &  Totale Unfallkosten 
\end{conditions}
}

Mithilfe der Optimierung der Zielfunktion und der anschliessenden Analyse der Resultate, wird diejenige Intervention bestimmt, die den totalen Nutzen über den betrachteten Zeitraum am meisten steigert. 
Zur Übersicht fasst die Tabelle \ref{tab:t-04-01-Interessensgruppen} die Interessensgruppen, sowie die jeweiligen Kosten zusammen. Die Kosten, der verschienden Interessensgruppen werden im nachfolgenden Kapitel erläutert.

%=============================================================================
% Thesis Template in LaTex
%
% File:  t-05-01-IsingModel.tex -- Table for the Ising
% Author(s): Juergen Hackl <hackl@ibi.baug.ethz.ch>
%            Clemens Kielhauser <kielhauser@ibi.baug.ethz.ch>
%
% Creation:  27 Jan 2014
% Time-stamp: <Tue 2013-08-13 20:14 juergen>
%
% Copyright (c) 2014 Infrastructure Management Group (IMG)
%               http://ibi.ethz.ch
%
% More information on LaTeX: http://www.latex-project.org/
%=============================================================================

\begin{table}[hbt!]
\flushleft
%\small\renewcommand{\arraystretch}{1.2} 
%
%
\begin{tabular}{@{}p{3.5cm} p{4cm} p{1.35cm} p{4.65cm}@{}} \\   
\toprule
\textbf{Interessensgruppe} & \textbf{Kostentyp} & \textbf{Symbol} & \textbf{Einheitskosten} \\
\midrule
Besitzer                   & Unterhaltkosten (\textit{U})                    & $K_{U}(t)$    & 15'000 $\frac{CHF}{m^2 \ Jahr}$              \\
Nutzer		               & Reisezeitkosten (\textit{TT})                   & $K_{TT}(t)$   & 85 $\frac{CHF}{Stunde \ Jahr}$               \\
                           & Betriebskosten (\textit{B})            		 & $K_{B}(t)$    & 100 $\frac{CHF}{Nutzer \ Jahr}$              \\
Öffentliche Hand           & Kosten durch Belastung \newline der Umwelt \newline (Environment) (\textit{E})   & $K_{E}(t)$    & 5000 $\frac{CHF}{Fahrzeug \ Jahr}$      \\
                           & Unfallkosten (\textit{A})                       & $K_{A}(t)$    & 15'000 - 3.7mio \ $\frac{CHF}{Unfall \ Jahr}$    \\
\bottomrule

\end{tabular}
\caption{Tabelle der Interessensgruppen und Kostenstrukturen}
\label{tab:t-04-01-Interessensgruppen}
\end{table}


%=============================================================================
% EOF
%

%%% Local Variables:
%%% mode: latex
%%% TeX-master: "../guidelines"
%%% End:



\pagebreak

% ===========================================================================
% EOF
%

%%% Local Variables:
%%% mode: latex
%%% TeX-master: "../main"
%%% End:

	
\pagebreak

\section{Kostenstruktur}
\label{sec:Kostenstruktur}

Um das Risiko das von der Durchgführung einer Intervention ausgeht, berechnen zu können, muss die verwendete Kostenstruktur definiert werden. 
Im allgemeinen erfolgt die approximation der Kosten durch die Bestimmung der relevanten Faktoren, wie zum Beispiel der Länge und der Breite der Infrastruktur, des täglichen Verkehrsaufkommens, der benötigten Reisezeiten sowie weiterer Faktoren. Die ermittlung der Kosten erfolgt im Anschluss durch die Multiplikation dieser Faktoren mit den dazugehörigen Einheitskosten. (\cite{Adey2012}) 

%=============================================================================
% Thesis Template in LaTex
%
% File:  Kostenstruktur -- Fallstudie
% Author(s): Jürgen Hackl <hackl@ibi.baug.ethz.ch>
%            Clemens Kielhauser <kielhauser@ibi.baug.ethz.ch>
%
% Creation:  27 Jan 2014
% Time-stamp: <Tue 2013-08-13 20:14 juergen>
%
% Copyright (c) 2014 Infrastructure Management Group (IMG)
%               http://ibi.ethz.ch
%
% More information on LaTeX: http://www.latex-project.org/
%=============================================================================

% Unterkapitel Kostenstruktur
% ---------
\label{subsec:Kosten}

\newpage

In diesem Kapitel werden die Kosten, die für die verschienden Interessensverbände enstehen, erläutert und ihre Berechnung dargestellt.  \\
Um das Risiko, welches von einer Infrastruktur Intervention ausgeht, berechnen zu können, muss die verwendete Kostenstruktur definiert werden. 
Im allgemeinen erfolgt die approximation der Kosten durch die Bestimmung der relevanten Faktoren wie zum Beispiel der Länge und der Breite der Infrastruktur, des täglichen Verkehrsaufkommens, der benötigten Reisezeit sowie weiterer Faktoren. Die ermittlung der Kosten erfolgt im Anschluss durch die Multiplikation dieser Faktoren mit den dazugehörigen Einheitskosten.\footcite{Adey2012} \\
Die von 


\subsubsection{Unterhaltskosten}
\label{subsub:Unterhalt}

Die Berechnung der Unterhaltskosten $K_{U}$ der Infrastruktur werden in Formel \ref{eq.2} dargestellt. Sie setzen sich zusammen aus den einmaligen Investitionskosten für den Bau der Infrastruktur $K_{Bau}$ und den jährlich anfallenden Wartungskosten $K_{Wartung,t}$ gemäss Formel \ref{eq.3}. Die Baukosten und die Einheitskosten der Wartung werden nachfolgend erläutert.


\begin{align}
K_{U} &= K_{Bau} + \sum_{t=0}^T \  K_{Wartung,t}  \label{eq.2} \\
K_{Wartung,t} &= \sum_{t=1}^2 \ EK_{Wartung,k} \cdot s_{k} \cdot b_{k}  \label{eq.3} 
\end{align}

\begin{align*}
	  k &=
      \begin{cases}
        \begin{aligned}
          & 1 \\
          & 2
        \end{aligned} &
        \begin{aligned}
         & \text{für}\ \thinspace \\
         & \text{für}\ \thinspace
        \end{aligned}
        \begin{aligned}
          & Strasse \\
          & Unterfu"hrung
        \end{aligned}
      \end{cases} \\
\end{align*}

{\setstretch{0.6}
wobei:
\begin{conditions}
 K_{U}      	     			&  Totale Unterhaltskosten für $T$ Jahre  \\
 K_{Bau}           			    &  Baukosten der Variante     \\
 K_{Wartung,t}                  &  Wartungskosten pro Jahr     \\
 EK_{Wartung,k}      	     	&  Einheitskosten pro $m^2$   \\
 s_k	    	     			&  Länge der Infrastruktur in $m$ \\
 b_k	    	     			&  Breite der Infrastruktur in $m$   \\
 k								&  Art der Infrastruktur  
\end{conditions}
}

Die entstehenden Investitionskosten für den Bau der verschiedenen Instrastrukturen, habe ich nach \cite{Baukosten2010} folgendermassen angesetzt. Die Erstellung zweier neuer Radstreifen à $1.5 \, m$ Breite kostet $850 \, CHF$ pro Laufmeter. Die Investitionskosten pro Quadratmeter für den Bau einer Velounterführung unter dem Lastfall Eisenbahn, betragen $3750 \, CHF$. Der Bau der Zufahrtsrampen zu den Velounterführungen kostet pro Rampe $130'000 \, CHF$. 
Die Wartungskosten der verschiedenen Infrastruktur Typen habe ich nach einem Gespräch mit Herr Dr. Martani wie folgt angesetzt. Für die Instandhaltung der Strasse, inklusive der Fahrradstreifen und der Fussgängerwege nehme ich an, dass jährlich $5 \thinspace \frac{CHF}{m^2}$ anfallen. Die wartungsintesivere Infrastruktur der Unterführung wird jährlich mit $30 \thinspace \frac{CHF}{m^2}$ instand gehalten.  

\newpage

\subsubsection{Reisezeitkosten}
\label{subsub:Reisezeit}


Die beim benutzen der Infrastruktur enstehenden Reisezeitkosten ($TT$) (engl. Travel time cost), spiegeln die wirtschaftlichen Auswirkungen des Zeitverlustes auf den Verkehrsteilnehmer wieder und sind somit die Kosten der Reise in Form von Zeitverlust. In Anbetracht der Tatsache, dass in dieser verlorenen Zeit gearbeitet, sowie Freizeit verbracht hätte werden können, kann dieser Zeitverlust monetär beziffert werden. Dies erfolgt mit den nachfolgen beschriebenen Einheitskosten des Zeitverlustes.  
Die Berechnung der totalen Reisezeitkosten $K_{TT}$ für $T$ Jahre erfolgt gemäss Formel \ref{eq.4} und ist eine Vereinfachung der Berechnung der \textit{Travel time cost} \cite[vlg.][643]{Adey2012}.
  

\begin{align}
K_{TT} &= \sum_{t=0}^T \Biggl[ \sum_{j=1}^2 \ DTV_{j} \cdot t_{j} \cdot EK_{TT,j} \Biggr] \label{eq.4} \\
t &= \frac{s_{k}}{v_{j}} \Biggl( 1 + 0.15 \Bigl(\frac{DTV_{j}}{C_{j}} \Bigr)^4 \Biggr) \label{eg.5} 
\end{align}

\begin{align*}
	 j &=
      \begin{cases}
        \begin{aligned}
          & 1 \\
          & 2
        \end{aligned} &
        \begin{aligned}
         & \text{für}\ \thinspace \\
         & \text{für}\ \thinspace
        \end{aligned}
        \begin{aligned}
          & für MIV \\
          & für Velo
        \end{aligned}
      \end{cases} \\
\end{align*}

{\setstretch{0.75}
wobei:
\begin{conditions}
 K_{TT}		 	 &  Totale Reisezeitkosten für $T$ Jahre  \\
 DTV_{j}    	 &  Tägliches Verkehrsaufkommen nach Fahrzeugtyp \\
 t_{j} 			 &  Zeitverlust nach Fahrzeugtyp \\
 EK_{TT,j} 		 &  Einheitskosten der verlorenen Zeit  \\
 v_{j}			 &  Gefahrene Geschwindigkeit nach Fahrzeugtyp \\
 C_{j}			 &  Kapazität der Infrastruktur nach Fahrzeugtyp  \\
 j				 &  Art des Fahrzeugs   
\end{conditions}
}

Der Zeitverlust ist abhängig vom Zustand der Infrastruktur, genauer von der Beschaffenheit des Oberflächenbelags der Fahrbahn. sowie der momentanen Auslastung. Diese Beziehungen ist schwierig zu modelieren. Jedoch kann zwischen dem Zeitverlust auf der Infrastruktur, dem Auslastungsgrad, in Abhängigkeit von der gebauten Variante und der gefahrenen Geschwindigkeit eine Beziehung modeliert werden. Diese Approximation ermöglicht es mir, die verlorene Zeit, gemäss \ref{eg.5} zu berechnen.  
Die Zeit die man benötigt eine bestimmte Strecke zurück zu legen ist abhängig von der gefahrenen Geschwindigkeit welche wiederum abhängig ist vom Zustand der Strasse sowie der Kapazität der Infrastruktur. Wird die Kapazität durch eine zu hohe Nachfrage überschritten, kann dies zu einer Überlastung des Systems führen und somit zu Verstopfungen und daraus resultierenden Verspätungen.  

Die Einheitskosten der verlorenen Zeit $EK_{TT,j}$ werden anhand des schweizerischen Medianlohn von 2018 berechnet. Der Medianlohn betrug im Jahr 2018 $6538 \, CHF$ pro Monat bei einer durchschnittlichen Arbeitszeit von 42,5 Stunden pro Woche (Quelle: BFS). Daraus ergibt sich einen durchschnittlichen Bruttostundenlohn von $38,5 \, CHF/h$ pro Person, was im Falle dieser Berechnung den Einheitskosten der verlorenen Reisezeit eines Velofahrers entspricht. Unter der Annahme eines durchschnittlichen Auslastungsgrad von 1.6 Personen pro Fahrzeuge \cite{Mikrozensus2015}, betragen die Einheitskosten der verlorenen Reisezeit pro Fahrzeug $61,6 \, CHF/h$.



\subsubsection{Betriebskosten}
\label{subsub:Betrieb}


Die Betriebskosten $K_{B}$ die für die Nutzer der Infrastruktur, für den betrachteten Zeitraum von $T$ Jahren, anfallen, werden gemäss Formel \ref{eq.6} berechnet. So werden die Betriebskosten aus der Multiplikation der Anzahl Nutzer und der zurückgelegten Distanz mit den Einheitskosten pro Fahrzeugkilometer ermittelt.
Diese sogenannten Fahrzeugsbetriebkosten sind im Rahmen dieser Optimierung, als die jährlich pro Nutzer anfallenden Wartungskosten definiert und sind somit die Kosten, die für die Instandsetzung und den Betrieb eines Fahrzeugs, bei benützung der Infrastruktur, entstehen können. Diese setzen sich zusammen aus den Kosten der Arbeitssstunden für die Instandsetzung sowie der Kosten für die Ersatz- und Verschleissteile.
 
Diese Kosten sind abhängig von der Qualität des Fahrbahnbelags, von der Ausführung der Infrastruktur und von der Kapazität der Infrastruktur. Weiter ist ein entscheidender Faktor in der Bestimmung der Fahrzeugbetriebskosten die Strassengeometry. Diese beinhaltet die Anzahl und Form der Kurven, die Steigungen sowie die Breite der Strasse und die daraus resultierende Möglichkeit des sicheren Überholens. Die Anzahl an Kreuzungsstellen und die davon abhänginge Anzahl an Brems- und Beschleunigungsmanöver haben einen direkten Einfluss auf den Verschleiss der Mechanik des Fahrzeugs. So werden im Falle des Fahrrads die Kette und die Bremsbeläge durch vermehrtes Bremsen und Anfahren verstärkt abgenutzt und im Falle des Autos erhöhen sich die Betriebskosten bei vermehrtem \textit{Stop-and-Go} Verkehr.

\begin{equation}
K_{B} =  \sum_{t=0}^T \Biggl[ \sum_{j=1}^2 \ EK_{B,j} \cdot s_{k} \cdot DTV_{j} \Biggr]  \label{eq.6} \\
\end{equation}

{\setstretch{0.75}
wobei:
\begin{conditions}
 K_{B}			   &  Totale Fahrzeugbetriebskosten \\
 EK_{B,j}	       &  Einheitskosten pro $km$ \\
 s_j	    	   &  Länge der Infrastruktur nach Fahrzeugtyp in $km$ 
\end{conditions}
}

Zur Vereinfachung der Berechnung, werden die entstehenden Betriebskosten anhand der nachfolgenden Referenzwerte ermittelt.
Die Kosten der Arbeitsstunden sowie die Kosten der Materialien werden zusammengefasst als die Einheitskosten $EK_{B}$ für den Fahrzeugbetrieb.
Diese betragen pro Auto $0.7 CHF$ pro $km$ und pro Velo $0.15 CHF$ pro $km$ (Quelle: TCS). 

\newpage


\subsubsection{Kosten durch Belastung der Umwelt}
\label{subsubsec:Environment}


Die Kosten die durch die Belastung der Umwelt $K_{E}(t)$ (\textit{Englisch}: Environment) entstehen,
setzen sich auf den Kosten der Luftverschmutzung durch die Schadstoffbelastung $K_{S}$ und der Kosten durch die Lärmbelastung $K_{L}$ zusammen und werden gemäss Formel \ref{eq.7} berechnet. 

Die Kosten durch die Schadstoffbelastung $K_{S}$, sind die Kosten die für die Allgemeinheit durch die Schäden aufgrund der Emissionen der motorisierten Fahrzeuge, entstehen können. Diese Schäden können neben gesundheitlichen Problemen für die Anwohner und Nutzer der Strasse auch die Beeinträchtigung des Pflanzenwachstums entlang der Infrastruktur, sowie die Reduktion des Wertes der Liegenschaft sein. 
Die Kosten durch die Lärmbelastung $K_{L}$, sind die Kosten die für die Allgemeinheit durch übermässigen Lärm, welcher von der Strasse verursacht wird, entstehen können. 
Die Kosten sind in diesem Falle die Störung und Beeinträchtigung der Anwohner in Form von Kopfschmerzen, Bluthochdruck, Schlafstörrungen sowie psychischer Belastung. \\
Der Lärm entsteht mehrheitlich durch Motorengeräusche sowie der Abrollgeräusche der Reifen. \cite{Adey2012}

\begin{equation}
K_{E}(t) = \sum_{t=0}^T \ \biggl(K_{S,t} + K_{L,t} \biggr)  \label{eq.7} \\
\end{equation}

{\setstretch{0.75}
wobei:
\begin{conditions}
 K_{E}		   &  Totale Umwelkosten  \\
 K_{S,t}       &  Kosten durch die Schadstoffbelastung pro Jahr \\
 K_{L,t}       &  Kosten durch die Lärmbelastung pro Jahr  
\end{conditions} 
}

Die Kosten durch die \textbf{Schadstoffbelastung} werden gemäss Formel \ref{eq.8} berechnet.

\begin{equation}
K_{S,t} = EK_{S} \cdot DTV_{MIV,t} \cdot s_{i} \biggl( 1 - \Phi_{E-Auto,t} \biggr)   \label{eq.8} \\
\end{equation}

{\setstretch{0.75}
wobei:
\begin{conditions}
 EK_{S}         	&  Einheitskosten der Schadstoffbelastug pro Fahrzeugkilometer \\
 DTV_{MIV,t}    	&  Durchschnittliche tägliche Verkehrsaufkommen des MIV im Jahr $t$  \\
 s_{MIV}          	&  Zurückgelegte Distanz in $[km]$ \\
 \Phi_{E-Auto,t}    &  Marktanteil E-Autos am $DTV_{MIV,t}$ im Jahr $t$ 
\end{conditions} 
}

Die Kosten durch die \textbf{Lärmbelastung} werden gemäss Formel \ref{eq.9} berechnet.

\begin{equation}
K_{L,t} = EK_{L} \cdot DTV_{MIV,t} \cdot s_{i}  \label{eq.9} \\
\end{equation}

{\setstretch{0.75}
wobei:
\begin{conditions}
 EK_{L}         	&  Einheitskosten der Lärmbelastung pro Fahrzeugkilometer \\
 DTV_{MIV,t}    	&  Durchschnittliche tägliche Verkehrsaufkommen im Jahr $t$  \\
 s_{MIV}          	&  Zurückgelegte Distanz in $[km]$ pro Fahrzeug 
\end{conditions} 
}

Die Schadstoffbelastung ist eine Funktion der durchschnittlich gefahrenen Geschwindigkeit sowie der Häufigkeit des \textit{Stopp and Go - Verkehrs}. So nimmt die Belastung der Luft durch Schadstoffe deutlich zu, wenn vermehrt im \textit{Stopp and Go - Verkehr} gefahren wird. 
Da diese Beziehung schwierig zu modelieren ist, wird im Rahmen dieser Untersuchung die Einheitskosten der Schadstoffbelastung $EK_{S}$ pro Fahrzeugkilometer gemäss \cite[p.38]{Ecoplan2007} mit $0.0345 \, CHF/Fahrzeugkilometer$ angesetzt. \\
Die Einheitskosten der Lärmbelastung $EK_{L}$ werden gemäss \cite[p.127]{Lärm2000} mit $0.0149 \, CHF/Fahrzeugkilometer$ angenähert. 


\subsubsection{Unfallkosten}
\label{subsubsec:Unfall}


Die totalen Unfallkosten $K_{A}$ welche von der Allgemeinheit für den betrachteten Zeitraum von $T$ Jahren getragen werden müssen, werden gemäss Formel \ref{eq.10} berechnet. \\
Die Berechnung dieser Kosten basiert auf der Kostenberechnung in \cite{Adey2012}.
In Betracht gezogen werden drei verschiedene Unfaltypen [$a$,$b$,$c$].
Unfälle mit entstandenen Sachschäden und leichtverletzten Personen werden in die Kategorie $a$ eingeteilt. Für Unfälle mit schwerverletzten Beteiligten wird die Kategorie $b$ definiert und für Unfälle mit Todesfolge die Kategorie $c$. 
Die Kategorien unterscheidenen sich in der Häufigkeit des Unfalls pro Fahrzeug \( \gamma_{j,n} \) sowie der entstehenden Einheitskosten pro Unfall $EK_{j,n}$. \\
Die pro Unfall entstehenden Einheitskosten sowie die Unfallhäufigkeiten werden nachfolgend erläutert.
Wichtig anzumerken ist, dass die ermittelten Unfallrisiken die Anzahl Unfälle eines Unfalltyps pro Personenkilometer darstellen. Das bedeuted, dass für die Berechnung der Personenkilometer der mototrisierten Fahrzeuge, der Auslastungsgrad gemäss \cite{Mikrozensus2015} in Betracht gezogen werden muss. Somit wird in der Berechnung der Unfallkosten der $DTV_{MIV}$ mit einem Faktor 1.6 multipliziert.

\begin{equation}
K_{A} = \sum_{t=0}^T \Biggl[ \sum_{j=1}^2 \Bigl( \sum_{n=a}^c \ EK_{j,n} \cdot \gamma_{j,n} \Bigr) \cdot DTV_{j} \cdot s_j \Biggr] 
\label{eq.10}
\end{equation}

\begin{align*}
      n &=
      \begin{cases}
        \begin{aligned}
          & a  \\
          & b \\
          & c
        \end{aligned} &
        \begin{aligned}
         & \text{für}\ \thinspace \\
         & \text{für}\ \thinspace \\
         & \text{für}\ \thinspace
        \end{aligned}
        \begin{aligned}
          & {Sachsch"aden\,und\,Leichtverletzte} \\
          & {Schwerverletzte} \\
          & {Todesfall}
        \end{aligned}
      \end{cases}  \\
      j &=
      \begin{cases}
        \begin{aligned}
          & 1 \\
          & 2
        \end{aligned} &
        \begin{aligned}
         & \text{für}\ \thinspace \\
         & \text{für}\ \thinspace
        \end{aligned}
        \begin{aligned}
          & Velo \\
          & Auto
        \end{aligned}
      \end{cases} \\
\end{align*}

{\setstretch{0.75}
wobei:
\begin{conditions}
 K_{A}	 		 &  Totale Unfallkosten \\
 EK_{j,n} 		 &  Einheitskosten pro Unfall nach Fahrzeugtyp \\
 \gamma_{j,n} 	 &  Unfallwahrscheinlichkeit nach Fahrzeugtyp \\
 DTV_{j}		 &  Tägliches Verkehrsaufkommen nach Fahrzeugtyp \\
 s_j	    	 &  Länge der Infrastruktur nach Fahrzeugtyp in $km$  \\
 n 				 &  Unfallart  \\
 j          	 &  Art des Fahrzeugs  
\end{conditions}
} 

Die Anzahl Unfälle pro Personenkilometer und somit die Unfallwahrscheinlichkeit \( \gamma_{j,n} \) wird mithilfe der Risiken eines Unfall mit Todesfolge gemäss \cite{Unfallrisiko2019} ermittelt. 
So betrug das Sterberisiko pro zurückgelegter Distanz von 2008 bis 2017 für einen Personenwagen; ein Todesfall pro 828 Mio. Personenkilometer (Quelle: BFS). Aus diesem Risiko ermittle ich die Unfallwahrscheinlichkeit eines Unfalls mit Todesfolge, was beudeuted, dass ich die Anzahl Unfälle mit Todesfolge pro einem Personenkilometer ermittle. 
Um die Berechnung zu vereinfachen habe ich die Personenwagen und die Motorräder unter der Bezeichnung MIV zusammengefasst. Um der höheren Unfallwahrscheinlichkeit der Motorradfahrer rechnung zu tragen, habe ich die Unfallwahrscheinlichkeit des MIV wie folgt ermittelt. \\
$\gamma_{MIV,c} = Anteil_{Motorrad} \cdot \gamma_{Motorrad,c} + Anteil_{Auto,c} \cdot \gamma_{Auto,c}$ \\
Somit wurde der prozentuale Anteil an der Gesamtmenge an Strassenmotorfahrzeugen verwendet um das Unfallrisiko des MIV's zu berechnen.
Die Anzahl Strassenmotorfahrzeuge in der Schweiz betrug 2019 6'160'262 Fahrzeuge.\footcite[Vlg.]{Bestand2019}
Davon waren 744'542 Motorräder, was einem Anteil von 12.09\% entspricht. Der Rest wird in dieser Betrachtung als Autos definiert. 

Die Berechnung der Unfallrisiken der Unfalltypen $a$ und $b$ erfolgte mithilfe des prozentualen Anteile dieser Unfalltypen an der Gesamtanzahl an Unfällen im Jahr 2019. Die Unfallrisiken der Unfalltypen $a$ und $b$ wurden somit mithilfer dieser Anteile aus dem Unfallrisiko für die Unfälle des Typs $c$ geschätzt.
Die Anzahl Unfälle der verschiedenen Typen wird der Strassenverkehrsunfall-Statistik des Bundesamt für Strassen entnommen und die Werte beziehen sich auf das Jahr 2019.
So waren 2019 0.334\% aller Unfälle, Unfälle mit Todesfolge, 6.45\% aller Unfäller waren Unfälle mit Schwerverletzten und 93.21\% der Unfälle haten Sachschaden und Leichtverletzte Personen zur Folge.\footcite{Unfall2019}
Die ausführliche Berechnung der Unfallrisiken ist im Anhang unter \ref{subsec:Unfallrisiko} dargestellt.
Die nachfolgenden Tabelle \ref{tab:t-06-01-Unfallrisiko} listet die berechneten Unfallrisiken für die verschiedenen Fahrzeuge $j$ und die verschiedenen Unfalltypen $n$ auf. 

%=============================================================================
% Thesis Template in LaTex
%
% File:  t-05-01-IsingModel.tex -- Table for the Ising
% Author(s): Juergen Hackl <hackl@ibi.baug.ethz.ch>
%            Clemens Kielhauser <kielhauser@ibi.baug.ethz.ch>
%
% Creation:  27 Jan 2014
% Time-stamp: <Tue 2013-08-13 20:14 juergen>
%
% Copyright (c) 2014 Infrastructure Management Group (IMG)
%               http://ibi.ethz.ch
%
% More information on LaTeX: http://www.latex-project.org/
%=============================================================================

\begin{table}[hbt!]
\center
%\small\renewcommand{\arraystretch}{1.2} 
%
%
\begin{tabular}{@{}p{2.6cm} p{3.3cm} p{3.3cm} p{3.3cm}@{}} \\   
\toprule
\textbf{Fahrezugtyp\,j} & \textbf{Unfalltyp\,a} & \textbf{Unfalltyp\,b} & \textbf{Unfalltyp\,c} \\
\midrule
MIV      & \(1.317\,\mathrm{10^{-6}}\)  & \(9.116\,\mathrm{10^{-8}}\)  & \(4.7243\,\mathrm{10^{-9}}\)  \\
Velo	 & \(3.818\,\mathrm{10^{-6}}\)  & \(2.643\,\mathrm{10^{-7}}\)  & \(1.37\,\mathrm{10^{-8}}\)   \\

\bottomrule

\end{tabular}
\caption[Tabelle der Unfallrisiken]{Tabelle der Unfallrisiken $\gamma_{j,n}\,\Bigl[\frac{Unf"alle_{j,n}}{Pkm_{k}}\Bigl]$}
\label{tab:t-06-01-Unfallrisiko}
\end{table}


%=============================================================================
% EOF
%

%%% Local Variables:
%%% mode: latex
%%% TeX-master: "../guidelines"
%%% End:



\newpage

Nach der ausführlichen Betrachtung verschiedenster Literatur zum Thema: \textit{Kosten die durch Strassenverkehrsunfälle entstehen} und einem Gespräch mit Herr Dr. Martani habe ich für die Berechnung der Unfallkosten im Rahmen dieser Untersuchung die folgenden Einheitskosten der verschiedenen Unfalltypen festgelegt.


\paragraph{Katergorie $a$} Die Einheitskosten pro Unfall der Kategorie $a$ setzen sich aus den entstandenen Sachschäden und den Arbeits- und Materialkösten der Reperatur der Fahrzeuge zusammen. Unter der Annahme, dass das durchschnittliche Alter eines Personenenwagens in der Schweiz 8.5 Jahre beträgt und somit schon einen deutlichen Wertverlust erlitten hat, werden die enstehenden Einheitskosten der Kategorie $a$ mit $15'000 CHF/Unfall$ angesetzt. Die Kosten für die Behandlung leichtverletzter Personen wird in dieser Betrachtung aufgrund ihrer geringen grösse vernachlässigt.

\paragraph{Kategorie $b$} Die Einheitskosten die aufgrund der Unfälle der Kategorie $b$ entstehen, werden durch die anfallenden Behandlungskosten der verunfallten Person dominiert. Die entstehenden Kosten durch den Erwerbsausfall für die Dauer der Arbeitsunfähigkeit sowie die Kosten der entstandenen Sachschäden werden in dieser Berechnung aufgrund ihrer im Vergleich zu den Behandlungskosten geringen Grösse, vernachlässigt. Die durchschnittliche Kosten die durch eine schwerverletzte Person entstehen, werden mit $110'000 CHF/Unfall$ angesetzt. Dies entspricht 3\% der Kosten einer tödlich verunfallten Person.

\paragraph{Kategorie $c$} Und zuletzt die Einheitskosten für die Folgen eines Unfalls der Kategorie $c$. Diese Kosten, für einen Unfall mit Todesfolge, basieren auf der Schätzung des Werts eines statistischen Lebens. Hierfür werden $3.7mio CHF/Unfall$ angesetzt (Quelle: ASTRA).






% ===========================================================================
% EOF
%

%%% Local Variables:
%%% mode: latex
%%% TeX-master: "../main"
%%% End:

	
	
	\subsection{Unsichere Einflussfaktoren}
	\label{subsec:Uncertain}
	%=============================================================================
% Thesis Template in LaTex
%
% File:  Unsichere Parameter -- Fallstudie
% Author(s): Jürgen Hackl <hackl@ibi.baug.ethz.ch>
%            Clemens Kielhauser <kielhauser@ibi.baug.ethz.ch>
%
% Creation:  27 Jan 2014
% Time-stamp: <Tue 2013-08-13 20:14 juergen>
%
% Copyright (c) 2014 Infrastructure Management Group (IMG)
%               http://ibi.ethz.ch
%
% More information on LaTeX: http://www.latex-project.org/
%=============================================================================

% Unterkapitel Unsichere Parameter
% ---------

%\subsubsection*{Tägliches Verkehrsaufkommen}
%\label{subsubsec:DTV}

Um einen nachhaltige Verbesserung der Verkehrsproblematik in Uster zu erreichen, muss die optimale Lösung die Situation für die nächsten vierzig Jahre verbessern. Damit ein Zeitraum von vierzig Jahren untersucht werden kann, müssen die unsicheren zukünftigen Entwicklungen der wichtigsten Einflussfaktoren berücksichtigt werden. Die nachfolgende Auflistung stellt die wichtigsten Einflüsse auf die Verkehrssituation am Bahnübergang und somit auf das DTV dar. 

{\setstretch{0.6}
\begin{itemize}
\item Bevölkerungswachstum
\item Zentrumsentwicklung und Verkehrsberuhigung
\item Ausbau der Veloparkieranlagen am Bahnhof 
\item Aufwertung der Quartiere nördlich des Bahnhofs
\item Urbane Strassenraumgestaltung im Zentrumsgebiet
\item Förderung des Langsamverkehrs gemäss STEK 
\item Ausbau des Spital und der Sportanlage Buchholz
\end{itemize}
}

Alle diese Einflussfaktoren haben gemeinsam, dass einerseits ihre zukünftige Entwicklung und andererseits das Ausmass, in dem sie den DTV in der Zukunft beeinflussen, ungewiss sind. Diese Einflüsse müssen, um die Unsicherheiten hinsichtlich der zukünftigen Mobilitätssituation am Bahnübergang zu berücksichtigen und um eine optimale Lösung für die nächsten vierzig Jahre zu finden, im Rahmen dieser Optimierung modelliert werden. 

Da der Verkehr am Bahnübergang hauptsächlich aus Ziel- und Quellverkehr des Zentrums besteht, hat das Bevölkerungswachstum den grössten Einfluss auf das DTV am Bahnübergang. Gemäss STEK, leben in Uster 35'000 Einwohner. Die zu erwartende Entwicklung der Bevölkerung ist abhängig von verschiedenen Faktoren und demnach nur anhand von Prognosen vorhersagbar. Gemäss der Prognosen im STEK wird der Wachstumstrend in Zukunft anhalten. (\cite{STEK})

Der Bau der Uster Westumfahrung sowie der Bau der Moosackerstrasse haben gemäss STEK keinen nennenswerten Einfluss auf die Menge an Autos, die den Bahnübergang Brunnenstrasse in Zukunft passieren werden. Dies folgt, wie im ersten Abschnitt des Kapitel \ref{chap:Fallstudie} erläutert, der Annahme, dass die Umleitung des Durchgangsverkehr über die Oberlandstrasse bereits nahezu vollständig vollzogen ist und dass der gemessene DTV hauptsächlich aus Quell- und Zielverkehr ins Zentrums besteht.  (\cite{STEK})

\newpage


% ===========================================================================
% EOF
%

%%% Local Variables:
%%% mode: latex
%%% TeX-master: "../main"
%%% End:



%-----------------------

\section{Generierung möglicher Lösungen}
\label{sec:Varianten}
%=============================================================================
% Thesis Template in LaTex
%
% File:  Varianten -- Fallstudie
% Author(s): Jürgen Hackl <hackl@ibi.baug.ethz.ch>
%            Clemens Kielhauser <kielhauser@ibi.baug.ethz.ch>
%
% Creation:  27 Jan 2014
% Time-stamp: <Tue 2013-08-13 20:14 juergen>
%
% Copyright (c) 2014 Infrastructure Management Group (IMG)
%               http://ibi.ethz.ch
%
% More information on LaTeX: http://www.latex-project.org/
%=============================================================================

% Unterkapitel Varianten
% ---------

Die folgenden Varianten habe ich, im Rahmen der Optimierung der Verkehrssituation am Bahnübergang Brunnenstrasse erarbeitet, um die Situation für die nächsten vierzig Jahre nachhaltig zu verbessern. Anschliessend an die Darstellung der Varianten folgt eine Übersicht der wichtigsten Eigenschaften und Parameter der Infrastrukturen, die für die Berechnung der Kosten verwendet werden. \\
Die Gesamtlänge des betrachteten Infrastrukturabschnitt beträgt für alles Variante 80 Meter.


\subsection{Variante \ 1}
\label{subsec:V1}
	
Die Variante 1 stellt den Ist-Zustand der Infrastruktur dar. In dieser Variante beträgt die durchschnittliche Wartezeit pro Nutzer, wie in Abschnitt \ref{sub:Reisezeit} erläutert, 5 Minuten.  Mit dieser Variante kann der jetztige Zustand der Infrastruktur über den betrachteten Zeitraum von vierzig Jahren untersucht werden und so die Option "keine Veränderung Durchführen" überprüft werden. 

%\begin{figure}[h!]
  %\centering
  %\subfloat[][]{\label{img:V1Ü}\includegraphics[width=.6\textwidth]{./figures/1}}
  %\hfill
 % \subfloat[][]{\label{img:V1Q}\includegraphics[width=.4\textwidth]{./figures/1_2}}
%\caption[Variante 1]{Übersicht und Querschnitt der Variante 1}
 % \label{fig:V1}
%\end{figure}

\begin{figure}[h!]
	\centering
	\includegraphics[width=0.65\textwidth]{figures/f-04-05-01-a-V1}
	\caption[Übersicht Variante 1]{Übersicht über die Variante 1}
	\label{img:V1Ü}
\end{figure}

\pagebreak

Um die Verkehrssicherheit der Langsamverkehrsteilnehmer minimal zu erhöhen, ist die Anbringung zweier Velostreifen à je 1.5 Meter Breite geplant. Die angenommene Kapazität der beiden Velostreifen zusammen beträgt 3350 Velo pro Stunde. \\
Die Anbringung der Velostreifen erfordert eine geringfügige verjüngung der Fahrbahn von 4 auf 3.5 Meter pro Fahrbahn. Trotz Verjüngung wird angenommen, dass der zweispurige Strassenabschnitt (eine pro Richtung) eine Kapazität von 2'500 Fahrzeugen pro Stunde aufweist, bei einer geplanten, zulässigen Höchstgeschwindigkeit 50 $km/h$. Unter Berücksichtigung der Situatuion vor Ort wird angenommen, dass die durchschnittlich gefahrenen Geschwindigkeit des MIV 37 $km/h$ beträgt un die der Velofahrer durchschnittlich 15 $km/h$. (\cite{Nacto2018}) (\cite{Mikrozensus2015})

\begin{figure}[h!]
	\centering
	\includegraphics[width=0.7\textwidth]{figures/f-04-05-01-b-V1}
	\caption[Querschnitt Variante 1]{Querschnitt im Schnitt A-A der Variante 1}
	\label{img:V1Q}
\end{figure}

Die Erstellung zweier neuer Radstreifen à je 1.5 Meter Breite kostet gemäss Abschnitt \ref{sub:Unterhalt} 850 CHF pro Laufmeter. Bei einer Gesamtlänge von 80 Meter ergibt sich für den Bau der Variante 1 Kosten im Bereich von 68'000 CHF. (\cite{Baukosten2010}) 

\pagebreak

\subsection{Variante: \ 2}
\label{subsec:V2}
	
Die zweite Variante beinhaltet, wie in Abbildung \ref{img:V2Ü} ersichtlich, den Bau von zwei Velounterführungen, um die lange Wartezeit am Bahnübergang aufgrund der Schliesszeit der Bahnschranken zu verkürzen. Der Zeitverlust der Velofahrer setzt sich demnach in dieser Variante, nur aus dem Zeitverlust der gemäss Abschnitt \ref{sub:Reisezeit} aufgrund des Befahren der Infrastruktur entsteht zusammen. Die für den MIV angesetzte durschnittliche Wartezeit beträgt weiterhin 5'. 

Infolge der Rückklassierung der Brunnenstrasse wird ein Tempo 30 Regime eingeführt. Die angenommene durchschnittlich gefahrene Geschwindgkeit des MIV beträgt somit 30 $km/h$ und für die Velofahrer wird angenommen, dass sie mit durchschnittlich 20 $km/h$ durch die Unterführung fahren können. (\cite{Mikrozensus2015})

%\begin{figure}[h!]
 % \centering
 % \subfloat[][]{\label{img:V2Ü}\includegraphics[width=.6\textwidth]{./figures/2}}
  %\hfill
 % \subfloat[][]{\label{img:V2Q}\includegraphics[width=.4\textwidth]{./figures/2_2}}
%\caption[Variante 2]{Übersicht und Querschnitt der Variante 2}
  %\label{fig:V2}
%\end{figure}

\begin{figure}[h!]
	\centering
	\includegraphics[width=0.65\textwidth]{figures/f-04-05-02-a-V2}
	\caption[Übersicht Variante 2]{Übersicht über die Variante 2}
	\label{img:V2Ü}
\end{figure}

Durch den Bau der beidseitig mit einer lichten breite von 1.5 Meter ausgeführten Velounterführungen, wird einerseits die Verkehrssicherheit der Velofahrer verbessert und andererseits die Kapazität der gesamten Veloinfrastruktur auf 3767 Velos pro Stunde erhöht. Die Gesamtlänge einer Unterführung beträgt in dieser Variante 55 Meter. 

Um diese Unterführung bauen zu können, ist eine weitere verjüngung der Fahrbahn auf 3 Meter erforderlich, was jedoch zu keiner Reduktion der Kapazität des MIV führen wird.  (\cite{Nacto2018})

\begin{figure}[h!]
	\centering
	\includegraphics[width=0.7\textwidth]{figures/f-04-05-02-b-V2}
	\caption[Querschnitt Variante 2]{Querschnitt im Schnitt A-A der Variante 2}
	\label{img:V2Q}
\end{figure}

Die prognostizierten Baukosten der Variante 2 werden mithilfe der in Abschnitt \ref{sub:Unterhalt} erläuterten Einheitskosten für den Bau einer Velounterführung und den vorgängig erwähnten Abmessungen der Unterführungen berechnet. Die zu erwartenden Baukosten belaufen sich auf 1.16 Mio. CHF, wobei 520'000 CHF für die vier Rampen und 640'000 CHF für den Bau der Unterführungen unter dem Lastfall Eisenbahn sowie das Anbringen der Velostreifen anfallen. (\cite{Baukosten2010}) 

\pagebreak

\subsection{Variante: \ 3}
\label{subsec:V3}

Die dritte Variante habe ich ausgehend von der Variante 2 entwickelt und ist ein Versuch die Verkehrssicherheit sowie den Fahrkomfort für die Velofahrer zu steigern. Um dies zuerreichen wird, wie in Abbildung \ref{img:V3Ü} ersichtlich ist, die Velounteführung zweispurig ausgeführt, was zur Folge hat, dass die Strasse einspurig über den Bahnübergang geführt werden muss. Diese Strassenführung erfordert die Einführung eines Ampelsystems, was die durchschnittliche Wartezeit für den MIV, bei einem Rotlichtzyklus von einer Minute, auf 7 Minuten erhöht. Für das Ampelsystem wäre eine Busbevorzugungsanalge zu prüfen. \\

Die maximale Höchstgeschwindigkeit beträgt wie in Variante 2 30 $km/h$, wobei angenommen wird, dass die Velofahrer mit durchschnittlich 25 $km/h$ durch die Unterführung fahren können. (\cite{Mikrozensus2015})

%\begin{figure}[h!]
 % \centering
  %\subfloat[][]{\label{img:V3Ü}\includegraphics[width=.6\textwidth]{./figures/3}}
  %\hfill
 % \subfloat[][]{\label{img:V3Q}\includegraphics[width=.4\textwidth]{./figures/3_2}}
%\caption[Variante 3]{Übersicht und Querschnitt der Variante 3}
 % \label{fig:V3}
%\end{figure}

\begin{figure}[h!]
	\centering
	\includegraphics[width=0.65\textwidth]{figures/f-04-05-03-a-V3}
	\caption[Übersicht Variante 3]{Übersicht über die Variante 3}
	\label{img:V3Ü}
\end{figure}

Die lichte Breite der Velounterführung beträgt 2 Meter und die Gesamtlänge einer Unterführung beläuft sich auf 65 Meter. Die zweispurige Strassenführung wird mit Fahrbahnmarkierungen verdeutlicht. Es wird angenommen, dass durch diese Ausführung die Kapazität der gesamten Veloinfrasturktur auf 4600 Velos pro Stunde erhöht wird. (\cite{Nacto2018})

Infolge der Reduktion der Strasseninfrastruktur um eine Spur nehme ich, dass sich die Kapazität auf 1'250 Fahrzeuge pro Stunde halbiert. Die Fahrspur bleibt mit 5 Meter für grosse Busse weiterhin problemlos befahrbar.

\begin{figure}[h!]
	\centering
	\includegraphics[width=0.7\textwidth]{figures/f-04-05-03-b-V3}
	\caption[Querschnitt Variante 3]{Querschnitt im Schnitt A-A der Variante 3}
	\label{img:V3Q}
\end{figure}

Die Baukosten der Variante 3 belaufen sich auf insgesamt 1.51 Mio CHF und werden wie für Variante 2 erläutert berechnet. Für den Bau der Unterführung unter dem Lastfall Eisenbahn und das Anbringen der Velostreifen sind 988'000 CHF vorgesehen und für den Bau der vier Rampen fallen 520'000 CHF an. (\cite{Baukosten2010}) 


In Tabelle \ref{tab:t-08-01-Varianten} werden die, für die Berechnung der Kosten verwendeten Eigenschaften der Varianten zusammengefasst.
%=============================================================================
% Thesis Template in LaTex
%
% File:  t-05-01-IsingModel.tex -- Table for the Ising
% Author(s): Juergen Hackl <hackl@ibi.baug.ethz.ch>
%            Clemens Kielhauser <kielhauser@ibi.baug.ethz.ch>
%
% Creation:  27 Jan 2014
% Time-stamp: <Tue 2013-08-13 20:14 juergen>
%
% Copyright (c) 2014 Infrastructure Management Group (IMG)
%               http://ibi.ethz.ch
%
% More information on LaTeX: http://www.latex-project.org/
%=============================================================================
%\small\renewcommand{\arraystretch}{1.2} 
%


\begin{table}[h!]
\scriptsize
{\setstretch{0.6}
%\renewcommand{\arraystretch}{1.4}
\flushleft
\begin{tabular}{@{}p{5cm} p{2.5cm} p{2.5cm} p{2.5cm}@{}} \\   
\toprule 		
\textbf{Eigenschaften}  				   								&\textbf{Variante\,1}  & \textbf{Variante\,2} & \textbf{Variante\,3}   \\			
\midrule 
Länge Fahrbahn ($m$)         	 		   								& 80                    & 80    			   & 80             	\\
Länge Unterführung ($m$)       	 		   								& -                     & 55    			   & 65             	\\
Velospuren:					   											&  2				    &  2				   &  4         		\\
Fahrbahnen									 		   					&  2				    &  2				   &  1         		\vspace*{0.25mm} \\
Breite eines Veloweg ($m$):				   								&  1.5				    &  1.5				   &  2         		\\
Breite einer Fahrbahn ($m$):			 		   					    &  3.5				    &  3				   &  5         		\vspace*{0.25mm} \\
Tempolimit	($\frac{km}{h}$) 		   						    		& 50				    & 30				   & 30                	\\
\underline{$\varnothing$\,Geschwindigkeit} ($\frac{km}{h}$) 			&       	            &   				   &               		 \\
\hspace*{5mm}\textbullet\, Velo            		       					& 15  					& 20    			   & 30      			\\
\hspace*{5mm}\textbullet\, MIV            		       					& 37  					& 30    			   & 30      			\vspace*{0.25mm} \\
\underline{Kapazität} ($\frac{Fahrzeug}{h}$)		        			&    				    &  				       &                  	 \\
\hspace*{5mm}\textbullet\, Velo            	       					    & 3350 					& 3767    			   & 4600      			\\
\hspace*{5mm}\textbullet\, MIV         		       					    & 2500 					& 2500    			   & 1250      			\\
\underline{Wartezeit} ($Minuten$)        		    			   		&    				    &  				       &                  	 \\
\hspace*{5mm}\textbullet\, Velo            	       					    & 5 					& 5    			       & 7      			\\
\hspace*{5mm}\textbullet\, MIV         		       					    & 5  					& 0      			   & 0      			\\
Baukosten ($CHF$)														& 68'000				& 1.16 Mio.			   & 1.51 Mio.           \\
\bottomrule

\end{tabular}
\caption{Basis Informationen der Varianten}
\label{tab:t-08-01-Varianten}
}
\end{table}


%=============================================================================
% EOF
%

%%% Local Variables:
%%% mode: latex
%%% TeX-master: "../guidelines"
%%% End:



\pagebreak

% ===========================================================================
% EOF
%

%%% Local Variables:
%%% mode: latex
%%% TeX-master: "../main"
%%% End:


%----------------------

\section{Analyse der Lösungen}
\label{sec:Analyse}

Um den Vergleich der Varianten, zur Bestimmung der optimalen Lösungen, durchführen zu könne, müssen die Kosten anhand der unter Abschnitt \ref{sec:Kostenstruktur} dargestellten Formeln, berechnet werden. Um die Kosten über einen Zeitraum von vierzig Jahren berechnen zu könne, muss der Einfluss der unter Abschnitt \ref{subsec:Uncertain} bestimmten Faktoren, auf das DTV modelliert werden und somit der DTV für die Zukunft geschätzt werden. Dies geschieht anhand der nachfolgend vorgestellten Szenarien.


	\subsection{Modellierung des DTV}
	\label{subsec:Modellierung}
	
Die Zentrumsentwicklung, die Aufwertung der Quartiere nördlich des Bahnhofs, die Massnahmen zur Verkehrsberuhigung, der Ausbau der Veloparkieranlagen am Bahnhof sowie der Ausbau des Spitals und die im Rahmen der Umsetzung des Leitziels "Uster steigt um!" getroffenen Massnahmen zur Förderung des Langsamverkehrs, haben direkt oder indirekt einen Einfluss auf den Langsamverkehr am Bahnübergang. Unter anbetracht der Tatsache, dass alle diese genannten Einflussfaktoren auf die Verkehrssituation am Bahnübergang, zu mehr Veloverkehr führen werden, fasse ich diese unter dem Stichpunkt \textit{Umsetzung des STEK} zusammen und modelliere sie, mit den in Abschnitt \ref{subsubsec:Umsetzung} vorgestellten Szenarien. Da das Bevölkerungswachstum den grössten Effekt auf das DTV haben wird, untersuche ich diese Einfluss anhand der im Abschnitt \ref{subsubsec:Bevölkerung} dargestellten Szenarien, separat. 

Die zukünftige Situation am Bahnübergang wird somit sowohl durch die Bevölkerungsentwicklung als auch durch die Umsetzun des STEK beeinflusst. Diese Einflüsse tretten gleichzeitig auf, was dazu führt, dass, um die Berechnung der Kosten der Varianten unter Berücksichtigung der unsicheren zukünftigen Entwicklung der Einflussfaktoren durchführen zukönnen, eine Kombination der entwickelten Szenarien betrachtet werden muss. Eine solche Kombination stellt ein mögliches zukünftiges Ereigniss dar. 

\begin{figure}[h!]
	\centering
	\includegraphics[width=\textwidth]{figures/f-04-06-01-Entscheidungsbaum-Szenarien}
	\caption[Szenarienübersicht]{Übersicht über die Szenarien und ihre Eintrittswahrscheinlichkeiten}
	\label{img:EntscheidunSzenarien}
\end{figure}

Eine Vorhersage über die Zukunft zumachen, ist nur mit einer gewissen Unsicherheit möglich. Um diese Unsicherheit in die Optimierung mitein beziehen zu können, bewerte ich das Eintretten dieser Szenarien mit einer Wahrscheinlichkeit $[0,1]$, der sogenannte Eintrittswahrscheinlichkeit eines Szenarios.
Die Eintrittswahrscheinlichkeit eines kombinierten Szenarios, sprich eines zukünftigen Ereigniss, berechnet sich wie in Abbildung \ref{img:EntscheidunSzenarien} dargestellt aus der Multiplikation der Eintrittswahrscheinlichkeiten der Ausgangsszenarien.
Dies stellt zugleich ein Ausschnit aus dem Entscheidungprozess zur Bestimmung der optimalen Lösung dar.

\pagebreak

	%=============================================================================
% Thesis Template in LaTex
%
% File:  04-06-Szenarien.tex -- Fallstudie/Modellierung der ungewissen Parameter
% Author(s): Jürgen Hackl <hackl@ibi.baug.ethz.ch>
%            Clemens Kielhauser <kielhauser@ibi.baug.ethz.ch>
%
% Creation:  27 Jan 2014
% Time-stamp: <Tue 2013-08-13 20:14 juergen>
%
% Copyright (c) 2014 Infrastructure Management Group (IMG)
%               http://ibi.ethz.ch
%
% More information on LaTeX: http://www.latex-project.org/
%=============================================================================

% Unterkapitel Szenarien
% ---------

\subsubsection*{Bevölkerungswachstum}
\label{subsubsec:Bevölkerung}

Wie unter Abschnitt \ref{chap:Fallstudie} erwähnt, ist das DTV mehrhteilich von der zukünftigen demographischen Entwicklung abhängig.
Um den Effekt der das Bevölkerungswachstum auf das DTV haben wird, modellieren zu können, orientiere ich mich an den Wachstumsprognossen der Stadt Uster.
Die zu erwartende Bevölkerungsentwicklung habe ich dem Kapitel 3 \textit{Stadt Uster im Porträt} des STEK Schlussbericht entnommen und wird nachfolgend kurz beschrieben. 

\begin{description}
\item[Stagnation] \hfill \\
Geschätzte Anzahl an Einwohner im Jahr 2035 beträgt 38'760, somit wächst die Bevölkerung jährlich um 188 Einwohner bzw. um 0.5\% gegenüber 2015
\item[Trend restriktiv] \hfill \\
Geschätzte Anzahl an Einwohner im Jahr 2035 beträgt 42'260, somit wächst die Bevölkerung jährlich um 363 Einwohner bzw. um 1\% gegenüber 2015
\item[Trend Prosperität] \hfill \\
Geschätzte Anzahl an Einwohner im Jahr 2035 beträgt 45'620, somit wächst die Bevölkerung jährlich um 531 Einwohner bzw. um 1.5\% gegenüber 2015
\end{description}

Anhand dieser Wachstumsprognossen und unter der Annahme eines linearen Wachstums, bilde ich die drei nachfolgend dargestellten Szenarien. Mit diesen Szenarien werde ich in einem nächsten Schritt den zukünftigen DTV für den MIV und den Langsamverkehr, sprich Veloverkehr am Bahnübergang Brunnenstrasse ermitteln.

\pagebreak

\begin{itemize}
\item Szenario: SB 1
	\begin{itemize}
	\item Grundlage: Stagnation $\Rightarrow$ jährliches Wachstum um 0.5\%
	\item Eintrittswarscheinlichkeit: 25\%
	\end{itemize}
\item Szenario: SB 2
	\begin{itemize}
	\item Grundlage: Trend restriktiv  $\Rightarrow$ jährliches Wachstum um 1 \%
	\item Eintrittswarscheinlichkeit: 50\%
	\end{itemize}
\item Szenario: SB 3
	\begin{itemize}
	\item Grundlage: Trend Prosperität  $\Rightarrow$ jährliches Wachstum um 1.5\%
	\item Eintrittswarscheinlichkeit: 25\%
	\end{itemize}
\end{itemize}

Um die Wahrscheinlichkeit des eintrettens 
Die Wahrscheinlichkeit das Szenario SB 2 eintritt und das DTV jährlich um 120 Fahrzeuge zunimmt, bewerte ich mit 50\%. Unter der Annahme, dass der restriktive Trend das minimale Wachstumsziel von 20\% wiederspiegelt, welches gemäss dem STEK mit grösster Wahrscheinlichkeit eintretten wird und das dieses Szenario dem kantonalen Prognossen entspricht, erachte ich dieses Szenario als das wahrscheinlichste.

Das es zu einem verstärktem Wachstum von 1.5\% und somit zu einer Zunahme von 180 Fahrzeugen am DTV pro Jahr kommt, erachte ich nach den Angaben des STEK als unwahrscheinlich, da ein übermässiges Bevölkerungswachstum, aufgrund der nur beschränkt vorhandenen Kapazitäten zur Erweiterung der Wohnangebots, nur bedingt möglich ist. Das es zu einer Stagnation des Bevölkerungswachstums und im Zuge dieser Modellierung zu einem Verkehrswachstum von 0.5\% und einer Zunahme von 60 Fahrzeugen am DTV pro Jahr, kommt, erachte ich in Anbetracht der Prognossen zur demographische Entwicklung im Kanton Zürich, als unwahrscheinich. Deshalb bewerte ich die Szenarien SB 1 und SB 3 mit jeweils 25\%.

Mithilfe der verschiedenen Wachstumsraten $WR_{s}$ der Szenarien und der Formel \ref{eq.11} berechne ich das $DTV_{i}$ im Jahr $t_{i}$. \\
Gemäss GIS-Browser lag der durchschnittliche Werkverkehr im Jahr 2016 bei 12'023 Fahrzeugen. Diesen Wert benutzte ich als Startwert sowie Basiswert meiner Berechnungen. 

\begin{equation*}
DTV_{i} = DTV_{2016} + \bigl( t_{i} - t_{2016} \bigr) \cdot \bigl[WR_{s}\bigr] \cdot DTV_{2016}
\label{eq.11}
\end{equation*}


Da die Anzahl Velos, die den Bahnübergang Brunnenstrasse täglich passieren, nicht von einer Verkehrsmessstelle gezählt werden, muss diese Information aus der Menge an Autos hergeleitet werden. Dies erfolgt mithilfe der Daten der Verkehrsmessstelle an der etwas südlich von Uster gelegenen Seefeldstrasse, welche Niederuster mit Riedikon verbindet. 
Der im Jahr 2019 gemessen durchschnittliche DTV lag bei 8818 Motorfahrzeugen und 913 Velofahrer. Daraus ergibt sich einen Veloanteil von 10.35\%. \footcite{MIVSeefeld}\footcite{VeloSeefeld}

\begin{align*}
\mu &= \frac{DTV_{Velo,Seefeldstrasse}}{DTV_{MIV,Seefeldstrasse}}   \\
DTV_{Velo} &= DTV_{MIV} \cdot \mu_{Velo} 
\end{align*}

In den Abbildung \ref{fig:DTV} sind die Ergebnisse meiner Modellierung des DTV am Bahnübergang Brunnenstrasse dargestellt, die ich für die Berechnung der Kosten der Varianten verwenden werde. Eine ausführliche Tabelle aller modellierten DTV-Werte findet sich unter Abschnitt \ref{subsec:DTVModellierung}

\begin{figure}[h!]
  \centering
  \subfloat[][]{\label{fig:04-06-02-DTV(MIV)}\includegraphics[width=.45\textwidth]{./figures/f-04-06-02-DTV(MIV)-SzenarienBev-Wachstum}}
  \hfill
  \subfloat[][]{\label{fig:04-06-03-DTV(VELO)}\includegraphics[width=.45\textwidth]{./figures/f-04-06-03-DTV(VELO)-SzenarienBev-wachstum}}
\caption[Verkehrsaufkommen]{Tägliches Verkehraufkommen an der Brunnenstrasse}
  \label{fig:DTV}
\end{figure}



\subsubsection*{Umsetzung des STEK}
\label{subsubsec:Umsetzung}

Die folgenden Szenarien modellieren die Effekte, den die in Abschnitt \ref{subsec:Modellierung} unter dem Stickpunkt \textit{Umsetzung des STEK} zusammengefassten Einflussfaktoren, auf den $DTV_{Velo}$ haben werden. \\

Das meines erachtens, mit grösster Wahrscheinlichkeit eintrettende Szenario, entspricht der Verkehrsprognosse des Bundes, die eine Zunahme der Verkehrsleistung des Langsamverkehrs bis 2040 gegenüber 2010, um 32\% erwartet. (\cite{Perspektive2040}) 

Um die Ober- sowie Untergrenze meiner Prognosse ermitteln zu können, orientiere ich mich ein weiteres mal an dem STEK. Mithilfe der unter Kapitel 10 \textit{Stadt Uster im Porträt} des STEK vorgestellten Wachstumsprognossen für die Bevölkerungsentwicklung sowie der in Kapitel 7 \textit{Mobilität} der STEK gemäss regionalem Richtplan erstellten Verkehrsprognosse für den Anteil der Velofahrer am Gesamtverkehr, erstelle ich zwei weitere Szenarien. 
Einerseits berücksichtige ich den Fall einer ungenüngenden Umsetzung der Leitziele und der daraus resultierenden stagnierenden Entwicklung des Veloverkehr. Andererseits den Fall einer maximalen Umsetzung aller Ziele in Verbindung mit einer Verschiebung des Innerstädtischen Modalsplit in Richtung Langsamverkehr.

\begin{description}
\item[Stagnation] \hfill \\
Prognosse gemäss STEK: $\rightarrow$ jährliches Wachstum: 0.54 \% 
\item[Verkehrsperspektiven 2040] \hfill \\
Prognostizierte Zunahme der Verkehrsleistung: 32\% $\rightarrow$ jährliches Wachstum: 1.3 
\item[Umsetzung maximal] \hfill \\
Prognosse gemäss STEK und regionalem Richtplan: $\rightarrow$ jährliches Wachstum: 2 \% 
\end{description}

Eine Stagnation erachte ich, unter berücksichtigung der Entwicklung des Langsamverkehr über die letzten zehn Jahre, als unwahrscheinlich und bewerte das Eintretten dieser Prognosse demzufolge mit 5\%.
Das es zu einem Wachstum gemäss der Prognosse des Bundes kommen wird, erachte ich nach der Konsultation weitere Verkehrsprognossen, als das Szenario, welches mit grössten Sicherheit eintretten wird. Daher bewerte ich diese Szenario mit einer Eintrittswahrscheinlichkeit von 57.5\%. 
Das alle Ziele maximal erfüllt werden und eine Verschiebung des Innerstädtischen Modal-Split statt findet, erachte ich mit 32.5\% als deutlich plausibler als die Stagnation, jedoch als unwahrscheinlichr als die Prognosse des Bundes. 

\begin{itemize}
\item Szenario: SU 1
	\begin{itemize}
	\item Grundlage: Stagnation 
	\item Eintrittswarscheinlichkeit: 5\%
	\end{itemize}
\item Szenario: SB 2
	\begin{itemize}
	\item Grundlage: Verkehrsperspektiven 2040
	\item Eintrittswarscheinlichkeit: 57.5\%
	\end{itemize}
\item Szenario: SB 3
	\begin{itemize}
	\item Grundlage: Umsetzung maximal
	\item Eintrittswarscheinlichkeit: 32.5\%
	\end{itemize}
\end{itemize}

Anhand der zu Beginn dieses Abschnitts definierten Wachstumsraten, sowie ausgehend von den Messwerte des täglichen Veloverkehrs im Jahr 2016, habe ich die Anzahl Velos, die in jedem Szenario zusätzlich pro Tag auf der Infrastruktur unterwegs sein werden, ermittelt. 
Die Anzahl Velos die im Jahr 2016 täglich den Bahnübergang nutzten, lag, gemäss Abschnitt \ref{subsubsec:Bevölkerung}, bei 1245.

Das Szenario SU 1 führt somit zu einer Erhöhung des täglichen Veloverkehr um 7 Velos, das Szenario SU 2 zu einer Zunahme von 16 Velos und das Szenario SB 3 zu einer Erhöhung des $DTV_{Velo}$ um 25 Velos. 
Mit diesen Angaben berechne ich die Anzahl Velos, die je nach Szenario, zusätzlich zu den in Abschnitt \ref{subsubsec:Bevölkerung} berechneten DTV-Werten, auf der Infrastruktur unterwegs sein werden.

\pagebreak

% ===========================================================================
% EOF
%

%%% Local Variables:
%%% mode: latex
%%% TeX-master: "../main"
%%% End:



	\subsection{Berechnung der Kosten der Varianten}
	\label{subsec:Kostenberechnung}
	%=============================================================================
% Thesis Template in LaTex
%
% File:  04-07-Kosten -- Analyse der Lösungen - Fallstudie
% Author(s): Jürgen Hackl <hackl@ibi.baug.ethz.ch>
%            Clemens Kielhauser <kielhauser@ibi.baug.ethz.ch>
%
% Creation:  27 Jan 2014
% Time-stamp: <Tue 2013-08-13 20:14 juergen>
%
% Copyright (c) 2014 Infrastructure Management Group (IMG)
%               http://ibi.ethz.ch
%
% More information on LaTeX: http://www.latex-project.org/
%=============================================================================

% Unterkapitel Kosten
% ---------

Die Abbildung \ref{img:Kostenberechnung} zeigt die Kostenberechnung der ersten 8 Jahre am Beispiel der Variante 1 im Szenario SB1/SU1. 
Die Ermittlung der Gesamtkosten einer Variante in einem Szenario erfolgt anhand der Berechnung der Zielfunktion, über den betrachteten Zeitraum von vierzig Jahren. Übersichtshalber zeige ich die Formel \ref{eq.x1} der Zielfunktion erneut.

\begin{equation}
Min. \thinspace K = Min. \thinspace [K_{W} + K_{B} + K_{TT} + K_{E} + K_{A}]
\label{eq.x1}
\end{equation} 

\begin{figure}[h!]
	\centering
	\includegraphics[width=\textwidth]{figures/f-04-06-04-Kostenberechnung}
	\caption[Kostenberechnung]{Beispiel der Kostenberechnung}
	\label{img:Kostenberechnung}
\end{figure}

Die jährlichen DTV Werte sind gemäss dem Abschnitt \ref{subsec:Modellierung} berechnet und die Reisezeitverluste $t$ gemäss Abschnitt \ref{sub:Reisezeit}. Nach \ref{sub:Unterhalt} werden die Wartungskosten berechnet. Die Berechnung der Wartungskosten erfolgt mit den in Abschnitt \ref{sec:Varianten} beschriebenen Abmessungen der Variante. Die im Jahr 2020 anfallenden Interventionskosten belaufen sich gemäss Abschnitt \ref{sec:Varianten} auf 68'000 CHF.

Die Betriebskosten der Nutzer werden gemäss Abschnitt \ref{sub:Betrieb} berechnet, dies erfolgt duch die Multiplikation des DTV mit der Länge der jeweiligen Fahrbahn und den Einheitskosten des Fahrzeugbetriebs. Um die Kosten eines Jahres zu ermitteln, wird der berechnet Wert mit 365 multipliziert. \\
Die weiteren Kosten der Nutzer sind die Reisezeitkosten, die nach Abschnitt \ref{sub:Reisezeit} berechnet werden. Hierfür wird der im oberen Bereich der Tabelle dargestellte Zeitverlust pro Nutzer, berechnet aus dem Zeitverlust, der durch das Befahren der Infrastruktur und durch die durchschnittliche Wartezeit aufgrund der Bahnschranke gemäss Abschnitt \ref{sub:Reisezeit} entsteht, mit dem DTV und den Einheitskosten des Zeitverlust, multipliziert.

Die Berechnung der Umweltkosten erfolgt nach Abschnitt \ref{subsec:Environment} durch die Multiplikation des DTV mit den Einheitskosten und der Länge der Fahrbahn. Im Fall der Schadstoffbelastungskosten wird, vom $DTV_{MIV}$ der jährliche E-Auto Anteil abgezogen.

Die Berechnung der Unfallkosten erfolgt pro Unfallkategorie und Fahrzeugtyp durch die Multiplikation des DTV mit der Länger der Fahrbahn den Unfallrisiken gemäss Abschnitt \ref{subsubsec:Unfallrisiko}. Daraus ergibt sich die jeweilige Unfallanzahl nach Unfallart. Diese werden mit den Einheitskosten der jeweiligen Unfallart multipliziert und die berechneten Kosten, um die totalen Unfallkosten zu ermitteln, aufsummiert. Im Anhang unter Abschnitt \ref{sec:Anhangkostenberechnug} sind die Tabellen der berechneten Kosten für die Varianten 1, 2 und 3 mit den Grundannahmen der Kostenstruktur aufgeführt.



% ===========================================================================
% EOF
%

%%% Local Variables:
%%% mode: latex
%%% TeX-master: "../main"
%%% End:


%---------------------

\section{Bewertung der Lösungen}
\label{sec:Bewertung}

Da die berechneten Kosten einer Variante in einem Szenario vom jeweiligen Szenario abhängt, muss, um eine von den Szenarien unabhängige Bewertung der Varianten vornehmen zu können, die berechneten Kosten mit den Eintrittswahrscheinlichkeit des jeweiligen Szenarios gewichtete werden. Diese wahrscheinlichkeitsgewichteten Kosten die aufgrund der Ausführung einer Variante in einem der Szenarien entstehen, nenne ich im Rahmen dieser Untersuchung; Risiko. \\
Dies geschieht, um die Unsicherheiten, welche bei der Vorhersage der Wahrscheinlichkeit des Eintrettens der Szenarien entstehen, in die Bewertung der Varianten miteinfliessen zulassen. 

Als letzter Schritt der Bewertung wird mit Hilfe der Sensitivitätsanalysen der rechten Seite der Zielfunktion, die Robustheit der gefundenen optimalen Lösung überprüft sowie die Abhängigkeit von den getroffenen Annahmen aufgezeigt.


	\subsection{Berechnung der Risiken der Varianten}
	%=============================================================================
% Thesis Template in LaTex
%
% File: 04-08-Risiken - Bewertung der Lösung -- Fallstudie
% Author(s): Jürgen Hackl <hackl@ibi.baug.ethz.ch>
%            Clemens Kielhauser <kielhauser@ibi.baug.ethz.ch>
%
% Creation:  27 Jan 2014
% Time-stamp: <Tue 2013-08-13 20:14 juergen>
%
% Copyright (c) 2014 Infrastructure Management Group (IMG)
%               http://ibi.ethz.ch
%
% More information on LaTeX: http://www.latex-project.org/
%=============================================================================

% Unterkapitel Risikenberechnung 
% ---------
\label{subsec:BerechnungRisiken}

Dieser Abschnitt beschreibt die berechnung der Risiken mit den Entscheidungsbäumen.





% ===========================================================================
% EOF
%

%%% Local Variables:
%%% mode: latex
%%% TeX-master: "../main"
%%% End:




	\subsection{Sensitivitätsanalyse}
	%=============================================================================
% Thesis Template in LaTex
%
% File: Sensitivitätsanalyse -- Fallstudie
% Author(s): Jürgen Hackl <hackl@ibi.baug.ethz.ch>
%            Clemens Kielhauser <kielhauser@ibi.baug.ethz.ch>
%
% Creation:  27 Jan 2014
% Time-stamp: <Tue 2013-08-13 20:14 juergen>
%
% Copyright (c) 2014 Infrastructure Management Group (IMG)
%               http://ibi.ethz.ch
%
% More information on LaTeX: http://www.latex-project.org/
%=============================================================================

% Unterkapitel Sensitivitätsanalyse
% ---------
\label{subsec:Sensitivitätsanalyse}

Wie im Abschnitt \ref{sec:Sensitivität} erläutert, wird mithilfe der Sensitivitätsanalysen, die in einem ersten Schritt ermittelte Lösung weiter untersucht, um die Belastbarkeit der Ergebnisse, in einer allfälligen Diskussion der Variante, zu stärken. Da auf die Phase der Erarbeitung einer Infrastruktur Intervention eine politische Diskussionen folgt, die den weiteren Verlauf des Projekts massgeblich bestimmt, ist eine vertiefte Untersuchung der Ergebnisse unerlässlich. Dies dient der Erabeitung einer Argumentationsbasis für die Verteidung der entwickelten optimalen Variante.

In diesem Schritt werden die Parameter der Risikoberechung verändert, die im Rahmen der politischen Auseinandersetzung das grösste Diskussionspotential bieten.
Die nachfolgenden Abschnitte umschreiben die durchgeführten Sensitivitätsanalysen und stellen die jeweiligen veränderten Parameter dar.

\paragraph{Zustand 0} untersucht der Grundzustand der Parameter und ist der Referenzwert der Sensitivitätsanalyse.

\paragraph{Zustand 1} Da der im Zustand 0 angenommene E-Auto Anteil eine progressive Prognosse ist, erachte ich eine differenzierte Untersuchung für angebracht. So simuliert der Zustand 1 eine konservative Prognosse des E-Auto Anteils im Jahr 2050 von 50\%. Wie in Abschnitt \ref{subsubsec:Marktanteil}, habe ich unter der Anahme eines linearen Wachstum, den in Abbildung \ref{img:Marktanteil-Z1} dargestellten jährlichen E-Auto Anteil \( \Phi_{E-Auto} \) berechnet. (\cite{Bestand2019})


\begin{figure}[h!]
	\centering
	\includegraphics[width=.6\textwidth]{figures/f-04-09-01-Marktanteil-E-Autos-Zustand-1}
	\caption[Marktanteil E-Autos - Zustand 1]{Marktanteils der E-Autos am Strassenfahrzeugbestand - Zustand 1}
	\label{img:Marktanteil-Z1}
\end{figure}


\paragraph{Zustand 2} Da ich die Unfallwahrscheinlichkeiten anhand der gesamtschweizerischen Unfalldaten und Leistungen des Personenverkehrs berechne, erachte ich eine vertiefte Untersuchung dieser Parameter als notwenig. Um die effektive Gefahrenlage am Bahnübergang zu simulieren, verändere ich im Zustand 2 die Unfallwahrscheinlichkeiten. 

Um in der Variante 2 zu berücksichtigen, dass die erhöhte Durchfahrtsgeschwindigkeit in Verbindung mit der geringen Breite der Unterführung ein gewisses Sicherheitsrisiko darstellt, wird die Unfallwahrscheinlichkeit der Velofahrer um 50\% erhöht. \\
Im gleichen Schritt habe ich, unter der Annahme, dass die Einführung eines Ampelsystems und die dementsprechende einspurige Verkehrsführung, die Anzahl Unfälle auf dem Bahnübergang mit MIV-Beteiligung merklich reduziert sollte, die Unfallwahrscheinlichkeit des MIV wird in der Variante 3 demnach um 50\% gesenkt. 

Die ermittelten Unfallrisiken für den Zustand 2 sind in der nachfolgenden Tabelle zusammengefasst. 

\input{./tables/t-09-01-Unfallrisiko-Zustand-2}

\paragraph{Zustand 3} Die Eintrittswahrscheinlichkeiten im Zustand 0 habe ich anhand von Prognossen gesetzt. Um zu untersuchen welchen Effekt die Veränderung der Eintrittswahrscheinlichkeiten der Szenarien, auf die optimale Lösung hat, verändere ich im Zustand 3 die angenommen Eintrittswahrscheinlichkeiten. Die Eintrittswahrscheinlichkeit sind, wie in Abbildung \ref{img:EntscheidunSzenarien-Z3} ersichtlich, so verteilt, dass den Szenarien mit den grössten Wachstumsprognossen mehr Gewicht gegeben wird. 

\begin{figure}[h!]
	\centering
	\includegraphics[width=\textwidth]{figures/f-04-09-02-Entscheidungsbaum-Szenarien-Zustand-3}
	\caption[Szenarienübersicht - Zustand 3]{Übersicht über die Szenarien und ihre Eintrittswahrscheinlichkeiten - Zustand 3}
	\label{img:EntscheidunSzenarien-Z3}
\end{figure}

\paragraph{Zustand 4} Um den Faktor Reisezeit genauer zu untersuchen und zu analysieren welchen Effekt die Veränderung der Reisezeit auf das Ergebniss hat, habe ich die prognostizierte durchschnittliche Wartezeit der Variante 3 wird von 7' auf 5' gesenkt, unter der Annahme, dass das eingeführte Ampelsystem zu keiner zusätzlichen Wartezeit führen wird.



% ===========================================================================
% EOF
%

%%% Local Variables:
%%% mode: latex
%%% TeX-master: "../main"
%%% End:


% ===========================================================================
% EOF
%

%%% Local Variables:
%%% mode: latex
%%% TeX-master: "../main"
%%% End:


% Chapter 5
% ---------
%=============================================================================
% Thesis Template in LaTex
%
% File:  05-Resultate.tex -- Resultate
% Author(s): Cyrano Golliez <golliezc@student.ethz.ch>
%
% Creation:  27 Jan 2014
% Time-stamp: <Tue 2013-08-13 20:14 juergen>
%
% Copyright (c) 2014 Infrastructure Management Group (IMG)
%               http://ibi.ethz.ch
%
% More information on LaTeX: http://www.latex-project.org/
%=============================================================================

\chapter{Resultate}
\label{chap:Resultate}

Um die Verkehrssituation am Bahnübergang Brunnenstrasse nachhaltig zu verbessern, bedarfs es einer optimalen Variante, die durch den Vergleich der gemäss Abschnitt > berechneten Risiken der Varianten bestimmt wird. Der Vergleich der Risiken der Varianten erfolgt für die in Abschnitt < definierten Zustände um die gefunden optimale Lösung zu bestärken oder allenfalls zu verwerfen.
Nachfolgend dargestellt sind die Resulatet der Risikoberechnung sowie der Risikovergleich und die für die verschiedenen Zustände bestimmte optimale Variante. 

\begin{figure}[h!]
  \centering
  \subfloat[][]{\label{img:RisikoVar-Z0)}\includegraphics[width=.45\textwidth]{./figures/f-05-02-RisikenVarianteZ0}}
  \hfill
  \subfloat[][]{\label{img:VariantenwahlZ0)}\includegraphics[width=.45\textwidth]{./figures/f-05-01-Variantenwahl}}
\caption[Verkehrsaufkommen]{Risikovergleich und Entscheidungsprozess im Zustand 0}
  \label{fig:Z4}
\end{figure}

\paragraph{Zustand 0}

Wie in der Abbildung \ref{img:RisikovergleichVarianten} ersichtlich, beträgt das Risiko der Variante 1 im Zustand 0 721'206'455 CHF, das Risiko der Variante 2 683'460'615 CHF und das Risiko der Variante 3 941'116'798 CHF. Das Risiko der Variante 2 ist somit um 37'745'840 CHF geringer als das Risiko der Variante 1 und um 257'656'183 CHF kleiner als das Risiko der Variante 3. Die Differenz der Risiken der Varianten 1 und 3 beträgt 219'910'343 CHF. \\
Die Variante mit dem geringsten Risiko im Zustand 0 ist demnach Variante 2 und ist dementsprechen die optimale Variante, wie in Abbildung \ref{img:VariantenwahlZ0} dargestellt.

\paragraph{Zustand 1 bis 3}

Die nachfolgende Abbildung stellt die Risiken der Varianten in den Zuständen 1 bis 3 dar. Es ist klar ersichtlich, dass sich in den Zuständen 1 bis 3 keine Veränderung der optimalen Lösung ergibt und weiterhin die Variante 2 die optimale Lösung darstellt. Aus diesem Grund verzichte ich auf die ausführliche Darstellung der Zustände 1 bis 3 und verweise auf den Abschnitt \ref{chap:Diskussion} in dem ich die Zustände vergleiche und diskutiere.

\begin{figure}[h!]
	\centering
	\includegraphics[width=\textwidth]{figures/f-05-00-01-RisikenVergleichZustände1-3}
	\caption[Risikovergleich der Zustände 1 bis 4]{Risiken der Varianten - Vergleich der Zustände 1 bis 3}
	\label{img:RisikovergleichVarianten}
\end{figure}

\paragraph{Zustand 4}

Im Zustand 4 beträgt das Risiko der Variante 1 wie im Zustand 0 721'206'455 CHF. Das Risiko der Variante 2 beträgt 683'460'616 CHF und das Risiko der Variante 3 ist mit 683'422'520 CHF deutlich niedriger als in Zustand 0. In Abbildung \ref{img:Z4-V2/V3)} wird der Vergleich der Varianten 2 und 3 gezeigt, wo man erkennen kann, dass die Differenz der berechneten Risiken 38'095 CHF beträgt.

Aufgrund der berechneten Risiken im Zustand 4 und unter den getroffenen Annahmen, ist die Variante 3 die optimale Lösung zur bedarfsgerechten Verbesserung der Verkehrssituation am Bahnübergang. Eine Diskussion dieser Lösung folgt im Abschnitt \ref{chap:Diskussion}.

\begin{figure}[h!]
  \centering
  \subfloat[][]{\label{img:RisikenZ4)}\includegraphics[width=.45\textwidth]{./figures/f-05-05-RisikenVarianteZ4}}
  \hfill
  \subfloat[][]{\label{img:Z4-V2/V3)}\includegraphics[width=.45\textwidth]{./figures/f-05-06-RisikenVariante-2und3-Z4}}
\caption[Verkehrsaufkommen]{Tägliches Verkehraufkommen Brunnenstrasse}
  \label{fig:Z4}
\end{figure}



% ===========================================================================
% EOF
%

%%% Local Variables:
%%% mode: latex
%%% TeX-master: "../main"
%%% End:


% Chapter 6
% ---------
%=============================================================================
% Thesis Template in LaTex
%
% File:  06-Diskussion.tex -- Diskussion
% Author(s): Cyrano Golliez <golliezc@student.ethz.ch>
%
% Creation:  27 Jan 2014
% Time-stamp: <Tue 2013-08-13 20:14 juergen>
%
% Copyright (c) 2014 Infrastructure Management Group (IMG)
%               http://ibi.ethz.ch
%
% More information on LaTeX: http://www.latex-project.org/
%=============================================================================

\chapter{Diskussion}
\label{chap:Diskussion}

In diesem Kapitel wird die in Abschnitt \ref{chap:Resultate} als optimal erachtete Lösung untersucht. Einerseits werden die Zustände 1 bis 4 mit dem Grundzustand 0 verglichen und andererseits die einzelne Kostenstrukturen in den verschiedenen Zuständen untersucht. Dies geschieht, um die als optimal erachtete Variante auf ihre Belastbarkeit in einer allfälligen Diskussion zuprüfen.

\paragraph{Zustand 0}

Gemäss dem Risikovergleich in Abschnitt \ref{chap:Resultate} ist im Zustand 0 die bestmögliche Variante für die Zukunft von Uster die Variante 2.  
Bei näherer Betrachtung der in Abbildung \ref{img:KostenZ0} dargestellten Reisezeitkosten und Unterhaltskosten der Varianten wird verdeutlicht, dass die Kosten die in der Variante 1 aufgrund der verlängerten Wartezeit entstehen, die Mehrkosten infolge Bau und Wartung der Variante 2, bei einer Betrachtung der Gesamtkosten über einen Zeitraum von 40 Jahren, um ein vielfaches übersteigen. Die Mehrkosten die bei der Ausführung der Variante 2 infolge höherer Bau- und Wartungskosten enstehen, betragen 1'238'600 CHF wohingegen die Mehrkosten die infolge der Variante 1, aufgrund der höheren Reisezeitkosten entstehen 38'984'439 CHF betragen. 

Durch diesen Vergleich wird verdeutlicht, dass sich die Variante 1, welche die Option "nichts zu verändern" darstellt, über den betrachteten Zeitraum von 40 Jahren nicht lohnenen wird. Die Mehrkosten die den Nutzern des Bahnübergangs und somit auch indirekt der Stadt Uster infolge der verlängerten Wartezeit entstehen, übersteigen die geforderten Inverstitionskosten für den Bau der Velounterführung um ein vielfaches. Im Fall der Variante 3 trifft dies nicht zu, weshalb sich ein solches Inverstment unter den Annahmen des Grundzustandes nicht lohnt.

\begin{figure}[h!]
  \centering
  \subfloat[][]{\label{img:}\includegraphics[width=.45\textwidth]{./figures/06-01-Unterhaltskosten-Z0}}
  \hfill
  \subfloat[][]{\label{img:}\includegraphics[width=.45\textwidth]{./figures/06-02-Reisezeitkosten-Z0}}
\caption[Verkehrsaufkommen]{Tägliches Verkehraufkommen Brunnenstrasse}
  \label{img:KostenZ0}
\end{figure}

\begin{IMleftrightskip}
Die Kosten wurden analog der in Abschnitt \ref{subsec:BerechnungRisiken} erläuterten Risikoberechnung ermittelt. Dies gilt für alle weiteren erwähnten Kostenstrukturen.
\end{IMleftrightskip}

Demnach lohnt sich der Bau der Variante 2 für die Stadt Uster bei einer Betrachtung der Gesamtkosten über 40 Jahre und das entstehende Risiko rechtfertigt die höheren Investitionskosten zu Beginn des untersuchten Zeitraum. 

\paragraph{Zustand 1}

Gemäss Abschnitt \ref{chap:Resultate} ist auch in Zustand 1 die Variante 2 die optimale Lösung. Somit hat die Veränderung des E-Auto Anteiles keinen Einfluss auf die Wahl der optimalen Variante. \\
Infolge des Vergleichs der Zustände 0 und 1 ist ersichtlich, dass das Risiko der Variante 1 um 0.023\%, das Risiko der Varianten 2 um 0.024\% und das Risiko der Variante 3 um 0.017\% gegenüber dem jeweiligen Risiko im Zustand 0 ansteigt. Diese Abweichungen wäre, um den Effekt der die Veränderung des E-Auto Anteiles auf optimale Lösung hat, im Rahmen einer Hauptstudie zu untersuchen.

Bei der näheren Betrachtung der Umweltkosten wird ersichtlich, dass die Umweltkosten für alle 3 Varianten jeweils um einen Betrag von 163'555 CHF ansteigen. 
Ein konservative Prognose des E-Auto Anteil hat somit keinen Einfluss auf die Wahl der optimalen Variante. Demnach müsste die Variante 2 auch in einer Zukunft in der der Verbrennungsmotor weiterhin eine massgebende Rolle spielen wird, die als optimal zu erachtetende Variante sein, um die Verkehrssituation am Bahnübergang nachhaltig zu verbessern.


\paragraph{Zustand 2} 

Wie im Abschnitt \ref{chap:Resultate} dargestellt, hat die Veränderung der Unfallwahrscheinlichkeit keinen Einfluss auf die Wahl der optimalen Variante. 
Bei näherer Betrachtung der Unfallkosten, sieht man, dass sich die Veränderung der Unfallwahrscheinlichkeit deutlich auf die anfallenden Unfallkosten auswirkt. 
Die Abbildung \ref{img:UnfallVer.Z0-2} zeigt den Unfallkostenvergleich der Zustände 0 und 2. Die Unfallkosten aller Varianten im Zustand 0 betragen 451'469 CHF. Im Zustand 2 betragen die Unfallkosten der Variante 2 547'384 CHF und die Unfallkosten der Variante 3 321'649 CHF. Für die Variante 2 entspricht das einer Veränderung von 21.25\% und für die Variante 3 einer Abnahme von 28.75\%. 

\begin{figure}[h!]
	\centering
	\includegraphics[width=.45\textwidth]{figures/06-03-Unfallkostenvergleich-Z0-Z2}
	\caption[Unfallkostenvergleich: Zustand 0 und 2]{Unfallkostenvergleich: Zustand 0 und 2}
	\label{img:UnfallVer.Z0-2}
\end{figure}

Betrachtet nun hingegen die Veränderung der Risiken der Variante von Zustand 0 zu Zustand 2, so beträgt die Veränderung für die Variante 1 +0.01\% und für die Variante 3 -0.01\%. 
Somit ist die Variante 2 auch mit erhöhter Unfallgefahr die optimale Variante für die zukünftige Situation am Bahnübergang Brunnenstrasse. 
 

\paragraph{Zustand 3} 

Die Veränderung der Eintrittswahrscheinlichkeit, wie in Abschnitt \ref{subsec:Sensitivitätsanalyse} dargestellt, hat keinen Einfluss auf die Wahl der optimalen Variante. Jedoch ist die folgende Darstellung \ref{img:SzeVer-Z0} der Risiken der einzelnen Szenarien, insofern interessant, dass es den Effekt verdeutlicht der die Wahl der Eintrittswahrscheinlichkeit auf die Verteilung der Risiken hat. So kann mit der Gewichtung der möglichen zukünftigen Ereignisse, sprich den Prognosen der möglichen Zukunft, die berechneten Risiken beeinflusst werden. 

\begin{figure}[h!]
	\centering
	\includegraphics[width=.45\textwidth]{figures/f-06-04-RisikenSzenarienZ0}
	\caption[Szenarienvergleich im Zustand 0]{Vergleich der Risiken der Szenarien im Zustand 0}
	\label{img:SzeVer-Z0}
\end{figure} 

Im Zustand 3 wird der Schwerpunkt der Gewichtung der Szenarien so gelegt, dass diejenigen Szenarien mehr gewicht erhalten, welche die höchsten Wachstumsprognosen voraussagen und demnach das grösste Verkehrsaufkommen simulieren. Der Effekt einer solchen Verschiebung auf die Szenarien wird in der nachfolgenden Abbildung \ref{img:SzeVer-Z3} verdeutlicht. 

\begin{figure}[h!]
	\centering
	\includegraphics[width=.45\textwidth]{figures/f-06-05-RisikenSzenarienZ3}
	\caption[Szenarienvergleich im Zustand 3]{Vergleich der Risiken der Szenarien im Zustand 3}
	\label{img:SzeVer-Z3}
\end{figure} 

Auf die Wahl der optimalen Variante hat diese Veränderung keinen Einfluss, da die Risiken der Varianten jeweils um 5.5\% im Vergleich zum Zustand 0 steigen. Erst bei der Betrachtung der Nachkommastellen, lässt sich eine gewisse Unterscheidung feststellen. So beträgt die Veränderung des Risiko der Variante 1 vom Zustand 0 zum Zustand 3 5.530\%, die Veränderung des Risiko der Variante 2 beträgt 5.518\% und die Veränderung des Risikos der Variante 3 5.525\%. Aus diesen geringfügigen Abweichung einen Rückschluss auf die im Rahmen des Zustand 3 veränderten Eigenschaften der Risikoberechnung zu machen, war im Rahmen dieser Projektarbeit nicht möglich und wäre im Rahmen einer Hauptstudie weiter zu untersuchen.


\paragraph{Zustand 4} 

in die anderere Richtung die Argumentation gegen die V2 -> fals für alle 3 die wartezeit steigt, weil noch mehr züge kommen, so dass amepelsystem nicht zu extra wartzeit führt, dann aber nur dann wäre die Variante 3 der Variante 2 überlegen... um wie viel...

%Die Kapazität für den ÖV ist in keiner Variante beeinträchtigt. Einen Ausbau der Kapazität ist nicht geplannt. Neubauten oder umstrukturierungen verschiedener Bushaltestellen ist im Rahmen der genaueren Bauausführung zu prüfen. \\

%Hier werden die Resultate diskutiert. 

%Der geplanten Stadterschliessungen West und Süd-Ost sind für die Förderung des Langsamverkehrs in der Stadt Uster von zentraler Bedeutung. So kann eine Entlastung des Zentrums und der Nord-Süd Achse entlang der Bahnhofstrasse nur realisiert werden wenn der MIV entlang der Westumfahrung um das Stadtzentrum herum geführt wird. 

%Dies ermöglicht gemäss STEK eine komfortable, durchgehende und direkte Verindung mit allenfals zu prüfender Vortrittsberechtigung. Die Infrastruktur dieser Radwege sollte ausschliesslich für diese Art von Verkehr zugelassen sein und gemäss STEK Kap.7 Tab. 1 mind. 4.8m breit sein. Um eine kreuzungsfreie Durchfahrt zu gewährleisten ist eine Ausführung in Anlehnung an den Aufbau einer Autobahn zu prüfen.  

% ===========================================================================
% EOF
%

%%% Local Variables:
%%% mode: latex
%%% TeX-master: "../main"
%%% End:


% Chapter 7
% ---------
%=============================================================================
% Thesis Template in LaTex
%
% File:  07-Schlussfolgerung.tex -- Schlussfolgerung und Ausblick
% Author(s): Cyrano Golliez <golliezc@student.ethz.ch>
%
% Creation:  27 Jan 2014
% Time-stamp: <Tue 2013-08-13 20:14 juergen>
%
% Copyright (c) 2014 Infrastructure Management Group (IMG)
%               http://ibi.ethz.ch
%
% More information on LaTeX: http://www.latex-project.org/
%=============================================================================

\chapter{Schlussfolgerung und Ausblick}
\label{chap:Schlussfolgerung}


Basierend auf den in Kapitel \ref{chap:Resultate} dargestellten Resultaten und der in Kapitel \ref{chap:Diskussion} durchgeführten Diskussion, ist die Variante 2 die optimale Verbesserung der Verkehrssituation in Uster. 

Unter den in Abschnitt \ref{sec:Kostenstruktur} getroffenen Annahmen und der unter Abschnitt \ref{subsec:Modellierung} modellierten Veränderungen des Mobilitätsbedarfes, ergibt sich, dass sich die Mehrkosten durch den Bau der Variante 2 gegenüber Variante 1 bei einer Betrachtung der Gesamtkosten über den Zeitraum von 40 Jahren, lohnen würden. Insbesondere unter der in Kapitel \ref{chap:Diskussion} durchgeführten, Betrachtung der Reisezeitkosten ergibt sich eine eindeutige Lösung zur Optimierung des Bahnübergangs Brunnenstrasse. Die Variante 2 ist nicht nur in ökologischer Hinsicht der Variante 1 überlegen, sondern auch im ökonomischen Sinne.

Die höheren Investitions- und Wartungskosten der Variante 2 rechnen sich demnach im Vergleich zur Variante 1, bei einer Betrachtung der Situation über 40 Jahre. Die Variante 3, die weitaus höhere Investitionskosten verursachen würde, lohnt sich, nach der in Kapitel \ref{chap:Resultate} und Kapitel \ref{chap:Diskussion} durchgeführten Diskussion der Risiken, aufgrund der höheren Reisezeitkosten, die infolge der verlängerten Wartezeit entstehen, nicht. \\
Ausser in der vierten Parameterauswahl der Sensitivitätsanalyse, in der die zu erwartende Reisezeit verändert wird, kann die Variante 3, aufgrund der höheren Gesamtkosten, als optimale Variante ausgeschlossen werden. Somit lohnen sich die Mehrkosten der Variante 3 nicht.

Anzumerken ist, dass sich die Ergebnisse bei der Berücksichtigung weiterer Interessengruppen sowie einer komplexeren Modellierung der Einflussfaktoren auf das DTV, verändern könnten. Aufgrund der begrenzten Möglichkeiten, die mir im Rahmen dieser Projektarbeit zur Beurteilung der Situation zur Verfügung standen, konnte ich nicht alle relevanten Faktoren in der Entscheidungsfindung berücksichtigen. Das sollte im Rahmen einer erweiterten Studie zur Situation am Bahnübergang Brunnenstrasse erfolgen.

Um die Situation am Bahnübergang vollständig abzubilden, müssen die Fussgänger und der ÖV miteinbezogen werden. Insbesondere der ÖV hat grosses Interesse an der zukünftigen Entwicklung des Bahnübergangs, da alle Busse, um die Gleisanalage zu queren, den Bahnübergang Brunnenstrasse nutzen. Zusätzlich müsste, um die effektive Verkehrssituation vor Ort in den Entscheidungsprozess miteinzubeziehen, eine Verkehrssimulation durchgeführt werden. So könnte im Fall der Variante 3 die entstehende Wartezeit aufgrund des Ampelsystems exakter bestimmt werden sowie könnten die Variante auf ihre Gebrauchstauglichkeit untersucht werden. 

Zu Beginn des von mir durchgeführten Problemlösungsprozesses, habe ich zur Vereinfachung die beteiligten Personen auf die in Abschnitt \ref{subsec:Gruppen} definierten Interessensgruppen reduziert, um im gegeben Zeitrahmen eine ansprechende Lösung erarbeiten zu können. 
Eine Aufteilung der Kosten nach den verschiedenen Interessensverbänden wäre im Rahmen der Diskussion sehr aufschlussreich gewesen. Da die die Situation und die Möglichkeiten die mir für die Modellierung zur Verfügung standen zur Folge haben, dass die Reisezeitkosten den gesamten Risikovergleich der Varianten bestimmten, habe ich im Rahmen der Diskussion der Ergebnisse auf diese Unterteilung verzichtet. Es wäre jedoch sehr interessant, im Rahmen der weiteren Prüfung der Varianten, eine vertieftere Beurteilung der Nutzer- und Besitzerkosten durchzuführen.

Hinsichtlich der zukünftigen Entwicklung der Mobilität wird das Fahrrad voraussichtlich auch auf Strecken bis zu 30km eine entscheidende Rolle bei der Verkehrsmittelwahl spielen. Aufgrund dessen und in Anbetracht der Bestrebungen der Stadt Uster, ein Zentrum von regionaler Bedeutung zu werden, ist die Förderung des ökologischen und zukunftsorientierten Langsamverkehrs unerlässlich und insbesondere im Rahmen einer nachhaltigen Stadtentwicklung voranzutreiben. \\
So wäre eine Erweiterung der geplanten Veloinfrastruktur entlang der Pfäffikerstrasse bis zum Spital sowie entlang der Bahnhofstrasse bis zum Sternenplatz im Rahmen einer vertiefen Untersuchung zu berücksichtigen. Eine mögliche Variante einer solchen Veloschnellroute könnte anhand des Beispiels der Radschnellwege von Kopenhagen erstellt werden, die sich in Dänemark bereits bewährt und das Pendeln mit dem Velo leichter und sicherer gemacht haben.  \\
Insbesondere unter Anbetracht der aktuellen Situation mit CoVid-19, in der das Velo eine Renaissance erlebt, muss der Ausbau und die Förderung der innerstädtischen Langsamverkehrsnetze vorangetrieben werden.

Nichtsdestotrotz ist, um eine nachhaltige Verbesserung der Verkehrslage in Uster zu erreichen, eine Berücksichtigung der weiteren Schwachstellen des Ustemer Verkehrsnetzes unerlässlich. Insbesondere die zukünftige Entwicklung des Nüsslikreisel sowie der Zürichstrasse haben einen grossen Einfluss auf die Stadtentwicklung. Hinzu kommt, dass die Optimierung eines Teilstücks einer bestehenden Infrastruktur nur dann erfolgen sollte, wenn die Einflüsse auf die umliegenden Systeme berücksichtigt werden. So muss, um Uster nachhaltig zu verbessern, nicht nur das Velonetz ausgebaut sondern das gesamte Verkehrskonzept der Stadt überarbeitet werden. 

\paragraph{Ausblick}

Die von mir vorgeschlagenen Schritte, die Uster in der nächsten Zeit tätigen soll, sind in der nachfolgenden Abbildung \ref{img:UsterFuture} dargestellt. \\
So ergibt sich einerseits, wie ausführlich besprochen, der Weg, die Variante 2 in Betracht zu ziehen und die weiteren Schritte dementsprechend einzuleiten, wie zum Beispiel der Beginn einer Detailstudie oder aber bereits der Start des Planungsprozesses. Der zweite Weg der sich der Stadt Uster anbietet, ist eine  Hauptstudie, wodurch infolge der ausführlichen Prüfung der Situation, entweder die Wahl der optimalen Variante bestätigt oder aber eine andere Variante als optimal bestimmt wird. Die letzte Möglichkeit, die sich der Stadt Uster bietet, ist die Option: Variante 1 - "nichts zu machen", wobei für relativ wenig Geld die Verkehrssicherheit marginal erhöht jedoch über einen Zeitraum von vierzig Jahren hohe Mehrkosten generiert wird.

\begin{figure}[h!]
	\centering
	\includegraphics[width=\textwidth]{figures/f-07-01-EntscheidungsprozessUster}
	\caption[Entscheidungsprozess der Stadt Uster]{Möglicher Entscheidungsprozess der Stadt Uster für die nähere Zukunft}
	\label{img:UsterFuture}
\end{figure} 






% ===========================================================================
% EOF
%

%%% Local Variables:
%%% mode: latex
%%% TeX-master: "../main"
%%% End:



%___________________________________


% Reference
% ---------
% Style for Germen text
%\bibliographystyle{chicago-german}

% Style for Englisch text
%\bibliographystyle{chicago}

%\bibliography{literature}

\newpage
\printbibliography[heading=bibintoc, title=Literaturverzeichnis]

%----Glossar--------

\cleardoublepage
\addcontentsline{toc}{chapter}{Glossar}
\input{content/Glossar}


% Appendix
% --------
\appendix
\addchap{Anhang}
\label{chap:Anhang}
\refstepcounter{chapter}

\section{DTV Modellierung}
\label{subsec:DTVModellierung}

\begin{figure}[h!]
	\centering
	\includegraphics[width=\textwidth]{figures/Anhang/f-00-10-01-DTV-Modellierung}
	\caption{Berechnung des DTV SB1/SU1-3}
	\label{img:DTVModellierung}
\end{figure}

\begin{figure}[h!]
	\centering
	\includegraphics[width=\textwidth]{figures/Anhang/f-00-10-02-DTV-Modellierung}
	\caption{Berechnung des DTV SB2/SU1-3}
\end{figure}

\begin{figure}[h!]
	\centering
	\includegraphics[width=\textwidth]{figures/Anhang/f-00-10-03-DTV-Modellierung}
	\caption{Berechnung des DTV SB3/SU1-3}
	\label{img:DTVModellierung}
\end{figure}


\section{Kostenberechung der Varianten}
\label{sec:Ahangkostenberechnug}

%=============================================================================
% Thesis Template in LaTex
%
% File:  Anhang.tex -- Anhang
% Author(s): Cyrano Golliez <golliezc@student.ethz.ch>
%
% Creation:  27 Jan 2014
% Time-stamp: <Tue 2013-08-13 20:14 juergen>
%
% Copyright (c) 2014 Infrastructure Management Group (IMG)
%               http://ibi.ethz.ch
%
% More information on LaTeX: http://www.latex-project.org/
%=============================================================================


Hier werden alle Fotos und Dokumenten eingefügt.


Mainly Screenshots der Excell-Files

% ===========================================================================
% EOF
%

%%% Local Variables:
%%% mode: latex
%%% TeX-master: "../main"
%%% End:


% Index
% -----
\printindex

\end{document}

%=============================================================================
%%% Local Variables:
%%% mode: latex
%%% TeX-master: t
%%% End: