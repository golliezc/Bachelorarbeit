%=============================================================================
% Thesis Template in LaTex
%
% File:  t-05-01-IsingModel.tex -- Table for the Ising
% Author(s): Juergen Hackl <hackl@ibi.baug.ethz.ch>
%            Clemens Kielhauser <kielhauser@ibi.baug.ethz.ch>
%
% Creation:  27 Jan 2014
% Time-stamp: <Tue 2013-08-13 20:14 juergen>
%
% Copyright (c) 2014 Infrastructure Management Group (IMG)
%               http://ibi.ethz.ch
%
% More information on LaTeX: http://www.latex-project.org/
%=============================================================================

\begin{table}[hbt!]
\center
%\small\renewcommand{\arraystretch}{1.2} 
%
%
\begin{tabular}{@{}p{2.6cm} p{3.3cm} p{3.3cm} p{3.3cm}@{}} \\   
\toprule
\textbf{Fahrezugtyp\,j} & \textbf{Unfalltyp\,a} & \textbf{Unfalltyp\,b} & \textbf{Unfalltyp\,c} \\
\midrule
MIV      & \(1.317\,\mathrm{10^{-6}}\)  & \(9.116\,\mathrm{10^{-8}}\)  & \(4.7243\,\mathrm{10^{-9}}\)  \\
Velo	 & \(3.818\,\mathrm{10^{-6}}\)  & \(2.643\,\mathrm{10^{-7}}\)  & \(1.37\,\mathrm{10^{-8}}\)   \\

\bottomrule

\end{tabular}
\caption[Tabelle der Unfallrisiken]{Tabelle der Unfallrisiken $\gamma_{j,n}\,\Bigl[\frac{Unf"alle_{j,n}}{Pkm_{k}}\Bigl]$}
\label{tab:t-06-01-Unfallrisiko}
\end{table}


%=============================================================================
% EOF
%

%%% Local Variables:
%%% mode: latex
%%% TeX-master: "../guidelines"
%%% End:

