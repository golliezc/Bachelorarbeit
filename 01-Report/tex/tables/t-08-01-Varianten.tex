%=============================================================================
% Thesis Template in LaTex
%
% File:  t-05-01-IsingModel.tex -- Table for the Ising
% Author(s): Juergen Hackl <hackl@ibi.baug.ethz.ch>
%            Clemens Kielhauser <kielhauser@ibi.baug.ethz.ch>
%
% Creation:  27 Jan 2014
% Time-stamp: <Tue 2013-08-13 20:14 juergen>
%
% Copyright (c) 2014 Infrastructure Management Group (IMG)
%               http://ibi.ethz.ch
%
% More information on LaTeX: http://www.latex-project.org/
%=============================================================================
%\small\renewcommand{\arraystretch}{1.2} 
%


\begin{table}[h!]
{\setstretch{0.6}
%\renewcommand{\arraystretch}{1.4}
\flushleft
\begin{tabular}{@{}p{5cm} p{2.5cm} p{2.5cm} p{2.5cm}@{}} \\   
\toprule 		
\textbf{Eigenschaften}  				   								&\textbf{Variante\,1}  & \textbf{Variante\,2} & \textbf{Variante\,3}   \\			
\midrule 
Länge Fahrbahn ($m$)         	 		   								& 80                    & 80    			   & 80             	\\
Länge Unterführung ($m$)       	 		   								& -                     & 55    			   & 65             	\\
Velospuren:					   											&  2				    &  2				   &  4         		\\
Fahrbahnen									 		   					&  2				    &  2				   &  1         		\vspace*{0.25mm} \\
Breite eines Veloweg ($m$):				   								&  1.5				    &  1.5				   &  2         		\\
Breite einer Fahrbahn ($m$):			 		   					    &  3.5				    &  3				   &  5         		\vspace*{0.25mm} \\
Tempolimit	($\frac{km}{h}$) 		   						    		& 50				    & 30				   & 30                	\\
\underline{$\varnothing$\,Geschwindigkeit} ($\frac{km}{h}$) 			&       	            &   				   &               		 \\
\hspace*{5mm}\textbullet\, Velo            		       					& 15  					& 15    			   & 20      			\\
\hspace*{5mm}\textbullet\, MIV            		       					& 37  					& 30    			   & 30      			\vspace*{0.25mm} \\
\underline{Kapazität} ($\frac{Fahrzeug}{h}$)		        			&    				    &  				       &                  	 \\
\hspace*{5mm}\textbullet\, Veloweg            	       					& 3350 					& 3767    			   & 4600      			\\
\hspace*{5mm}\textbullet\, Strasse         		       					& 2500 					& 2500    			   & 1250      			\\
\bottomrule

\end{tabular}
\caption{Basis Informationen der Varianten}
\label{tab:t-08-01-Varianten}
}
\end{table}


%=============================================================================
% EOF
%

%%% Local Variables:
%%% mode: latex
%%% TeX-master: "../guidelines"
%%% End:

